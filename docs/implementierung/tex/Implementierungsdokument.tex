% !TEX encoding = UTF-8 Unicode
\documentclass[parskip=full,11pt,twoside]{scrartcl}
\usepackage[utf8]{inputenc}

% section numbers in margins:
\renewcommand\sectionlinesformat[4]{\makebox[0pt][r]{#3}#4}

% header & footer
\usepackage{scrlayer-scrpage}
\lofoot{\today}
\refoot{\today}
\pagestyle{scrheadings}
\usepackage[T1]{fontenc}
\usepackage[sfdefault]{roboto}
\usepackage[german]{babel}
\usepackage[yyyymmdd]{datetime} % must be after babel
\renewcommand{\dateseparator}{-} % ISO8601 date format
\usepackage{hyperref}
\usepackage[nameinlink]{cleveref}
\crefname{figure}{Abb}{Abb}
\usepackage[section]{placeins}
\usepackage{xcolor}
\usepackage{graphicx}
\hypersetup{
	pdftitle={Implementierungsdokumentation},
	bookmarks=true,
}
\usepackage{csquotes}
\usepackage{upgreek}

\usepackage{amsmath} % for $\text{}$
\newcommand\urlpart[2]{$\underbrace{\text{\texttt{#1}}}_{\text{#2}}$}


\begin{titlepage}

\subject{Implementierungsdokumentation}
\title{\textbf{$\uplambda$}urora}
\subtitle{The Lambda Calculus IDE}


\author{Iuliia Patrusheva, Alexander von Heyden\\
Younis Bensalah, Max Nowak\\
Nikolai Polley, Randy Seng}

\end{titlepage}




\begin{document}
\maketitle
\tableofcontents
\newpage
%=====================EINLEITUNG=======================
\section{Einleitung}
Die Implementierung war die dritte Phase unseres PSE-Projekts und streckte sich über vier Wochen.
Im Folgenden erläutern wir, wie aus dem sehr guten Entwurf der vorherigen Phase ein lauffähiger Prototyp unserer Webanwendung
entstanden ist, welche technischen Hürden wir dabei überwinden mussten und welche organisatorischen Entscheidungen
dabei getroffen wurden.
\newline
\newpage

%=====================ÄNDERUNGEN=====================
\section{Änderungen zum Entwurfsdokument}
Der Entwurf aus der vorherigen Phase hat sich als sehr zuverlässig und gut umsetzbar erwiesen.
Dennoch haben wir an einigen Stellen kleine Abänderungen vorgenommen, und ihn teilweise leicht erweitert.
Im Speziellen handelt es sich hierbei um folgendes:
\begin{itemize}
    \item CodeMirrorPanel (pkg: aurora.client.view.editor)
    \newline
    Ursprünglich war geplant, einen vorhandenen CodeMirror GWT Wrapper als dependency in Aurora einzubinden.
    Leider hat es sich herausgestellt, dass der Wrapper nicht nur auf einer alten Version von CodeMirror beruht, sondern scheinbar auch einen Bug hatte, der es uns erschwert hat, ihn ohne Fehler zu benutzen.
    Deswegen haben wir uns dazu entschieden, einen eigenen Wrapper für CodeMirror zu schreiben.
    Bei Bedarf könnte er wieder ausgegliedert werden, uns als dependency eingebunden werden.

	\item Umbennennung der Klasse (aurora.backend.tree.Libraryterm)
	\newline
	Zuallererst wurde diese Klasse von Libraryterm in Function umbenannt. Wir fanden es unschön, dass alle Klassen in
	diesem Package von Term erben, aber bei dieser Klasse dies extra dazustand. Auch ist für den abstrakten Syntaxbaum
	nicht relevant, ob die Funktion aus einer Library oder etwas anderem kommt. Deswegen wurde diese Klasse in Function
	umbenannt.

	\item Hinzufügung eines Attributs der Klasse (aurora.backend.tree.Function)
	\newline
	Viele Visitoren müssen den Funktionsnamen mit den Bibliotheken abgleichen und nachschauen, ob der Name der Funktion
	vorhanden ist und dann den Term bekommen, der mit dem Namen assoziiert ist. Da die Visitoren die Bibliotheken und
	den Presenter nicht kennen, müsste der 	Presenter jedem einzelnen Visitor die Bibliotheken übergeben. Da der
	Presenter auch immer nur die aktuellste Bibliothek kennt, würde es zu Schwierigkeiten kommen, wenn sich die
	Bibliothek während die Visitoren executen ändert.
	Deshalb werden jetzt schon beim Parsen den Funktionen ihre Terme, die sie repräsentieren, vom Presenter übergeben.
	Diese werden in Function.term gespeichert. Danach kann nur erneutes Parsen, welches einen neuen abstrakten
	Syntaxbaum aufbauen würde, die Funktionsassoziation ändern.

    \item Umbenennung der Klasse (aurora.backend.library.Library)
    \newline
     Library wurde in HashLibrary umbenannt, dann ein Interface namens Library extrahiert.
Es wurde MultiLibrary eingeführt, welche die Abrfrage von LibraryItems aus mehreren Libraries erleichtern soll.
Jetzt muss z.B. der LambdaParser nur eine Library als Konstruktor-Parameter annehmen, hinter der sich der Benutzer- und Standardbibliotheken verbergen.
So können in Zukunft mehr Bibliotheken hinzugefügt werden, ohne den LambdaParser zu verändern.
    \newline

    \item Übergebung der Indizes
    \newline
    Alle Share-Events übergeben jetzt den Step index anstatt ein HighligtedLambdaExpression. Auch wurden die Listen von HighlightedLambdaExpression aus den Share-Events entfernt, da der Presenter diese Informationen bereits hat. Die View soll keine Liste von Steps speichern, nur ihre GWT-spezifische Darstellung.

    \item Dem SidebarDisplay wurden mehrere neue Funktionen zur Behandlung von Fehlern hinzugefügt, die beim Hinzufügen einer neuen Benutzerbibliotheksfunktion entstehen können:
    	\begin{itemize}
        	\item \texttt{displayAddLibraryItemSyntaxError} und \newline
        	\texttt{displayAddLibraryItemSemanticError} wurden hinzugefügt und werden aufgerufen wenn LambdaLexer oder LambdaParser fehlschlägt, und dienen dazu, dem Benutzer den Fehler anzuzeigen. Es werden Objekte übergeben die detailierte Informationen enthalten was und wo genau der Fehler ist.
        	\item \texttt{displayAddLibraryItemNameAlreadyTaken} wurde hinzugefügt und wird aufgerufen wenn es den gewünschten Funktionsnamen bereits in der Standard- oder Benutzerbibliotk gibt. Es soll dem Benutzer eine Fehlermeldung angezeigt werden.
            \item \texttt{removeStandardLibraryItem} und \newline
            	\texttt{removeUserLibraryItem} wurden entfernt. Diese Funktionen werden nicht gebraucht, da die View den Presenter über ein Event aufrufen soll, und diese Information sowieso schon hat.
        \end{itemize}
    \item Der EditorDisplay wurde auch an das moderne Zeitalter angepasst:
    	\begin{itemize}
    		\item \texttt{displaySyntaxError(String)} wurde
            durch \newline \texttt{displaySyntaxError(SyntaxException)} und \newline
        	\texttt{displaySemanticError(SemanticException)} ersetzt und werden aufgerufen wenn LambdaLexer oder LambdaParser fehlschlägt, und dienen dazu, dem Benutzer den Fehler anzuzeigen. Es werden Objekte übergeben die detailierte Informationen enthalten was und wo genau der Fehler ist.
            \item \texttt{addNextStep} verfügt jetzt über eine Liste von HighlightedLambdaExpressions. Grund ist, dass sehr oft mehr als ein Schritt auf ein mal ausgeführt werden. Es wird auch der start-index des ersten Schrittes in der übergebenen Liste übergeben, damit die View die korrekten Schritt Inidizes anzeigen kann.
            \item \texttt{finishedFinished} Ist ein Äquivalent zum displayResult, und wird aufgerufen wenn der zweite Durchlauf beendet ist. Mit zweiter Durchlauf ist gemeint, wenn der Benutzer bereits den fertig reduzierten Term in der Ausgabe stehen hat, aber sich (aus Lernzwecken) die Zwischenschritte nacheinander anzeigen lassen will. Wenn es keinen weiteren Schritt zum anzeigen gibt, wird diese Funktion aufgerufen.
    	\end{itemize}

    \item Neu hinzugefügte Events:
    	\begin{itemize}
    		\item
    	\end{itemize}
\end{itemize}



%=============IMPLEMENTIERUNGSPROZESS=====================
\section{Implementierungsprozess}
\subsection{Richtlinien}
Bevor wir mit der Implementierung begonnen haben, haben wir uns gemeinsam dafür entschieden,
wärend des gesamten Prozesses einige grundlegende Regeln einzuhalten.
Hiervon haben wir uns einen saubereren Implementierungsprozess erhofft.
Wichtig waren vorallem folgende Regeln:
\begin{itemize}
	\item Alle haben die benötigten Tools korrekt installiert
	\newline
	Am ersten Treffen mussten alle Teammitglieder die benötigten Tools installieren und testen ob sie funktionieren.
	Alle Teammitglieder mussten ein Mergetool installiert haben, weil es bereits Probleme im Entwurf wegen fehlerhaften Merges gegeben hatte.
	Auch mussten die Teammitglieder alle Funktionen auf der Kommandozeile zusätzlich 	zu der IDE ausführen können. Dies wurde vorrausgesetzt, um mögliche Fehler von den IDEs ausschließten zu können.
	Tools wie Ant, Ivy, Codemirror, Checkstyle, mussten zusätzlich installiert und eingerichtet werden.
	Natürlich musste auch GWT installiert und in die IDE eingebunden werden.

    \item Protected master
        \newline
        Der git master branch wurde so konfiguriert, dass nur noch die Phasenverantwortlichen andere Branches in den master mergen konnten.
        Auf dem master selber durfte nicht geschrieben werde.
    \item Checkstyle
    \newline
    Wir haben uns für die Checkstyle-Regeln von Google entschieden. Nur die maximale Linienlänge haben wir von 100 auf 120 Zeichen erhöht, da wir alle mit Widescreen Monitoren arbeiten und entwickeln. Die Checkstyle-Regeln waren Pflicht und mussten von allen Gruppenmitgliedern eingehalten werden.
    \item Git CI Runner
        \newline
        In unserem git Repository haben wir einen CI Runner eingerichtet, der für alle Commits Aurora kompiliert hat, unsere Testfälle durchgegangen ist, sowie Checkstyle überprüft hat.
        So konnten wir sicherstellen, dass nur guter Code in den master Branch gemerged wurde.

    \item Continuous Deployment
        \newline
        Auf einem externen Server hatten wir durchgehend den aktuellen master Commit bereitgestellt.
        Zum Einen hat uns dies erlaubt, Aurora ohne großen Aufwand auf verschiedenen Endgeräten zu testen.
        Zum Anderen haben wir dadurch gewährlsiten können, dass unser Code nicht nur lokal sondern auch von einem Server geladen ohne Fehler ausgeführt werden kann.
		Dazu wurde der CI Runner dahingehened angepasst, dass jeder Commit auf dem master Branch zunächst als WAR-Datei
		gebaut und anschließend an den Deployment-Server gesendet wurde.
		Außerdem wurde ein Skript geschrieben, welches auf dem Deployment-Server ausgeführt wurde und hochgeladene
		WAR-Dateien auf einem öffentlich zugänglichen HTTP-Server entpackt und gehostet hat.

       \item JUnit Tests
       \newline
      	Es wurde von den Backend-Entwicklern, die komplett in Java programmieren, JUnit-Tests zu schreiben.
      	Es wurde zwar keine Testüberdeckung vorrausgesetzt, aber man musste kontrollieren, dass der Code bei normaler Bedienung fehlerfrei funktioniert.
      	Die Algorithmen, die zum Parsen und zum Reduzieren benutzt werden, mussten auf Implementierungsfehler übreprüft werden, damit korrekte Ergebnise beim berechnen eines Lambda-Terms berechnet werden.
      	Da im Frontend keine schnellen JUnit-Tests verwendet werden konnten, sollten die Entwickler von Hand testen, ob sich die Objekte verhalten, wie man es erwartet.
      	Falls benötigt, mussten Klassen mit Mockito gemocket werden.
      	\item Issues
        \newline
      	Bugs und Anomalien im Code unserer Teammitglieder wurden stets in Form von Issues auf GitLab kommuniziert.
      	\end{itemize}

\subsection{Implementierungsplan}
Nachdem wir uns darauf geeinigt hatten, wie der Implementierungsprozess ablaufen sollte, mussten wir uns nun dafür entscheiden, was von wem implementiert werden musste.
Hierzu haben wir uns an die in der Entwurfsphase erstellte Struktur gehalten, und Pakete einzelnen Personen zugewiesen.
Wir haben versucht abzuschätzen wie viel Aufwand die jeweiligen Pakete darstellen und haben so verteilt, dass möglicht alle Gruppenmitglieder gleich viel Arbeit zu verrichten hatten.
Die Personen hatten bei ihren zugeteilten Paketen auch beim Entwurf mitgearbeitet, damit Wissen aus dem Entwurf nicht verloren gehen konnte.
Kleinere Pakete wurden direkt einer Person zugeteilt, wohingegen größere Pakete immer noch ein weiterer \enquote{Helfer} zugeteilt wurde.
Die Personen haben dann jeweils in ihren Pakten Prioritäten festgelegt, um eine Reihenfolge der Implementierung festzulegen.

Es mussten hierbei natürlich auch die Abhängigkeiten der Pakete beachtet werden, damit man die Pakete möglichst schnell testen kann.


\subsubsection{Verteilung der Pakete und deren Priorität}
Im Folgenden ist das Ergebnis unsere Paketeverteilung zu sehen.
\begin{description}


    \item [Alexander]\hfill \\
    \begin{enumerate}
        \item CodeMirror
            \newline
            Zur Darstellung von Code haben wir uns entschieden, CodeMirror zu benutzen.
            Hierzu war geplant, einen bereits vorhandenen CodeMirror Wrapper in unser Projekt einzubinden.
        \item Highlighting
            \newline
            CodeMirror ermöglicht durch eine Schnittstelle statisches Syntaxhighlighting.
            Zur Darstellung dynamischen Syntaxhighlightings reicht diese nicht aus, deswegen benutzen wir hier die HighlightedLambdaExpressions.
        \item Editor (pkg: aurora.client.view.editor)
            \newline
            Den größten Teil der graphischen Oberfläche wird vom EditorPanel eingenommen.
            Das EditorPanel selbst beinhaltet mehrere CodeMirrorPanel, die ihrer Aufgaben entsprechend konfiguriert werden müssen.
        \item Encoder (pkg: aurora.backend.encoders)
            \newline
            Encoder ermöglochen es uns, vorhandenen Input und Bibliotheken zu serialisieren, und wieder einzulesen.
            Hier existiert unter anderem die Kombination eines JSON und Pastebin Encoders, der ein erstelltes JSON Objekt auf Pastebin speichert.
    \end{enumerate}
    \item [Nikolai]\hfill \\
    \begin{enumerate}
    \item Churchnummern in Abstraktionsgestalt von den Integers in denen sie gespeichert sind generieren. \\(class aurora.backend.tree.Churchnumber)
    \item  Die Libraryfunktionen implementieren \\(pkg: aurora.backend.library)
    \item Die Redexpathklasse füllen \\class: (aurora.backend.RedexPath)
    \item Mit der fertigen RedexPathklasse die Normalorder implementieren. Die Normalorder erstellt als Ergebnis einen Redexpath. \\(class: aurora.backend.betareduction.strategies.NormalOrder)
    \item Wenn die Normalorder fertig ist das Betareducerpackage implementieren und dann mit der fertigen Normalorder testen.
    \\ (pkg: backend.betareduction)
    \item Wenn dies klappt werden Call by Value und Call by Name implementiert  \\(pkg: aurora.backend.betareduction.strategies)
\item Dannach wird die User Strategy implementiert \\(pkg: aurora.backend.betareduction.strategies)

\item Die ShareLatex klasse muss zuletzt implementiert werden\\(class: aurora.backend.ShareLatex)
    \end{enumerate}
Diese Reihenfolge wurde gewählt, damit möglichst alles gleich testbar war und deshalb wurden zuerst die Abhängigkeiten implementiert.
    \item [Iuliia]\hfill \\


    \begin{enumerate}


    \item Sprachauswahl ermöglichen.  \\(pkg aurora.client.*)

    \begin{itemize}

	\item[--] Verknüpfungen der UiBinder Labels, die beim Sprachwechsel übersetzt werden sollen

	\item[--] Erzeugung der gesamten Sprachbibliothek für jede Sprache (2 Sprachen)

	\item[--] Anlegen einer Struktur für die Sprachdokumente (pkg aurora.client.*)

    \end{itemize}

    \item Für den Sprachwechsel den Menü Button implementieren \\(pkg: aurora.client.*)

    \item Frontend styling

    \begin{itemize}

	\item[--] Anpassen der GWT UiBinder Struktur für die css Styling Zwecke

	\item[--] CSS Dateien mit entsprechendem Styling (anhand des GUI Entwurfs der 1.Phase)

    \end{itemize}
    
    \end{enumerate}
    
    \item [Max]\hfill \\
    	\begin{itemize}
        	\item Behandlung von Events vom EventBus im Presenter, also:
            \begin{itemize}
            	\item Sharing events.
                \item Run, Pause, Continue, Reset, Step.
                \item Events die Zustand wichtig für die Ausführung speichern (Anzahl der auf einmal auszuführenden Schritte).
            \end{itemize}
            \item Implementation aller möglichen Presenter.
        \end{itemize}
    \item [Younis]\hfill \\
	\begin{enumerate}
		\item Lexer
		\item Parser
		\item Ehhh... Rudolf?
	\end{enumerate}

    \item [Randy]\hfill \\
    \begin{enumerate}
    \item Einen Bug beheben, der beim Starten der Aurora WebApp, nicht die GUI, sondern eine blanke Seite anzeigt. Der Bug muss zuerst behoben werden, da sonst alle, die im Frontend, ihre Ergebnisse nicht einsehen können. Zuerst wurde vermutet, dass der Fehler sich im Package aurora.client.view befindet. Letzendlich hatte sich die Vermutung als falsch rausgestellt und der Fehler musste in der Klasse aurora.client.Aurora behoben werden. \\(pkg: aurora.client.Aurora)
    \item Die Aufgabe ist es, sich für die Benutzung von GWTTestCase zu informieren.
    \item Die State Machine in AuroraView implementieren. Die State Machine konnte unabhängig von allen anderen Klassen implementiert werden. \\(class: aurora.client.view.AuroraView)
    \item Aushilfe beim Presenter. Vor allem war der Schwerpunkt darauf gelegt, sich um das Event Wiring auf der View Seite zu kümmern.
    \end{enumerate}
\end{description}



\subsection{Änderungen des Implementierungsplans während der Implementierung}

\begin{description}

\item [Alexander]\hfill \\
Der erste Punkt des Implementierungsplans, CodeMirror, hat sich als wesentlich schwieriger als erwartet herausgestellt.
Anstatt wir urspünglich geplant eine Dependency einzubiden, wurde extra Zeit damit verbracht, einen eigenen Wrapper zu schreiben.
Dies hatte zur Folge, dass sich der Zeitplan stark nach hinten verschoben hat.
Abgesehen von dieser Verschiebung sind keine weiteren Abweichungen aufgetreten.
\item [Nikolai]\hfill \\
Großteilig wurde sich an den Plan gehalten, es wurde nur Call By Name und Call by Value vor dem Betareducer geschrieben.
Es wurde  bemerkt, dass das Prinzip dieser Reduktionsstrategien sehr ähnlich zu der Normalenordnung war und zu dem Zeitpunkt es passender erschien diese zu implementieren als in ein komplett neues Thema einzusteigen.
Nach dem Betareduzierer wurde noch den Comparer geschrieben, der für einfacheres debuggen geeignet war. Ob er sich auch für den Simplifier eignet muss sich in der Qualitätssicherung zeigen.
Da bei der User Strategy Implemntierung des Frontends, Presenters und Backends benötigt wird und andere Teammitglieder, vitalere Dinge zu erledigen hatten
wurde die Implementierung dieser Reduktionsstrategie in die Qualitätssicherung verschoben.
Dannach wurde von Younis Aufgaben die Klasse HighlightedLambdaExpression gefüllt und  Highlightedlambdaexpression.toString(t) geschrieben, da die Aufgabenverteilung nicht ganz fair gelungen war, da unterschätzt worden war, wie aufwändig die Toolchain einzurichten war.

\item [Iuliia]\hfill \\
\item [Max]\hfill \\
\item [Younis]\hfill \\
Die Konfiguration des Buildsystems (Ant) und des CI Runners hat wesentlich mehr Zeit in Ansprung genommen
als ursprünglich geplant, was dazu geführt hat, dass sich die Implementierung des Lexers und Parsers verzögert hat.

Desweiteren wurde zusätzlich ein Continuous Deployment-Server eingerichtet.
\item [Randy]\hfill \\
Es wurde sich im großen und ganzen am Implementierungsplan gehalten. Anzumerken ist es, dass die Aushilfe beim Presenter, dessen Schwerpunkt auf das EventWiring lag, nun gegen
Ende der Implementierungsphase etwas genauer beschrieben werden kann. Zuerst wurde die SidebarView und dann die EditorView mit dem EventBus gekoppelt.
Letztendlich war es die Aufgabe, die implementierte StateMachine mit der AuroraView zu koppeln.

\end{description}

\subsubsection{Unerwartete Probleme}
Während der Implementierung wurden nur wenige wirklich unerwartete Probleme gefunden, welche aber dafür um so mehr Zeit gekostet haben, um wirklich vollends behoben werden zu können:
\begin{itemize}
    \item ant (Buildsystem)
        \newline
        \textbf{Das sollte vllt. am besten Younis schreiben :)}
    \item Continuous Deployment
        \newline
        \textbf{Das vllt. auch...}
    \item Parser
        \newline
        \textbf{Younis?}
    \item CodeMirror
        \newline
        \textbf{Gerade kein Bock, schreib ich später. Alex}
    \item BetaReducer
    \newline
Es war bereits im Entwurf ein Prototyp eines Betareduzieres geschrieben worden und er hatte, mit etwas hardcoding, ein richtiges Ergebnis für eine Beispielrechnung geliefert.
Dass dieses richtige Ergebnis purer Zufall war und dass der Betareduzierer falsch gerechnet hatte musste recht lange debuggt werden.
Als das Problem stellte sich am Ende ein falscher Shift der DebruijnIndizes heraus, die nur bei manchen Reduktionen relevant waren und der Bug deswegen gut versteckt war.

    \item Sprachauswahl (i18n)
    \newline
GWT liefert eine Möglichkeit die i18n Lokalisierung zu verwenden. Pro Klasse, in der es eine Nachricht gibt, die Übersetzt werden soll, wird eine Properties Datei mit Hashwerten und entsprechenden Nachrichten erzeugt. Nachdem man alle i18n Properties Dateien erzeugt hat, fügt man  sie in einer umfassenden Protpertiesdatei zusammen, um die Nachrichten schnell wieder anpassen zu können, ohne die Dateien einzeln durchgehen zu müssen. Für das GWT ist es aber wichtig, dass die endgültige Properties Dateien in einem bestimmten Paket liegen müssen (nämlich aurora.com.google.gwt.i18n.client).

Das Paket lag in einem Package das zu weit entfernt gewesen wäre, wenn man eine konsistente Struktur erwarten würde.


Um die Projektstruktur doch einheitlich zu haben, gab es Versuche die i18n Properties Dateien in andere Pakete zu legen oder die Dateien neu zu erzeugen. Das Problem konnte nicht gelöst werden - GWT kann die i18n Properties nur in dem oben genanntem Paket sehen.

	\item CSS Styling
    	\newline
	GWT generiert die Struktur des endgültigen Endprojekts so, dass die html Struktur am Ende sehr verschachtelt ist. Dies ergab Schwierigkeiten bei dem Zugriff auf einzelne html Blöcke.

    \item Restepping.
    \newline
    Die genaue interaktion mit dem Benutzer bezüglick Run, Pause, Continue, Reset, Step, ist sehr kompliziert. Beim Pausieren werden nur die paar zuletzt reduzierten Schrrite ausgegeben, und beim Fortfahren werden die Schritte wieder gelöscht.
    \newline
    Man kann sich das Ergebnis anzeigen lassen, und die Zwischenschritte erst später. Dies war recht schwierig umzusetzen im Presenter, und resultierte in Spaghetti code. Es ist eines der ersten Sachen die zu beheben sein werden in der nächsten Phase.

\end{itemize}

\newpage


%=====================FEATURES=====================
\section{Implementierte Kriterien}

\subsection{Musskriterien}
Alle im Pflichtenheft definierten Musskriterien sind in der aktuellen Version von Aurora umgesetzt worden.

\subsection{Wunschkriterien}
Neben den Pflichtkriterien wurden in der ersten Phase auch Wunschkriterien definiert.
Im folgenden ist eine Auflistung aller implementierten Wunschkriterien zu sehen.
\newline
\newline
\textbf{K1} Zeilennummer
\newline
\textbf{K2} Kommentare einfügen
\newline
\textbf{K3} Autovervollständigung von Klammern
\newline
\textbf{K4} Syntax-Highlighting
\newline
\textbf{K7} Standardbibliothek
\newline
\textbf{K8} Benutzerbibliothek
\newline
\textbf{K9} Call By Name und Call By Value
\newline
\textbf{K11} Mehrere Schritte auswerten.
\newline

\subsection{Weggefallene Kriterien}
Einige Wunschkriterien sind noch nicht vollständig implementiert, aber es wurden keine Kriterien von der Implementierung ausgeschlossen.
\newpage


%=====================STATISTIKEN=====================
\section{Statistiken}

\subsection{Continuous Integration}
Anzahl Commits: \textasciitilde  1200\newline

Anzahl Lines of code in Java: \textasciitilde 4200
Anzahl Lines of code in XML: 255

Anzahl lines of Code in Tests: \textasciitilde 1300


\subsection{Testabdeckung}
Das Package aurora.backend hat ungefähr 70 \% Codeüberdeckung.
Die Strategien, die einen Redexpath liefern und die Betareduktion wurden ausgiebig getestet.
Der Parser konnte zusätzlich einfach getestet werden, da er als Chatbot in einem Messenger integriert wurde.
Dort konnte die Teammitglieder korrekte und falsche eingaben parsen und schauen,
ob korrekt geparsed wurde.
Es wurden mehr als 127 Messages and den Bot geschickt und dadurch konnten zwei
Fehler des Parsers behoben werden.
Später wurde auch die Fähigkeit des Betareduzierens dem Bot hinzugefügt.
\newpage


\end{document}
