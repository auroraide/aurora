% !TEX encoding = UTF-8 Unicode
\documentclass[parskip=full,11pt,twoside]{scrartcl}
\usepackage[utf8]{inputenc}

% section numbers in margins:
\renewcommand\sectionlinesformat[4]{\makebox[0pt][r]{#3}#4}

% header & footer
\usepackage{scrlayer-scrpage}
\lofoot{\today}
\refoot{\today}
\pagestyle{scrheadings}
\usepackage[T1]{fontenc}
\usepackage[german]{babel}
\usepackage[yyyymmdd]{datetime} % must be after babel
\renewcommand{\dateseparator}{-} % ISO8601 date format
\usepackage{hyperref}
\usepackage[nameinlink]{cleveref}
\crefname{figure}{Abb}{Abb}
\usepackage[section]{placeins}
\usepackage{xcolor}
\usepackage{graphicx}
\hypersetup{
	pdftitle={Implementierungsdokumentation},
	bookmarks=true,
}
\usepackage{csquotes}
\usepackage{upgreek}

\usepackage{amsmath} % for $\text{}$
\newcommand\urlpart[2]{$\underbrace{\text{\texttt{#1}}}_{\text{#2}}$}


\begin{titlepage}

\subject{Implementierungsdokumentation}
\title{\textbf{$\uplambda$}urora}
\subtitle{The Lambda Calculus IDE}


\author{Julia Patrusheva, Alexander von Heyden\\
Younis Bensalah, Max Nowak\\
Nikolai Polley, Randy Seng}

\end{titlepage}




\begin{document}
\maketitle
\tableofcontents
\newpage
%=====================EINLEITUNG=======================
\section{Einleitung}
Die Implementierungsphase.
Der Teil von PSE, in der wir endlich damit anfangen können, unserem Projekt Leben einzuhauchen.
\newline
\textbf{Kann iwann weg}
was ist geplannt, worum geht es, Stand, worum ging es in der IPh
\newpage

%=====================ÄNDERUNGEN=====================
\section{Änderungen zum Entwurfsdokument}
Der Entwurf aus der vorherigen Phase hat sich als sehr zuverlässig und gut umsetzbar erwiesen.
Dennoch haben wir an einigen Stellen kleine Abänderungen vorgenommen, und ihn teilweise leicht erweitert.
Im Speziellen handelt es sich hierbei um folgendes:
\begin{itemize}
    \item CodeMirrorPanel (pkg: aurora.client.view.editor)
    \newline
    Ursprünglich war geplant, einen vorhandenen CodeMirror GWT Wrapper als dependency in Aurora einzubinden.
    Leider hat es sich herausgestellt, dass der Wrapper nicht nur auf einer alten Version von CodeMirror beruht, sondern scheinbar auch einen Bug hatte, der es uns erschwert hat, ihn ohne Fehler zu benutzen.
    Deswegen haben wir uns dazu entschieden, einen eigenen Wrapper für CodeMirror zu schreiben.
    Bei Bedarf könnte er wieder ausgegliedert werden, uns als dependency eingebunden werden.
	
	\item Umbennennung der Klasse aurora.backend.tree.Libraryterm
	\newline
	Zuallererst wurde diese Klasse von Libraryterm in Function umbenannt. Wir fanden es unschön, dass alle Klassen in dieser Package von Term erben, aber bei dieser Klasse dies extra dazustand. Auch ist für den abstrakten Syntaxbaum nicht relevant, ob die Funktion aus einer Library oder etwas anderem kommt. Deswegen wurde diese Klasse in Function umbenannt.
	
	\item Hinzufügung eines Attributs der Klasse aurora.backend.tree.Function
	\newline
	Viele Visitoren müssen den Funktionsnamen mit den Bibliotheken abgleichen und nachschauen, ob der Name der Funktion vorhanden ist und 		dann den Term bekommen, der mit dem Namen assoziiert ist. Da die Visitoren die Bibliotheken und den Presenter nicht kennen, müsste der 	Presenter jedem einzelnen Visitor die Bibliotheken übergeben. Da der Presenter auch immer nur die aktuellste Bibliothek kennt, würde 		es zu Schwierigkeiten kommen, wenn sich die Bibliothek während die Visitoren executen ändert. 
	Deshalb werden jetzt schon beim Parsen den Funktionen ihre Terme, die sie repräsentieren, vom Presenter übergeben.
	Diese werden in Function.term gespeichert. Dannach kann nur ein neues parsen, welches einen neuen abstrakten syntaxbaum aufbauen würde 	die Funktionsassoziation ändern.
	
	
	
	
    \item NEUE EVENTS???
\end{itemize}
\subsection{Kriterien die weggefallen sind (Krierien nummer + ein Grund)}
\newpage

%=============IMPLEMENTIERUNGSPROZESS=====================
\section{Implementierungsprozess}
\subsection{Regeln während der Implementierung}
Bevor wir mit der Implementierung begonnen haben, haben wir uns gemeinsam dafür entschieden, wärend des gesamten Prozesses einige grundlegende Regeln einzuhalten.
Hiervon haben wir uns einen saubereren Implementierungsprozess erhofft.
Wichtig waren vorallem folgende Regeln:
\begin{itemize}
    \item Protected master
        \newline
        Der git master branch wurde so konfiguriert, dass nur noch die Phasenverantwortlichen andere Branches in den master mergen konnten. 
    \item Checkstyle
    \newline
    Wir haben uns für die Checkstyleregeln von Google entschieden. Nur die maximale Linienlänge haben wir von 100 auf 120 Zeichen erhöht, da wir alle mit Widescreen Monitoren arbeiten und entwickeln. Die Checkstyleregeln waren Pflicht und mussten von allen Gruppenmitgliedern eingehalten werden.
    \item Git CI runner
        \newline
        In unserem git Repository haben wir einen CI Runner eingerichtet, der für alle Commits Aurora kompiliert hat, unsere Testfälle durchgegangen ist, sowie Checkstyle überprüft hat.
        So konnten wir sicherstellen, dass nur guter Code in den master Branch gemerged wurde.
    \item Continuous Deployment
        \newline
        Auf einem externen Server hatten wir durchgehend den aktuellen master Commit bereitgestellt.
        Zum Einen hat uns dies erlaubt, Aurora ohne großen Aufwand auf verschiedenen Endgeräten zu testen.
        Zum Anderen haben wir dadurch gewährlsiten können, dass unser Code nicht nur lokal sondern auch von einem Server geladen ohne Fehler ausgeführt werden kann.        
\end{itemize}

\subsection{Ursprünglicher Implementierungsplan}
Nachdem wir uns darauf geeinigt hatten, wie der Implementierungsprozess ablaufen sollte, mussten wir uns nun dafür entscheiden, was von wem implementiert werden musste.
Hierzu haben wir uns an die in der Entwurfsphase erstellte Struktur gehalten, und Pakete einzelnen Personen zugewiesen.
Wir haben versucht abzuschätzen wie viel Aufwand die jeweiligen Pakete darstellen und haben so verteilt, dass möglicht alle Gruppenmitglieder gleich viel Arbeit zu verrichten hatten.
Die Personen hatten bei ihren zugeteilten Paketen auch beim Entwurf mitgearbeitet, damit Wissen aus dem Entwurf nicht verloren gehen konnte.
Kleinere Pakete wurden direkt einer Person zugeteilt, wohingegen größere Pakete immer noch ein weiterer \enquote{Helfer} zugeteilt wurde. 
Die Personen haben dann jeweils in ihren Pakten Prioritäten festgelegt, um eine Reihenfolge der Implementierung festzulegen. 
Es mussten hierbei natürlich auch die Abhängigkeiten der Pakete beachtet werden, damit man die Pakete möglichst schnell testen kann.


\subsubsection{Verteilung der Pakete und deren Priorität}
Im Folgenden ist das Ergebnis unsere Paketeverteilung zu sehen.
\begin{description}


    \item [Alexander]\hfill \\
      wird zu großen Teilen die Darstellung des Editor Bereiches implementieren. 
        Hierzu zählt zu aller erst das Anzeigen von Code, was unter anderem Zeilennummerierung und Syntaxhighlighting beinhaltet.
        Im Anschluss daran wird er den Encoder implentieren, der es erlauben wird, Sessions zu teilen.
    \item [Nikolai]\hfill \\
    \begin{enumerate}
    \item Churchnummern in Abstraktionsgestalt von den Integers in denen sie gespeichert sind generieren. \\(class aurora.backend.tree.Churchnumber)
    \item  Die Libraryfunktionen implementieren \\pkg: aurora.backend.library
    \item Die Redexpathklasse zu füllen \\class: aurora.backend.RedexPath
    \item Mit der fertigen RedexPathklasse die Normalorder implementieren. Die Normalorder erstellt als Ergebnis einen Redexpath. \\class: aurora.backend.betareduction.strategies.NormalOrder
    \item Wenn die Normalorder fertig ist die Betareducer klassepackage implementieren und dann mit der fertigen Normalorder testen. backend.betareduction
    \item Wenn dies klappt werden Call by Value und Call by Name implementiert  \\(pkg: aurora.backend.betareduction.strategies)
\item Dannach wird die User Strategy implementiert \\(pkg: aurora.backend.betareduction.strategies)

\item Die ShareLatex klasse wird implementiert \\(class: aurora.backend.ShareLatex)
    \end{enumerate}
Ich habe diese Reihenfolge gewählt, damit ich möglichst alles gleich testen konnte und deshalb habe ich die Abhängigkeiten zuerst implementiert.
    \item [Iuliia]\hfill \\
    \item [Max]\hfill \\
    \item [Younis]\hfill \\
    \item [Randy]\hfill \\
\end{description}


\subsubsection{Verteilung nach Module, die ungefähr unabhängig von ein ander implementiert werden können}

\subsection{Änderungen des Implementierungsplans während der Implementierung}

\subsubsection{Unerwartete Probleme}
Während der Implementierung sind wir nur auf wenige wirklich unerwartete Probleme gestoßen, die uns dafür aber um so mehr Zeit gekostet haben, um wirklich vollends behoben werden zu können:
\begin{itemize}
    \item ant (Buildsystem)
        \newline
        \textbf{Das sollte vllt. am besten Younis schreiben :)}
    \item Continuous Deployment
        \newline
        \textbf{Das vllt. auch...}
    \item Parser
        \newline
        \textbf{Younis?}
    \item CodeMirror
        \newline
        \textbf{Gerade kein Bock, schreib ich später. Alex}
    \item BetaReducer
    \newline
    Wir hatten im Entwurf bereits ein Prototyp eines Betareduzieres geschrieben und er hatte ein richtiges Ergebnis für eine Beispielrechnung geliefert. Dass dieses richtige Ergebnis purer Zufall war und dass der Betareduzierer falsch gerechnet hatte musste ich recht lange debuggen. Als das Problem stellte sich am Ende ein falscher Shift der DebruijnIndizes heraus, die nur bei manchen Reduktionen relevant sind.
   
\end{itemize}
\subsubsection{Ablauf der Implementierung}
\subsubsection{Was ist geändert und woran liegt es}
\subsubsection{git übersicht vll}
 (Realität und wieso es sich unterscheidet von dem Anfangsplan)
\newpage

%=====================UNIT TESTS=====================
\section{Unit tests}
\subsection{Backend TestZeug}
\subsubsection{Ziel}
\subsubsection{Voraussetzungen}
\subsubsection{Werkzeuge}
\subsection{Frontend TestZeug}
\subsubsection{Ziel}
\subsubsection{Voraussetzungen \& Probleme die auftauchen koennen}
\subsubsection{Werkzeuge}
\newpage

\end{document}
