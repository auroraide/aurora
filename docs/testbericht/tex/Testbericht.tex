% !TEX encoding = UTF-8 Unicode
\documentclass[parskip=full,11pt,twoside]{scrartcl}
\usepackage[utf8]{inputenc}

% section numbers in margins:
\renewcommand\sectionlinesformat[4]{\makebox[0pt][r]{#3}#4}

% header & footer
\usepackage{scrlayer-scrpage}
\lofoot{\today}
\refoot{\today}
\pagestyle{scrheadings}
\usepackage[T1]{fontenc}
\usepackage[sfdefault]{roboto}
\usepackage[german]{babel}
\usepackage[yyyymmdd]{datetime} % must be after babel
\renewcommand{\dateseparator}{-} % ISO8601 date format
\usepackage{hyperref}
\usepackage[nameinlink]{cleveref}
\crefname{figure}{Abb}{Abb}
\usepackage[section]{placeins}
\usepackage{xcolor}
\usepackage{graphicx}
\hypersetup{
pdftitle={Testbericht},
bookmarks=true,
}
\usepackage{csquotes}
\usepackage{upgreek}

\usepackage{amsmath} % for $\text{}$
\newcommand\urlpart[2]{$\underbrace{\text{\texttt{#1}}}_{\text{#2}}$}
\newcommand{\issueref}[1]{
    \href{https://git.scc.kit.edu/ap/Aurora/issues/#1}{(#1)}
}
% Titel GitLab issue number, Symptome, Grund, Behebung.
\newcommand{\regrtest}[5]{
    \subsection{#1 \issueref{#2}}
    \begin{itemize}
        \item \textbf{Symptom}
            #3
            %Hat man einen Term wie z.B. $\$plus\ 2\ 3\ \#\ foo$ eingegeben und klickt auf Step, so wurde der erste Schritt
            %fälschlicherweise als $\$plus\ 2\ 3\ foo$ ausgegeben.
        \item \textbf{Grund}
            #4
            %Javascript RegExp und die Java Regular Expression Bibliotheken unterscheiden sich in der Behandlung von dem
            %Regex-Shorthand \texttt{\textbackslash v}.
        \item \textbf{Behebung}
            #5
            %Der Regex-Shorthand wurde explizit ausgeschrieben so wie es vorgesehen war.
    \end{itemize}
}

\begin{titlepage}

    \subject{Testbericht}
    \title{\textbf{$\uplambda$}urora}
    \subtitle{The Lambda Calculus IDE}


    \author{Iuliia Patrusheva, Alexander von Heyden\\
    Younis Bensalah, Max Nowak\\
    Nikolai Polley, Randy Seng}

\end{titlepage}




\begin{document}
    \maketitle
    \tableofcontents
    \newpage

    \section{Einleitung}


    \section{Testfallszenarien}

    \section{Testfallüberdeckung per Package}
    % Viele Screenshots von den überdeckungsberichtdingers.
    % Vielleich auch Überdeckung mit nur JUnit, und dann noch mit Selenium mit dazu?

    \section{Regression Tests}
    Gesquäschte Bugs: 1337.
    
    \regrtest{Leerer Input in StepNumber TextBox führt zu nicht valide, dargestellten Wert}{133}{
        Lösche den gesamten Input in StepNumber TextBox, so dass diese leer ist. Klicke auf eine beliebige Stelle in der Aurora WebApp.
        Klicke zum Beispiel auf den Input Editor, so dass die StepNumber TextBox den Fokus verliert. Die StepNumber TextBox bleibt leer,
        obwohl sie eigentlich eine valide Schrittzahl anzeigen sollte.
    }{
        Die StepNumber TextBox überprüft den Input nach jeder neuen Eingabe. Falls diese nicht valide ist, wird der letzte valide Wert angenommen.
        Zudem wird auch eine leere TextBox akzeptiert, da auch valide Werte der Länge 1 eingegeben werden sollen. Wenn die TextBox leer ist und
        den Fokus verliert, dann wird nicht der letzte valide Wert angenommen, da kein BlurHandler der StepNumber Textbox registriert wurde, 
        der den Wert wieder auf den letzten validen Wert setzt.
    }{
        Der StepNumber TextBox wird ein BlurHandler hinzugefügt, der den letzten validen Wert der StepNumber TextBox zuweist.
    }
    
    \regrtest{Das Ausführen einer nicht definierten Bibliotheksfunktion sollte einen Fehler anzeigen}{143}{
    	Man gibt einen regulären $\lambda$-Term ein, der eine nicht definierte Bibliotheksfunktion enthält, ein und drückt auf den Run Button
		Es wird keine Fehlermeldung angezeigt.
    }{
    	Das Anzeigen von Fehlern, bei Eingabe eines ungültigen $\lambda$-Terms, 
		wurde noch nicht implementiert.
    }{
    	Implementiere die Funktionen displaySemanticError und displaySyntaxError in der Klasse EditorView.
		Bei Ausführen der Berechnung eines ungültigen $\lambda$-Terms sollte in ein Zustandsübergang von RunningState auf DefaultState erfolgen.
		Deswegen wird jedem Zustand eine neue Kante errorDisplayed hinzugefügt und implementiert.
    }

    \regrtest{Kommentare mitgeparst}{151}{
        Hat man einen Term wie z.B. $\$plus\ 2\ 3\ \#\ foo$ eingegeben und klickt auf Step, so wurde der erste Schritt
        fälschlicherweise als $\$plus\ 2\ 3\ foo$ ausgegeben.
    }{
        Javascript RegExp und die Java Regular Expression Bibliotheken unterscheiden sich in der Behandlung von dem
        Regex-Shorthand \texttt{\textbackslash v}.
    }{
        Der Regex-Shorthand wurde explizit ausgeschrieben so wie es vorgesehen war.
    }
    
    \regrtext{Hinzufügen einer benutzerdefinierten Bibliotheksfunktion soll in allen Zuständen der View außer RunningState möglich sein}{172}{
    	Gibt man einen gültigen $\lambda$-Term in das Feld ein und klickt auf den Run Button. 
		Dann ist das Hinzufügen einer benutzerdefinierte Bibliotheksfunktion nicht möglich.
    }{
    	Der SidebarView wird mitgeteilt, dass nur in den Zuständen DefaultState und 
		FinishedFinishedState das Hinzufügen von benutzerdefinierten Bibliotheksfunktionen möglich ist.
    }{
    	Wir erlauben das Betätigen des AddFunctionButton in allen Zuständen außer RunningState,
		indem wir auf den AddFunctionButton die Methode setEnabled(true) aufrufen.
    }

    \regrtest{Ungültige Funktionsnamen bewirken Exception}{179}{
        Stand: Dialog zum Hinzufügen von einer neuen Benutzerbibliotheksfunktion is offen.
        Gibt man einen ungültigen Namen ein und drückt auf Add, schließt sich der Dialog und die Funktion wird nicht
        hinzugefügt. Auch sieht man eine UmbrellaException in der JavaScript Konsole.
    }{
        Der Fehler ist klassisch. Es wurde vergessen, den Rückgabewert von \texttt{RegExp.exec(..)} auf \texttt{null} zu überprüfen.
    }{
        Der mögliche \texttt{null}-return wird abgefangen und jetzt richtig behandelt.
  }

    \section{Entfallene Kriterien und Testfallszenarien}

    \section{Glossar}
\end{document}
