% !TEX encoding = UTF-8 Unicode
\documentclass[parskip=full,11pt,twoside]{scrartcl}
\usepackage[utf8]{inputenc}

% section numbers in margins:
\renewcommand\sectionlinesformat[4]{\makebox[0pt][r]{#3}#4}

% header & footer
\usepackage{scrlayer-scrpage}
\lofoot{\today}
\refoot{\today}
\pagestyle{scrheadings}
\usepackage[T1]{fontenc}
\usepackage[sfdefault]{roboto}
\usepackage[german]{babel}
\usepackage[yyyymmdd]{datetime} % must be after babel
\renewcommand{\dateseparator}{-} % ISO8601 date format
\usepackage{hyperref}
\usepackage{amsmath} % for $\text{}$
\usepackage[nameinlink]{cleveref}
\crefname{figure}{Abb}{Abb}
\usepackage[section]{placeins}
\usepackage{xcolor}
\usepackage{graphicx}
\usepackage{colortbl}
\hypersetup{
pdftitle={Testbericht},
bookmarks=true,
}
\usepackage{csquotes}
\usepackage{upgreek}

\usepackage{ifthen}

\newcommand\urlpart[2]{$\underbrace{\text{\texttt{#1}}}_{\text{#2}}$}
\newcommand{\issueref}[1]{
    \href{https://git.scc.kit.edu/ap/Aurora/issues/#1}{(#1)}
}


% Titel GitLab issue number, Symptome, Grund, Behebung.
\newcommand{\regrtest}[5]{
    \subsection{#1 \issueref{#2}}
    \begin{itemize}
        \item \textbf{Symptom}
            #3
            %Hat man einen Term wie z.B. $\$plus\ 2\ 3\ \#\ foo$ eingegeben und klickt auf Step, so wurde der erste Schritt
            %fälschlicherweise als $\$plus\ 2\ 3\ foo$ ausgegeben.
        \item \textbf{Grund}
            #4
            %Javascript RegExp und die Java Regular Expression Bibliotheken unterscheiden sich in der Behandlung von dem
            %Regex-Shorthand \texttt{\textbackslash v}.
        \item \textbf{Behebung}
            #5
            %Der Regex-Shorthand wurde explizit ausgeschrieben so wie es vorgesehen war.
    \end{itemize}
}

\newcommand{\testline}[2]{
    \texttt{#1} & \ifthenelse{\equal{#2}{JA}}{\cellcolor{green!20}}{\cellcolor{red!20}}#2 \\ \hline
}

\begin{titlepage}
    \subject{Testbericht}
    \title{\textbf{$\uplambda$}urora}
    \subtitle{The Lambda Calculus IDE}


    \author{Iuliia Patrusheva, Alexander von Heyden\\
    Younis Bensalah, Max Nowak\\
    Nikolai Polley, Randy Seng}
\end{titlepage}


\begin{document}
\maketitle
\tableofcontents
\newpage

\section{Einleitung}


\section{Testfallszenarien}
    Die Testfallszenarien kommen aus dem Pflichtenheft. (Einführungstxt)

\subsection{Sprachenauswahl}
    Getestet:
    Sprache ist auswählbar (F37)

    \label{shortcuts}
    \begin{center}
        \begin{tabular}{ p{9cm} p{4cm}}
            Testfall & Abdeckung \\ \hline
            \testline{T1.1}{JA}
            \testline{T1.2}{NEIN}
        \end{tabular}
    \end{center}

\subsection{Teilen-Funktion}
    Getestet:
    LaTeX-Export (F55)

    \label{shortcuts}
    \begin{center}
        \begin{tabular}{ p{9cm} p{4cm}}
            Testfall & Abdeckung \\ \hline
            \testline{T2.1}{JA}
            \testline{T2.2}{NEIN}
            \testline{T2.3}{NEIN}
        \end{tabular}
    \end{center}

\subsection{Darstellung in Chrome}
    Getestet:
    Korrekte Darstellung in Google Chrome (N1),
    Eingabefeld (F1),
    Zeilennummern (F2),
    Night Mode (F38),
    Mobile Darstellung (F39)

    \label{shortcuts}
    \begin{center}
        \begin{tabular}{ p{9cm} p{4cm}}
            Testfall & Abdeckung \\ \hline
            \testline{T3.1}{JA}
            \testline{T3.2}{NEIN}
            \testline{T3.3}{NEIN}
        \end{tabular}
    \end{center}

\subsection{Eingabe und korrektes Ergebnis}
    Getestet:
    Eingabefeld (F1),
    Zeilennummern (F2),
    Eingabefeld bearbeiten (F3),
    Autovervollständigung von Klammern (F4),
    Keine Änderungen während der Auswertung (F6),
    Schleifenerkennung (F42),
    Kein Timeout (F43),
    Run (F14),
    Normalreihenfolge (F27),
    $\beta$-Reduktion (F40),
    Auswertung beendet (F41),
    Ausgabefeld (F32),
    Reset (F16),
    Pause (F17),
    Continue (F18),
    Ausgabefeld nicht editierbar (F33),
    Kommentare (F8)

    \label{shortcuts}
    \begin{center}
        \begin{tabular}{ p{9cm} p{4cm}}
            Testfall & Abdeckung \\ \hline
            \testline{T4.1}{JA}
            \testline{T4.2}{NEIN}
            \testline{T4.3}{NEIN}
            \testline{T4.4}{JA}
            \testline{T4.5}{NEIN}
            \testline{T4.6}{NEIN}
            \testline{T4.7}{JA}
            \testline{T4.8}{NEIN}
            \testline{T4.9}{NEIN}
            \testline{T4.10}{JA}
            \testline{T4.11}{NEIN}
        \end{tabular}
    \end{center}

\subsection{Testen von Highlighting}
    Getestet:
    Highlighting von Lambdas und Punkten (F9),
    Highlighting von Klammern (F10),
    Highlighting von Funktionsnamen (F11),
    Highlighting des nächsten auszuwertenden Redex (F12),
    Highlighting des zuletzt ausgeführten Redex (F13)

    \label{shortcuts}
    \begin{center}
        \begin{tabular}{ p{9cm} p{4cm}}
            Testfall & Abdeckung \\ \hline
            \testline{T5.1}{JA}
            \testline{T5.2}{NEIN}
            \testline{T5.3}{NEIN}
        \end{tabular}
    \end{center}

\subsection{Testen der Standard- und Benutzerbibliotheken}
    Getestet:
    Eingabefeld (F1),
    Zeilennummern (F2),
    Eingabefeld bearbeiten (F3),
    Autovervollständigung von Klammern (F4),
    Run (F14),
    Funktionsnamen-Konvention (F25),
    Standardbibliothek (F20),
    Benutzerbibliothek (F21),
    Erweitern der Benutzerbibliothek (F22),
    Definition einer Funktion (F24),
    Normalreihenfolge (F27),
    $\beta$-Reduktion (F40),
    Auswertung beendet (F41),
    Ausgabefeld (F32),
    Entfernen aus der Benutzerbibliothek (F23),
    Äquivalenz von Funktionen und deren Definition (F26),
    $\beta$-Reduktionen angezeigen (F34),
    Anzahl von dargestellten Schritten auswählen (F5)

    \label{shortcuts}
    \begin{center}
        \begin{tabular}{ p{9cm} p{4cm}}
            Testfall & Abdeckung \\ \hline
            \testline{T6.1}{JA}
            \testline{T6.2}{NEIN}
            \testline{T6.3}{NEIN}
            \testline{T6.4}{JA}
            \testline{T6.5}{NEIN}
            \testline{T6.6}{NEIN}
            \testline{T6.7}{JA}
            \testline{T6.8}{NEIN}
            \testline{T6.9}{NEIN}
            \testline{T6.10}{JA}
        \end{tabular}
    \end{center}

\subsection{Auswertungsstrategie und de Brujin}
    Getestet:
    Normalreihenfolge (F27),
    Call-by-Name (F28),
    Call-by-Value (F29),
    Auswahl der Auswertungsstrategie (F30),
    Benutzerdefinierte Redex Wahl (F31),
    Share (F19),
    De Bruijn-Indizes (F35)

    \label{shortcuts}
    \begin{center}
        \begin{tabular}{ p{9cm} p{4cm}}
            Testfall & Abdeckung \\ \hline
            \testline{T7.1}{JA}
            \testline{T7.2}{NEIN}
            \testline{T7.3}{NEIN}
        \end{tabular}
    \end{center}

\subsection{Sharing und Speichern}
    Getestet:
    In neuem Tab öffnen (F54),
    Historie (F44),
    Permanenz der Historie (F45),
    Wiederherstellen der Historie (F46),
    Benutzerbibliothek in der Historie (F48),
    Löschen der Historie (F47),
    Share (F19),
    E-Mail Link (F56)

    \label{shortcuts}
    \begin{center}
        \begin{tabular}{ p{9cm} p{4cm}}
            Testfall & Abdeckung \\ \hline
            \testline{T8.1}{JA}
            \testline{T8.2}{NEIN}
            \testline{T8.3}{NEIN}
            \testline{T8.4}{JA}
            \testline{T8.5}{NEIN}
            \testline{T8.6}{NEIN}
        \end{tabular}
    \end{center}

\subsection{Tutorial und Prettyprint}
    Getestet:
    Tutorial (F57),
    Pretty Print (F7)

    \label{shortcuts}
    \begin{center}
        \begin{tabular}{ p{9cm} p{4cm}}
            Testfall & Abdeckung \\ \hline
            \testline{T9.1}{JA}
            \testline{T9.1}{NEIN}
        \end{tabular}
    \end{center}

\subsection{Ausgabe von Zwischenschritten}
    Getestet:
    Anzahl der angezeigten Schritte (F36),
    Step (F15)

    \label{shortcuts}
    \begin{center}
        \begin{tabular}{ p{9cm} p{4cm}}
            Testfall & Abdeckung \\ \hline
            \testline{T10.1}{JA}
            \testline{T10.2}{NEIN}
            \testline{T10.3}{NEIN}
        \end{tabular}
    \end{center}

\subsection{Shortcuts}
    Getestet:
    Lambda-Shortcut (F49),
    Run-Shortcut (F50),
    Reset-Shortcut (F51),
    Pause-Shortcut (F52),
    Step-Shortcut (F53)

    \label{shortcuts}
    \begin{center}
        \begin{tabular}{ p{9cm} p{4cm}}
            Testfall & Abdeckung \\ \hline
            \testline{T11.1}{JA}
            \testline{T11.2}{NEIN}
            \testline{T11.3}{NEIN}
            \testline{T11.4}{NEIN}
            \testline{T11.5}{NEIN}
        \end{tabular}
    \end{center}

\section{Testfallüberdeckung per Package}
    % Viele Screenshots von den überdeckungsberichtdingers.
    % Vielleich auch Überdeckung mit nur JUnit, und dann noch mit Selenium mit dazu?

\section{Regression Tests}
    \textbf{Was das?}
    Gesquäschte Bugs: 1337.


    \regrtest{Schrittzahl im Schrittfenster fängt mit falschen Index an}{120}{
        Wir stellen die Schrittzahl auf 2 ein.
        Es wird ($\lambda$x. x x)($\lambda$x. x x) in den Code-Editor eingegeben,
            dann auf den Run Button klickt, dann ein paar Sekunden gewartet und auf den Pause-Button geklickt.
        Es werden 2 Schritte im Schrittfenster angezeigt.
        Beispielsweise tragen diese die Nummern 101 und 102.
        Klickt man nun auf den Step Button, dann werden die 2 Schritte, die ausgegeben werden, die Nummern 104 und 105.
        Dabei wird die Nummer 103 ausgelassen.
    }{
        Der EditorPresenter übergibt der EditorView via dem EditorDisplay den falschen Schrittindex.
        Der EditorPresenter hat bei der Berechnung des Schrittindex einen falschen Offset angegeben, welcher zum Fehler führt.
    }{
        Es wird beim EditorPresenter der richtige Offset zur Berechnung des Schrittindex angegeben.
    }

    \regrtest{Leerer Input in StepNumber TextBox führt zu nicht valide, dargestellten Wert}{133}{
        Lösche den gesamten Input in StepNumber TextBox, so dass diese leer ist. Klicke auf eine beliebige Stelle in der Aurora WebApp.
        Klicke zum Beispiel auf den Input Editor, so dass die StepNumber TextBox den Fokus verliert.
        Die StepNumber TextBox bleibt leer, obwohl sie eigentlich eine valide Schrittzahl anzeigen sollte.
    }{
        Die StepNumber TextBox überprüft den Input nach jeder neuen Eingabe.
        Falls diese nicht valide ist, wird der letzte valide Wert angenommen.
        Zudem wird auch eine leere TextBox akzeptiert, da auch valide Werte der Länge 1 eingegeben werden sollen.
        Wenn die TextBox leer ist und den Fokus verliert, dann wird nicht der letzte valide Wert angenommen, da kein BlurHandler der StepNumber Textbox registriert wurde,
            der den Wert wieder auf den letzten validen Wert setzt.
    }{
        Der StepNumber TextBox wird ein BlurHandler hinzugefügt, der den letzten validen Wert der StepNumber TextBox zuweist.
    }

    \regrtest{Das Ausführen einer nicht definierten Bibliotheksfunktion sollte einen Fehler anzeigen}{143}{
        Man gibt einen regulären $\lambda$-Term ein, der eine nicht definierte Bibliotheksfunktion enthält, ein und drückt auf den Run Button.
        Es wird keine Fehlermeldung angezeigt.
    }{
        Das Anzeigen von Fehlern, bei Eingabe eines ungültigen $\lambda$-Terms, wurde noch nicht implementiert.
    }{
        Implementiere die Funktionen displaySemanticError und displaySyntaxError in der Klasse EditorView.
        Bei Ausführen der Berechnung eines ungültigen $\lambda$-Terms sollte in ein Zustandsübergang von RunningState auf DefaultState erfolgen.
        Deswegen wird jedem Zustand eine neue Kante errorDisplayed hinzugefügt und implementiert.
    }

    \regrtest{Kommentare mitgeparst}{151}{
        Hat man einen Term wie z.B. $\$plus\ 2\ 3\ \#\ foo$ eingegeben und klickt auf Step, so wurde der erste Schritt fälschlicherweise als $\$plus\ 2\ 3\ foo$ ausgegeben.
    }{
        Javascript RegExp und die Java Regular Expression Bibliotheken unterscheiden sich in der Behandlung von dem Regex-Shorthand \texttt{\textbackslash v}.
    }{
        Der Regex-Shorthand wurde explizit ausgeschrieben so wie es vorgesehen war.
    }

    \regrtest{Hinzufügen einer benutzerdefinierten Bibliotheksfunktion soll in allen Zuständen der View außer RunningState möglich sein}{172}{
        Gibt man einen gültigen $\lambda$-Term in das Feld ein und klickt auf den Run Button.
        Dann ist das Hinzufügen einer benutzerdefinierte Bibliotheksfunktion nicht möglich.
    }{
        Der SidebarView wird mitgeteilt, dass nur in den Zuständen DefaultState und
            FinishedFinishedState das Hinzufügen von benutzerdefinierten Bibliotheksfunktionen möglich ist.
    }{
        Wir erlauben das Betätigen des AddFunctionButton in allen Zuständen außer RunningState,
            indem wir auf den AddFunctionButton die Methode setEnabled(true) aufrufen.
    }

    \regrtest{Ungültige Funktionsnamen bewirken Exception}{179}{
        Stand: Dialog zum Hinzufügen von einer neuen Benutzerbibliotheksfunktion is offen.
        Gibt man einen ungültigen Namen ein und drückt auf Add, schließt sich der Dialog und die Funktion wird nicht hinzugefügt.
        Auch sieht man eine UmbrellaException in der JavaScript Konsole.
    }{
        Der Fehler ist klassisch. Es wurde vergessen, den Rückgabewert von \texttt{RegExp.exec(..)} auf \texttt{null} zu überprüfen.
    }{
        Der mögliche \texttt{null}-return wird abgefangen und jetzt richtig behandelt.
    }

    \regrtest{Reduktion mit einer Libraryfunktion schlägt fehl}{172}{
        Als die Funktion $\$$infinity mit dem Term ($\lambda$x. x x)($\lambda$x. x x) der
        Benutzerbibliothek hinzugefügt wurde und sie in
        das Eingabefeld geschrieben wurde, wurde der Term nicht reduziert.
    }{
        Die Strategie baut einen Redexpath, der auf eine Applikation zeigt.
        Als die Applikation aber in einer Funktion war, wurde sie nicht gefunden und es wurde ein Error geworfen.
        Bei der Entwicklung des Betareduzierers wurde vernachlässigt, dass es auch Redexe in Funktionen geben kann.
        Die meisten Funktionen sind zwar nicht mit sich selbst reduzierbar, da man sonst die Funktion zuerst reduzieren würde und dann hinzufügen würde.
        Dennoch muss es möglich sein auch Funktionen nicht in Normalform hinzuzufügen.
    }{
        Wenn der Redexpath entlanggegangen wird und er auf eine Funktion zeigt, wird diese in ihren Term umgewandelt und dann wird dieser Term abgestiegen.
    }

    \regrtest{Alphakonversion ist nicht korrekt}{134}{
        Als 2 2 in das Inputfeld gegeben wurde und der Simplifier noch nicht implementiert war, wurde im Resultfeld das Ergebnis:
        $\lambda$z.$\lambda$z.z1(z1(z1(z1 z)) angezeigt.
        Intern war die Berechnung zwar korrekt und die Churchzahl 4 war berechnet worden,
            aber die Alphakonversion war inkorrekt, da festgelegt wurde, dass jede Variable, der eine Nummer angehängt wurde,
            eine freie Variable und keine gebundene Variable ist.
    }{
        Zu diesem Zeitpunk lief Alphakonversion folgendermaßen ab.
        Jede Abstraktion suchte nach freien Variablen, die den gleichen Namen wie die Abstraktion hatten.
        Falls dies der Fall war, wurde der freien Variable eine 1 angefügt.
        Bei der Suche nach den freien Variablen wurde gleichzeitig jede gebundene Variable, die auf die Abstraktion zeigt,
            mit einer freien Variable mit dem Namen der Abstraktion ersetzt.
        Das Ergebnis ohne Alphakonversion wäre in Debruijnindizes:
            $\lambda$z. $\lambda$z. 2 (2(2(2 1))).

        Nach diesem ersten Schritt wäre es $\lambda$z. $\lambda$z. z(z(z(z 1))) wobei die $z$ freie Variablen wären und nicht mehr gebundene Variablen.
        Vergessen wurde, dass es möglich ist, dass eine Abstraktion eine andere Abstraktion mit selben Namen im Body hat.
        Jetzt wurden bei der zweiten Abstraktion erkannt, dass die freien Variablen den gleichen Namen wie die zweite Abstraktion haben,
            und dann wurden sie alle alphakonvertiert zu z1 was dem Nutzer gesagt hätte, dass dies freie Variablen wären.
        Durch die gebundene Variable $z$ am Schluss ist ein weiteres Problem aufgefallen:
        Wie lassen sich diese  zwei unterschiedliche Ausdrücke noch in Debruijndarstellung mit Alphakonversion kenntlich machen.
            $\lambda x.\lambda x. 1$
            $\lambda x. \lambda x. 2$
        Dies wurde zu diesem Zeitpunkt noch nicht berücksichtigt.
    }{
        Da dieses erste Problem dadurch herrührte, das die gebundenen Variablen gleichzeitig mit der Suche nach den freien Variablen passierte,
            wurde dies in einen extra Schritt verfrachtet.
        Es wurden zuerst freie Variablen auf Alphakonversion überprüft und erst dann die gebundenen Variablen ersetzt.
        Um das zweite Problem zu beheben, wurde ein Visitor vor die Suche nach den freien Variablen gesetzt, der Namen von Abstraktionen ändert,
            sollten sie gleich heißen wie eine Abstraktion in deren Body sie sind.
        Dies führte zu einer funktionellen Alphakonversion, allerdings wurde uns mitgeteilt, dass man nicht den Namen von freien Variablen ändern sollte.
        Sollte eine freie Variable mit gleichen Namen gefunden werden, muss der Name der Abstraktion geändert werden, nicht der Name der freien Variable.

        Dies führt final zu folgender Struktur:
           Es gibt einen Visitor, der den Term traversiert und bei jeder Abstraktion folgende Aktionen ausführt:
           Er initialisiert einen weiteren Visitor, der den Term traversiert und nach Abstraktionen sucht, die den gleichen Namen haben wie die Abstraktion,
            von der der Visitor gestartet ist.
        Falls eine gefunden wurde, wird der Name dieser zweiten Abstraktion eine Zahl angehängt.

        Als Beispiel: $\lambda z. \lambda z. \lambda z. ...$ wird zu $\lambda z. \lambda z1. \lambda z1. ...$
            und wenn der ursprüngliche Visitor zu der zweiten Abstraktion kommt, wird daraus $\lambda z. \lambda z1. \lambda z2. ...$.
           Dann wird nach freien Variablen gesucht, die den gleichen Namen haben wie die ursprüngliche Abstraktion.
        Falls dies der Fall ist, wird der ursprünglichen Abstraktion ein $\_ alpha$ hinzugefügt.
        In dem Beispiel $\lambda z. z$, bei dem das zweite $z$ eine freie Variable ist, wird dies zu $\lambda z \_ alpha. \  z$ verändert.
           Hier wird keine Zahl an die Abstraktion gehängt, um dem Benutzer, klar zu machen, welche Form von Alphakonversion stattgefunden hat.
           Wenn der Name der Abstraktion geändert wird, muss  nach neuen freien Variablen gesucht werden, die den gleichen Namen wie die neue Abstraktion haben.
        Es könnte vorkommen, dass ein Tester diesen folgenden Term auf Alphakonversion testet:
            $\lambda z. z \ z\_ alpha$ wobei $z$ und $z\_ alpha$ freie Variablen sind.
           Hier wäre nach nur einem Schritt $z\_ alpha$ als gebundene Variable zu erkennen und nicht als freie Variable.
           Deswegen muss erneut nach freien Variablen gesucht werden die den gleichen Namen besitzen.
           In diesem Sonderfall würde erneut ein $\_ alpha$ an den Abstraktionsnamen angehängt werden.

           Wenn dies vollbracht ist, werden gebundene Variablen durch freie Variablen mit dem Namen der Abstraktion ersetzt.
    }

\section{Entfallene/Adaptierte Kriterien und Testfallszenarien}
\subsection{Ausgabe mit De Brujin-Indizes (K22)}
    Es wird immer noch intern mit De Brujin-Indizes gerechnet, jedoch hat das Hinzufügen und Testen einer zusätzlichen Option,
        die vom Benutzer verändert werden kann, zeitlich nicht gereicht.
    Die ausgabe erfolgt jetzt mit Variablennamen und es wird $\alpha$-Konversion durchgeführt.

%\subsection{Benutzerdefinierte Reduktionsstrategie (K10)}
    % TODO kann gut sein, dass das noch implementiert wird.
    % Das selbe gilt für:
    % K5: Redexe highlighten.
    % K16: Öffnen im neuen Tab.
    % K20: Aurora Tutorial.

\subsection{Mobile Darstellung (K25)}
    Eine mobile Darstellung würde eine sehr grundlegende Änderung vom graphischen Benutzerinterfacekonzept erfordern.
    Es wurde sehr früh entschieden, dieses Wunschkriterium auszuschließen.
    % TODO oder machen wirs doch noch :P ?

\section{Glossar}

\end{document}
