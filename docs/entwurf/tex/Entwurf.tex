% !TEX encoding = UTF-8 Unicode
\documentclass[parskip=full,11pt,twoside]{scrartcl}
\usepackage[utf8]{inputenc}

% section numbers in margins:
\renewcommand\sectionlinesformat[4]{\makebox[0pt][r]{#3}#4}

% header & footer
\usepackage{scrlayer-scrpage}
\lofoot{\today}
\refoot{\today}
\pagestyle{scrheadings}

\usepackage[sfdefault,light]{roboto}
\usepackage[T1]{fontenc}
\usepackage[german]{babel}
\usepackage[yyyymmdd]{datetime} % must be after babel
\renewcommand{\dateseparator}{-} % ISO8601 date format
\usepackage{hyperref}
\usepackage[nameinlink]{cleveref}
\crefname{figure}{Abb}{Abb}
\usepackage[section]{placeins}
\usepackage{xcolor}
\usepackage{graphicx}
\hypersetup{
	pdftitle={Entwurfsdokumentation},
	bookmarks=true,
}
\usepackage{csquotes}

\usepackage{amsmath} % for $\text{}$
\newcommand\urlpart[2]{$\underbrace{\text{\texttt{#1}}}_{\text{#2}}$}


\begin{titlepage}

\subject{Entwurfsdokumentation}
\title{$\lambda$urora}
\subtitle{The Lambda Calculus IDE}


\author{Iuliia Patrusheva, Alexander von Heyden\\
Younis Bensalah, Max Nowak\\
Nikolai Polley, Randy Seng}

\end{titlepage}




\begin{document}
\maketitle
\pagebreak
\tableofcontents
\pagebreak
\section{Einleitung}
\subsection{Zweck der Software}
Das Ziel ist es, Studenten eine Lernumgebung anzubieten. Man soll ein Gefühl für das $\lambda$-Kalkül entwickeln, durch Ausprobieren
und durch farbige hervorhebungen um die Struktur der Terme anschaulicher zu machen.
Ein Nebenziel ist es, das Erstellen von Lernmaterial zu erleichtern (insb. \LaTeX-Export), sowie Code schnell und zugänglich anderen zu Verfügung zu stellen.
\subsection{Entwurfsziele}
\pagebreak

\section{Grobentwurf}
\subsection{Architekturmuster und/oder Systemzerlegung}
MVC oder MVP oder darf noch was sein?
Vorteile gegenüber anderer Modelle
\subsubsection{Übersicht}
Hier kommt ein grober Übersicht von MVC oder MVP im Kontext des Auroras
\subsubsection{Model}
Was beinhaltet dieses Teil (Datenmodelle) + wen kennt es?
\subsubsection{View}
Was beinhaltet dieses Teil (Darstellung) + wen kennt es?
\subsubsection{Presenter}
Wir haben MVP gewählt, weil GWT mit für MVP entworfen wurde. Es wäre auch möglich, MVC zu wählen, allerdings würden sich
einige Schwierigkeiten daraus ergeben: % TODO warum nicht MVC?
\subsubsection{Lambda-Bibliothek}
Man kann einen Teil des Projektes als eine wiedeverwendbare Bibliotek betrachten. Im Code ist dieses nicht
explicit kenntlich gemacht worden, außer einem einenem Package.
Diese Bibliothek existiert, um den Umgang mit Lambda-Termen zu erleichtern, und implementiert die
Funktionalität zum Beta-Reduzieren, Substitution von Variablen, finden von Redexen, sowie verschiede Auswertungsstrategien von Lambda-Termen.


\subsection{Sicherheit und Zugrifskontrole}
\subsection{Kontrollfluss}
Wenn MVC dann erfolgt durch Controller...  + SEQUENZDIAGRAMMEN (sollen sie getrennt sein oder passt es hier?)
\subsection{Datenbank?}
Es wird nichts Serverseitig gespeichert. Die Sharing-Funktionen codieren das Nötige in der URL.
Diese Designentscheiding wurde der Einfachheit halber getroffen.
\subsection{Randbedingugen}
Entwurfsanforderungen, die an das System gestellt werden. (anhand sequenzdiagrammen)
\pagebreak

\section{Feinentwurf}
Hierfür haben wir doch schon UML?
\subsection{Richtlinien der Schnittstellendokumentation}
wie die Doku gemacht ist (Latex, Latex Javadoc)
Es wird TeXdoclet verwendet. Dieser hat ein paar Macken :P.
\subsection{Pakete}
\subsection{Entwurfsmuster (welche sind benutzt)}
\subsection{Klassen und Schnittstellen}
\pagebreak

\end{document}
