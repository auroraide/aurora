% !TEX encoding = UTF-8 Unicode
%\documentclass[parskip=full,11pt,twoside]{scrartcl}
\documentclass[parskip=full,11pt,twoside]{scrbook}
\usepackage[utf8]{inputenc}

\usepackage[T1]{fontenc}
\usepackage{lmodern}

\usepackage{color}
\usepackage{hyperref}

\pagestyle{myheadings}

\def\packagename{}
\def\classname{}

% section numbers in margins:
\renewcommand\sectionlinesformat[4]{\makebox[0pt][r]{#3}#4}

% header & footer
\usepackage{scrlayer-scrpage}
\lofoot{\today}
\refoot{\today}
\pagestyle{scrheadings}

\markboth{\protect\packagename{} -- \protect\classname{}}{\protect\packagename{} -- \protect\classname{}}

% hot mess.. I mean.. hotfix
\makeatletter
\DeclareOldFontCommand{\rm}{\normalfont\rmfamily}{\mathrm}
\DeclareOldFontCommand{\sf}{\normalfont\sffamily}{\mathsf}
\DeclareOldFontCommand{\tt}{\normalfont\ttfamily}{\mathtt}
\DeclareOldFontCommand{\bf}{\normalfont\bfseries}{\mathbf}
\DeclareOldFontCommand{\it}{\normalfont\itshape}{\mathit}
\DeclareOldFontCommand{\sl}{\normalfont\slshape}{\@nomath\sl}
\DeclareOldFontCommand{\sc}{\normalfont\scshape}{\@nomath\sc}
\makeatother

\usepackage[sfdefault,light]{roboto}
\usepackage[T1]{fontenc}
\usepackage[german]{babel}
\usepackage[yyyymmdd]{datetime} % must be after babel
\renewcommand{\dateseparator}{-} % ISO8601 date format
\usepackage{hyperref}
\usepackage[nameinlink]{cleveref}
\crefname{figure}{Abb}{Abb}
\usepackage[section]{placeins}
\usepackage{xcolor}
\usepackage{graphicx}
\hypersetup{
	pdftitle={Entwurfsdokumentation},
	bookmarks=true,
}
\usepackage{csquotes}

\usepackage{amsmath} % for $\text{}$
\newcommand\urlpart[2]{$\underbrace{\text{\texttt{#1}}}_{\text{#2}}$}

% ---------------------------------------------------------------------------
% TexDoc macros start - everything below this point should be copied to your
% own document and adapted to your style/language if needed
% ---------------------------------------------------------------------------

% Environment used to simulate html <p> </p>
\newenvironment{texdocp}{}{

}
% Environment for packages
\newenvironment{texdocpackage}[1]{%
	\newpage{}\gdef\packagename{#1}\chapter{Package \texttt{#1}}
	\rule{\hsize}{.7mm}
}{}

% Environment for classes, interfaces
% Argument 1: "class" or "interface"
% Argument 2: the name of the class/interface
\newenvironment{texdocclass}[2]{%
	\gdef\classname{#2}
	\section{\texttt{#1 \textbf{#2}}}
}{\newpage{}}

% Environment for class description
\newenvironment{texdocclassintro}{
	\subsection*{Description}
}{
}

% Environment around class fields
\newenvironment{texdocclassfields}{%
	\subsection*{Attributes}
	\begin{itemize}
}{%
	\end{itemize}
}

% Environment around class methods
\newenvironment{texdocclassmethods}{%
	\subsection*{Methods}
	\begin{itemize}
}{%
	\end{itemize}
}

% Environment around class Constructors
\newenvironment{texdocclassconstructors}{%
	\subsection*{Constructors}
	\begin{itemize}
}{%
	\end{itemize}
}

% Environment around enum constants
\newenvironment{texdocenums}{%
	\subsection*{Enum Constants}
	\begin{itemize}
}{%
	\end{itemize}
}

% Environment around "See also"-Blocks (\texdocsee invocations)
%  Argument 1: Text preceding the references
\newenvironment{texdocsees}[1]{

	\textbf{#1:}
	\begin{itemize}
}{%
	\end{itemize}
}
% Formats a single field
%  Argument 1: modifiers
%  Argument 2: type
%  Argument 3: name
%  Argument 4: Documentation text
\newcommand{\texdocfield}[4]{\item \texttt{#1 #2 \textbf{#3}} \\ #4}

% Formats an enum element
%  Argument 1: name
%  Argument 2: documentation text
\newcommand{\texdocenum}[2]{\item \texttt{\textbf{#1}} \\ #2}

% Formats a single method
%  Argument 1: modifiers
%  Argument 2: return type
%  Argument 3: name
%  Argument 4: part after name (parameters)
%  Argument 5: Documentation text
%  Argument 6: Documentation of parameters/exceptions/return values
\newcommand{\texdocmethod}[6]{\item \texttt{#1 #2 \textbf{#3}#4} \\ #5#6}

% Formats a single constructor
%  Argument 1: modifiers
%  Argument 2: name
%  Argument 3: part after name (parameters)
%  Argument 4: Documentation text
%  Argument 5: Documentation of parameters/exceptions/return values
\newcommand{\texdocconstructor}[5]{\item \texttt{#1 \textbf{#2}#3} \\ #4#5}

% Inserted when @inheritdoc is used
%  Argument 1: Class where the documentation was inherited from
%  Argument 2: Documentation
\newcommand{\texdocinheritdoc}[2]{#2 (\textit{documentation inherited from \texttt{#1})}}

% Formats a single see-BlockTag
%  Argument 1: text
%  Argument 2: reference label
\newcommand{\texdocsee}[2]{\item \texttt{#1 (\ref{#2})}}

% Environment around \texdocparameter invocations
\newenvironment{texdocparameters}{%
	\minisec{Parameters}
	\begin{tabular}{ll}
}{%
	\end{tabular}
}

% Environment around \texdocthrow invocations
\newenvironment{texdocthrows}{%
        \minisec{Throws}
        \begin{tabular}{ll}
}{%
        \end{tabular}
}

\newcommand{\texdocreturn}[1]{\minisec{Returns} #1}

% Formats a parameter (this gets put inside the input of a \texdocmethod or
% \texdocconstructor macro)
%  Argument 1: name
%  Argument 2: description text
\newcommand{\texdocparameter}[2]{\texttt{\textbf{#1}} & \begin{minipage}[t]{0.8\textwidth}#2\end{minipage} \\}

% Formats a throws tag
%  Argument 1: exception name
%  Argument 2: description text
\newcommand{\texdocthrow}[2]{\texttt{\textbf{#1}} & \begin{minipage}[t]{0.6\textwidth}#2\end{minipage} \\}

% Used to simulate html <br/>
\newcommand{\texdocbr}{\mbox{}\newline{}}

% Used to simulate html <h[1-9]> - </h[1-9]>
% Argument 1: number of heading (5 for a <h5>)
% Argument 2: heading text
\newcommand{\headref}[2]{\minisec{#2}}

\newcommand{\refdefined}[1]{
\expandafter\ifx\csname r@#1\endcsname\relax
\relax\else
{$($ in \ref{#1}, page \pageref{#1}$)$}
\fi}

% ---------------------------------------------------------------------------
% TexDoc macros end
% ---------------------------------------------------------------------------

\begin{titlepage}

\subject{Entwurfsdokumentation}
\title{$\lambda$urora}
\subtitle{The Lambda Calculus IDE}


\author{Iuliia Patrusheva, Alexander von Heyden\\
Younis Bensalah, Max Nowak\\
Nikolai Polley, Randy Seng}

\end{titlepage}




\begin{document}
\maketitle
\pagebreak
\tableofcontents
\pagebreak
\section{Einleitung}
\subsection{Zweck der Software}
Das Ziel ist es, Studenten eine Lernumgebung anzubieten in der man eigene Lambda Terme reduziert. Man soll ein Gefühl für das $\lambda$-Kalkül entwickeln, durch Ausprobieren
und durch farbige Hervorhebungen um die Struktur der Terme und der Berechnungen anschaulicher zu machen.
Dies wird außerdem durch vordefinierte häufig benutzte Terme aus der Standardbibliothek und Church Zahlen in numerischer Schreibweise ermöglicht.
Ein Nebenziel ist es, das Erstellen von Lernmaterial zu erleichtern (insb. \LaTeX-Export), sowie Code schnell und zugänglich anderen zu Verfügung zu stellen.
\subsection{Entwurfsziele}
\pagebreak

\section{Grobentwurf}
\subsection{Architekturmuster und/oder Systemzerlegung}
\subsubsection{Übersicht}
Das Programm is clientseitig und läuft auf einem modernen Browser. Da in PSE eine objektorientierte Sprache benutzt werden muss, haben wir uns auf die Programmierung in Java und der Verwendung des Webframeworks Google Web Toolkit geeinigt. GWT lässt den Entwickler eine Javascrip Webanwendung in Java zu schreiben.
$\lambda$urora ist ein klassisches Programm mit einer GUI für den Input und einem Backend für Berechnungen. In der Vorlesung Softwaretechnik I wurde für dieses Szenario Model-View-Controller vorgestellt.
Wir haben allerdings  Model-View-Presenter gewählt, weil GWT für MVP entworfen wurde. Es wäre zwar auch möglich gewesen MVC zu wählen, allerdings würden sich vor allem bei Testcases
einige Schwierigkeiten daraus ergeben. Mit MVP kann man sehr viel mit klassischen Java Tests abdecken, bei MVC wäre dies nicht so einfach möglich und man müsste mehr GWT Tests schreiben die eine sehr langsame Ausführungszeit haben.
\subsubsection{Model}
\subsubsection{View}
Was beinhaltet dieses Teil (Darstellung) + wen kennt es?
\subsubsection{Presenter}

\subsubsection{Lambda-Bibliothek}
Man kann einen Teil des Projektes als eine wiedeverwendbare Bibliotek betrachten. Im Code ist dieses nicht
explizit kenntlich gemacht worden, außer einem eigenem Package.
Diese Bibliothek existiert, um den Umgang mit Lambda-Termen zu erleichtern, und implementiert die
Funktionalität zum Beta-Reduzieren, Substitution von Variablen, finden von Redexen, sowie verschiedene Auswertungsstrategien von Lambda-Termen.


\subsection{Sicherheit und Zugrifskontrole}
\subsection{Kontrollfluss}
Wenn MVC dann erfolgt durch Controller...  + SEQUENZDIAGRAMMEN (sollen sie getrennt sein oder passt es hier?)
\subsection{Randbedingugen}
Entwurfsanforderungen, die an das System gestellt werden. (anhand sequenzdiagrammen)
\pagebreak

\section{Feinentwurf}

% include crap from texdoclet
%\tableofcontents
Do \textit{not} change or commit this file!


\subsection{Richtlinien der Schnittstellendokumentation}

wie die Doku gemacht ist (Latex, Latex Javadoc)
Es wird TeXdoclet verwendet. Dieser hat ein paar Macken :P. (Ein "paar"... Ja, genau... fml... *sigh* -younis)
\subsection{Pakete}
\begin{description}
	
	
	\item[aurora] ~\par
	Alle Pakete und Klassen sind in diesem Oberpaket definiert.
	
	\item[aurora.backend] ~\par
	Im Backend sind alle Berechnungen und Datenstrukturen die für das Lambda Kalkül gebraucht werden. 
	
	\item[aurora.backend.parser] ~\par
	In diesem Paket wird eine Eingabe geparsed.

	\item[aurora.backend.tree] ~\par
	Dies ist die grundlegende Datenstruktur in der die Lambdaterme gespeichert werden.
	
	\item[aurora.backend.library] ~\par
	Dieses Paket wird für die Benutzer- und Standardbibliothek benutzt.
	
	\item[aurora.backend.betareduction] ~\par
	Dieses Paket umfasst alles was für die Beta-Reduktion benötigt wird.
	
	\item[aurora.backend.betareduction.visitors] ~\par
	 In diesem Paket sind die Besucher, welche den Tree traversieren. Sie werden für die Beta-Reduktion benötigt.  
	\item[aurora.backend.betareduction.strategies] ~\par
	Hier liegen alle Reduktionsstrategien die für die Beta-Reduktion benutzt werden können.

	\item[aurora.backend.encoders] ~\par
	Das Paket wird für das encoden und decoden von Input benutzt.
	
	\item[aurora.backend.simplifier] ~\par
	In diesem Paket wird versucht ein komplexer Term in eine simple Ausgabe zu überführen.
	
	\item[aurora.client] ~\par
	Hier sind alle Funktionen die für den Client benötigt werden.
	
	\item[aurora.client.event] ~\par
	Das Paket beinhaltet alle Events die ausgelöst werden können.
	
	\item[aurora.client.view] ~\par
	Hier liegt die GUI.
	
	\item[aurora.client.view.editor] ~\par
	Der Editor für die Benutzereingabe.
	
	\item[aurora.client.view.editor.components] ~\par
	Die Bestandteile der GUI.
	
	\item[aurora.client.view.popup] ~\par
	Das Paket beinhaltet alle Popups.
	
	\item[aurora.client.view.sidebar] ~\par
	An der linken Seite der GUI ist eine Sidebar. Sie wird aus diesem Paket hergestellt
	
	\item[aurora.client.view.sidebar.components] ~\par 
	Die Bestandteile der Sidebar.
	
	\item[aurora.client.presenter] ~\par
	Der Presenter der den Daten- und Kontrollfluss steuert.
\end{description}
\subsection{Entwurfsmuster (welche sind benutzt)}
Für das Traversieren der Terme wird das Visitor Pattern (Besucher) benutzt. Es werden jeweils konkrete Besucher erstellt, die jeweils unterschiedliche Funktionen besitzen um den Term auszuwerten.
\subsection{Klassen und Schnittstellen}
\pagebreak

\section{Änderungen zum Pflichtenheft}
Das Wunschkriterium \textbf{K6. Pretty Print} ist in diesem Entwurf nicht mehr vorhanden. Das Team hat sich dafür entschieden, dass dieses Kriterium zuviel Zeit bei der Implementierung beanstanden würde und das Feature nicht wirklich gewünscht wurde.

Gleiches gilt für das Wunschkriterium \textbf{K19. Schleifenerkennung}. Dieses Feature hätte nur sehr selten Anwendung gefunden und wäre großteils unbenutzt geblieben.

Das Wunschkriterium \textbf{K23. Lokale Sessions} ist auch gestrichen worden. Zu viel Aufwand wäre in die Entwicklung dieses Features geflossen und wir haben beschlossen, dass dieses Feature den Aufwand nicht wert ist.

Das Wunschkriterium \textbf{K25. Mobile-Darstellung} ist in dem jetzigen Entwurf nicht vorhanden. Die Gruppe lässt sich aber offen, wenn noch Zeit bei der Implementierung übrig bleibt, dieses Feature über das rückläufiges Wassermodell wieder einzufügen.

In Entwicklungsumgebung ist als \textbf{Build System} Bazel weggefallen, für GWT eignet sich Ant besser.
Auch die \textbf{IDE} Eclipse wird von keinem Mitglied des Teams verwendet.
\end{document}
