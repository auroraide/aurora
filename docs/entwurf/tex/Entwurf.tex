% !TEX encoding = UTF-8 Unicode
%\documentclass[parskip=full,11pt,twoside]{scrartcl}
\documentclass[parskip=full,11pt,twoside]{scrbook}
\usepackage[utf8]{inputenc}

\usepackage[T1]{fontenc}
\usepackage{lmodern}

\usepackage{color}
\usepackage{hyperref}
\usepackage{pdfpages}
\pagestyle{myheadings}

\def\packagename{}
\def\classname{}

% section numbers in margins:
\renewcommand\sectionlinesformat[4]{\makebox[0pt][r]{#3}#4}

% header & footer
\usepackage{scrlayer-scrpage}
\lofoot{\today}
\refoot{\today}
\pagestyle{scrheadings}

\markboth{\protect\packagename{} -- \protect\classname{}}{\protect\packagename{} -- \protect\classname{}}

% hot mess.. I mean.. hotfix
\makeatletter
\DeclareOldFontCommand{\rm}{\normalfont\rmfamily}{\mathrm}
\DeclareOldFontCommand{\sf}{\normalfont\sffamily}{\mathsf}
\DeclareOldFontCommand{\tt}{\normalfont\ttfamily}{\mathtt}
\DeclareOldFontCommand{\bf}{\normalfont\bfseries}{\mathbf}
\DeclareOldFontCommand{\it}{\normalfont\itshape}{\mathit}
\DeclareOldFontCommand{\sl}{\normalfont\slshape}{\@nomath\sl}
\DeclareOldFontCommand{\sc}{\normalfont\scshape}{\@nomath\sc}
\makeatother

\usepackage[sfdefault,light]{roboto}
\usepackage[T1]{fontenc}
\usepackage[german]{babel}
\usepackage[yyyymmdd]{datetime} % must be after babel
\renewcommand{\dateseparator}{-} % ISO8601 date format
\usepackage{hyperref}
\usepackage[nameinlink]{cleveref}
\crefname{figure}{Abb}{Abb}
\usepackage[section]{placeins}
\usepackage{xcolor}
\usepackage{graphicx}
\hypersetup{
	pdftitle={Entwurfsdokumentation},
	bookmarks=true,
}
\usepackage{csquotes}

\usepackage{amsmath} % for $\text{}$
\newcommand\urlpart[2]{$\underbrace{\text{\texttt{#1}}}_{\text{#2}}$}

\renewcommand{\thesection}{\arabic{section}}
% ---------------------------------------------------------------------------
% TexDoc macros start - everything below this point should be copied to your
% own document and adapted to your style/language if needed
% ---------------------------------------------------------------------------

% Environment used to simulate html <p> </p>
\newenvironment{texdocp}{}{

}
% Environment for packages
\newenvironment{texdocpackage}[1]{%
	%\newpage{}
    \gdef\packagename{#1}\subsection{Package \texttt{#1}}
	\rule{\hsize}{.7mm}
}{}


% Environment for classes, interfaces
% Argument 1: "class" or "interface"
% Argument 2: the name of the class/interface
\newenvironment{texdocclass}[2]{%
	\gdef\classname{#2}
	\subsubsection{\texttt{#1 #2}}
}{\newpage{}}

% Environment for class description
\newenvironment{texdocclassintro}{
	\subsubsection*{Description}
}{
}

% Environment around class fields
\newenvironment{texdocclassfields}{%
	\subsubsection*{Attributes}
	\begin{itemize}
}{%
	\end{itemize}
}

% Environment around class methods
\newenvironment{texdocclassmethods}{%
	\subsubsection*{Methods}
	\begin{itemize}
}{%
	\end{itemize}
}

% Environment around class Constructors
\newenvironment{texdocclassconstructors}{%
	\subsubsection*{Constructors}
	\begin{itemize}
}{%
	\end{itemize}
}

% Environment around enum constants
\newenvironment{texdocenums}{%
	\subsubsection*{Enum Constants}
	\begin{itemize}
}{%
	\end{itemize}
}

% Environment around "See also"-Blocks (\texdocsee invocations)
%  Argument 1: Text preceding the references
\newenvironment{texdocsees}[1]{

	\textbf{#1:}
	\begin{itemize}
}{%
	\end{itemize}
}
% Formats a single field
%  Argument 1: modifiers
%  Argument 2: type
%  Argument 3: name
%  Argument 4: Documentation text
\newcommand{\texdocfield}[4]{\item \texttt{#1 #2 \textbf{#3}} \\ #4}

% Formats an enum element
%  Argument 1: name
%  Argument 2: documentation text
\newcommand{\texdocenum}[2]{\item \texttt{\textbf{#1}} \\ #2}

% Formats a single method
%  Argument 1: modifiers
%  Argument 2: return type
%  Argument 3: name
%  Argument 4: part after name (parameters)
%  Argument 5: Documentation text
%  Argument 6: Documentation of parameters/exceptions/return values
\newcommand{\texdocmethod}[6]{\item \texttt{#1 #2 \textbf{#3}#4} \\ #5#6}

% Formats a single constructor
%  Argument 1: modifiers
%  Argument 2: name
%  Argument 3: part after name (parameters)
%  Argument 4: Documentation text
%  Argument 5: Documentation of parameters/exceptions/return values
\newcommand{\texdocconstructor}[5]{\item \texttt{#1 \textbf{#2}#3} \\ #4#5}

% Inserted when @inheritdoc is used
%  Argument 1: Class where the documentation was inherited from
%  Argument 2: Documentation
\newcommand{\texdocinheritdoc}[2]{#2 (\textit{documentation inherited from \texttt{#1})}}

% Formats a single see-BlockTag
%  Argument 1: text
%  Argument 2: reference label
\newcommand{\texdocsee}[2]{\item \texttt{#1 (\ref{#2})}}

% Environment around \texdocparameter invocations
\newenvironment{texdocparameters}{%
	\minisec{Parameters}
	\begin{tabular}{ll}
}{%
	\end{tabular}
}

% Environment around \texdocthrow invocations
\newenvironment{texdocthrows}{%
        \minisec{Throws}
        \begin{tabular}{ll}
}{%
        \end{tabular}
}

\newcommand{\texdocreturn}[1]{\minisec{Returns} #1}

% Formats a parameter (this gets put inside the input of a \texdocmethod or
% \texdocconstructor macro)
%  Argument 1: name
%  Argument 2: description text
\newcommand{\texdocparameter}[2]{\texttt{\textbf{#1}} & \begin{minipage}[t]{0.8\textwidth}#2\end{minipage} \\}

% Formats a throws tag
%  Argument 1: exception name
%  Argument 2: description text
\newcommand{\texdocthrow}[2]{\texttt{\textbf{#1}} & \begin{minipage}[t]{0.6\textwidth}#2\end{minipage} \\}

% Used to simulate html <br/>
\newcommand{\texdocbr}{\mbox{}\newline{}}

% Used to simulate html <h[1-9]> - </h[1-9]>
% Argument 1: number of heading (5 for a <h5>)
% Argument 2: heading text
\newcommand{\headref}[2]{\minisec{#2}}

\newcommand{\refdefined}[1]{
\expandafter\ifx\csname r@#1\endcsname\relax
\relax\else
{$($ in \ref{#1}, page \pageref{#1}$)$}
\fi}

% ---------------------------------------------------------------------------
% TexDoc macros end
% ---------------------------------------------------------------------------




\begin{titlepage}

\subject{Entwurfsdokumentation}
\title{$\lambda$urora}
\subtitle{The Lambda Calculus IDE}


\author{Iuliia Patrusheva, Alexander von Heyden\\
Younis Bensalah, Max Nowak\\
Nikolai Polley, Randy Seng}

\end{titlepage}




\begin{document}
\maketitle
\pagebreak
%dontbreak
\setcounter{tocdepth}{4}
\tableofcontents

\newpage
\section{Einleitung}

\subsection{Zweck der Software}
Ziel der Software Aurora ist es, Studenten eine Lernumgebung zu bieten, in der sie mit eigenen Lambda Termen sowohl arbeiten als auch experimentieren können.
Hierdurch soll es den Studenten ermöglicht werden, ein besseres Gefühl für das $\lambda$-Kalkül zu entwickeln.
Hilfestellungen werden neben farbiger Hervorhebung der Termstrukturen auch durch eine anschauliche Darstellung von Zwischenschritte gegeben.
Darüberhinaus existiert eine vordefinierte Standardbibliothek, in der häufig benutzte Funktionen und Church Zahlen (in numerischer Schreibweise) vordefiniert sind.
\newline
Weitere Features sind unter anderem das einfache Erstellen von Lernmaterial durch einen \LaTeX-Export, als auch das Teilen des eigenen Inputs, inklusive dazugehöriger Benutzerbibliothek.

\subsection{Entwurfsziele}
Die Erstellung dieses Entwurfes hatte folgende Zielsetzungen:
\begin{description}
	
		\item[\textbf{Geheimhaltungsprinzip}] ~\par
			Klassen geben so wenig Informationen wie möglich über Implementierungsdetails preis.
			
		\item[\textbf{Starke Kohäsion in Pakete und Klassen}] ~\par
			Eine Klasse hat nur Methoden, die sie direkt betrifft. 
            Klassen innerhalb eines Paketes werden meist zusammen benutzt.
			
		\item[\textbf{Lose Koppelung zwischen Klassen und Paketen}]~\par
			Eine Klassen kennt so wenig wie mögliche andere Klassen.
			
		\item[\textbf{Lokalitätsprinzip beachten}] ~\par
			Eine Klasse soll auch alleine stehend verständlich sein.
            Die Klasse Term ist eine Ausnahme, da sie im Backend nahezu allen bekannt sein muss (siehe Abgrenzungskriterium \textbf{A3}).
            Deswegen ist die Klasse Term abstrakt, und sehr schlank gehalten.
			
		\item[\textbf{Keine Verwendung von instanceof}] ~\par
            Auch als sich \enquote{instanceof} als Entwurfsentscheidung sehr gut angeboten hat, haben wir eine andere Methode gewählt.
	
\end{description}
\pagebreak

\section{Grobentwurf}
\subsection{Architekturmuster und Systemzerlegung}
\subsubsection{Übersicht}
Bei dem Programm handelt es sich um eine clientseitige Anwendung, die auf \enquote{Google Chrome} 62 lauffähig ist.
Unter der Voraussetzung, dass in PSE eine objektorientierte Sprache verwendet werden muss, haben wir uns für Java 8 entschieden.
Damit unser Programm in Chrome ausgeführt werden kann, haben wir uns für die Verwendung des \enquote{Google Web Toolkits} entschieden (\enquote{GWT});
Dieses erlaubt es uns, Java Sourcecode nach Javascript zu übersetzen.
\newline
Aurora ist ein klassische aufgebautes Programm.
Somit besteht es aus einer GUI, in die der Benutzer seinen Input eingeben kann, und einem Backend, in dem alle benötigten Berechnungen durchgeführt werden. 
In der Vorlesung \enquote{Softwaretechnik I} wurde für dieses Szenario das Muster \enquote{Model-View-Controller} vorgestellt.
Im Gegensatz dazu haben wir uns allerdings für \enquote{Model-View-Presenter} entschieden, was vorallem daran liegt, dass \enquote{GWT} mit \enquote{MVP} im Kopf entworfen wurde.
Es wäre  zwar auch möglich gewesen \enquote{MVC} zu wählen, allerdings würden sich hier vor allem beim Schreiben von Testcases Schwierigkeiten auftun.
Die Verwendung von \enquote{MVP} erlaubt in großen Teilen das Ausführen von klassischen Java Tests;
Bei \enquote{MVC} wäre dies nicht so einfach möglich, hier müsste man die Tests speziell an \enquote{GWT} anpassen, was unter anderem zu einer längeren Laufzeit der Teste führen würde.

\subsubsection{Model-View-Presenter}
\enquote{MVP} ist ein Entwurfsmuster, welches aus drei Komponenten besteht, Model, View, und Presenter.
\begin{description}
	\item[Model] ~\par
	Das Model beinhaltet die Logik und alle Funktionalität der View.
    Im Gegensatz zum \enquote{MVC} hat hier das Model weder Zugriff auf die View, noch auf den Presenter.

	\item[View] ~\par
	Die View wird für das Darstellen der Ein- und Ausgaben benutzt.
    Somit ist die grafische Oberfläche des Programms ein Teil der View;
    Sie enthält aber keinerlei Logik und kennt weder das Model noch den Presenter.
	
	\item[Presenter] ~\par
	Der Presenter kennt sowohl View als auch das Model und verknüpft ihre Funktionen.
    Er steuert die Kommunikation zwischen den zwei Komponenten und koordiniert die logischen Abläufe.
\end{description}

\subsubsection{Lambda-Bibliothek}
Ein Teil des Projektes kann als eigene Lambda-Bibliothek betrachtet werden.
Im Code wird diese Eigenschaft nicht explizit sichtbar gemacht, lediglich ein eigenes Package weist darauf hin;
Bei Bedarf könnte es jedoch in eine externe Bibliothek ausgelagert, und anschließend in Aurora eingebunden werden.
Diese \enquote{Bibliothek} existiert, um den Umgang mit Lambda-Termen zu erleichtern.
Sie implementiert unter anderem Funktionalität zum Beta-Reduzieren, Substituieren von Variablen, Finden von Redexen, sowie verschiedene Auswertungsstrategien von Lambda-Termen.

\subsection{Paketbeschreibungen}
\begin{description}
	
	
	\item[aurora] ~\par
	Alle Pakete und Klassen sind in diesem Oberpaket definiert.
	
	\item[aurora.backend] ~\par
	Im Backend sind alle Berechnungen und Datenstrukturen, die für das Lambda Kalkül gebraucht werden. 
	
	\item[aurora.backend.parser] ~\par
	In diesem Paket wird eine Eingabe geparsed.
	
	\item[aurora.backend.parser.exceptions] ~\par
	Die Exceptions, die bei beim Parsen geworfen werden können.
	
	\item[aurora.backend.tree] ~\par
	Dies ist die grundlegende Datenstruktur, in der die Lambdaterme gespeichert werden.
	
	\item[aurora.backend.library] ~\par
	Dieses Paket wird für die Benutzer- und Standardbibliothek benutzt.
	
	\item[aurora.backend.librarby.exceptions] ~\par
	Die Exceptions, die bei der Library geworfen werden können.
	
	\item[aurora.backend.betareduction] ~\par
	Dieses Paket umfasst alles, was für die Beta-Reduktion benötigt wird.
	
	\item[aurora.backend.betareduction.visitors] ~\par
	In diesem Paket sind die Besucher, welche den Tree traversieren.
	Sie werden für die Beta-Reduktion benötigt.
	
	\item[aurora.backend.betareduction.strategies] ~\par
	Hier liegen alle Reduktionsstrategien die für die Beta-Reduktion benutzt werden können.
	
	\item[aurora.backend.encoders] ~\par
	Das Paket wird für das Encoden und Decoden des Inputs benutzt.
	
	\item[aurora.backend.encoders.exceptions] ~\par
	Die Exceptions, die beim Encoden/Decoden geworfen werden können.
	
	\item[aurora.backend.simplifier] ~\par
	In diesem Paket wird versucht, einen komplexer Term in eine simple Ausgabe zu überführen.
	
	\item[aurora.client] ~\par
	Hier sind alle Funktionen, die für den Client benötigt werden.
	
	\item[aurora.client.event] ~\par
	Das Paket beinhaltet alle Events, die ausgelöst werden können.
	
	\item[aurora.client.view] ~\par
	Hier liegt die GUI.
	
	\item[aurora.client.view.editor] ~\par
	Der Editor für die Benutzereingabe.
	
	\item[aurora.client.view.editor.actionbar] ~\par
	Die Actionbar des Editors.
	
	\item[aurora.client.view.popup] ~\par
	Dieses Paket beinhaltet alle Popups.
	
	\item[aurora.client.view.sidebar] ~\par
	An der linken Seite der GUI ist eine Sidebar.
	Sie wird von diesem Paket bereitgestellt.
	
	\item[aurora.client.view.sidebar.strategy] ~\par 
	Die Strategieauswahl in der Sidebar.
	
	\item[aurora.client.presenter] ~\par
	Der Presenter, der den Daten- und Kontrollfluss steuert.
	
\end{description}


\subsection{Entwurfsmuster}
Für das Traversieren der Terme wird das Visitor Pattern (Besucher) benutzt. Es werden jeweils konkrete Besucher erstellt, die jeweils unterschiedliche Funktionen besitzen um den Term auszuwerten.

Die Sidebar verwendet das Entwurfsmuster Kompositum.

Das Eventhandling wird mit Observern vollzogen.

%\pagebreak

\newpage
\section{Sequenzdiagramme}
\subsection{AppLaunch}
\includegraphics[width=0.75\textwidth]{../uml/SD/AppLaunch.png}
\subsection{Share LaTeX}
\includegraphics[width=0.75\textwidth]{../uml/SD/ShareLaTeX.png}
\subsection{Step interaction}
\includegraphics[width=0.75\textwidth]{../uml/SD/StepInteraction.png}
\subsection{Beta Reducer}
\includegraphics[width=0.75\textwidth]{../uml/SD/BetaReducer.png}


\newpage
\section{Feinentwurf}


%\newpage
\subsection{Klassen und Schnittstellen}
% include crap from texdoclet
%\tableofcontents
Do \textit{not} change or commit this file!

\pagebreak

\section{Änderungen zum Pflichtenheft}
    \begin{itemize}
        \item Das Wunschkriterium \textbf{K6. Pretty Print} ist in diesem Entwurf nicht mehr vorhanden.
            Das Team hat sich dafür entschieden, dass dieses Kriterium zuviel Zeit bei der Implementierung beanspruchen würde und das Feature nicht wirklich gewünscht war.

        \item Gleiches gilt für das Wunschkriterium \textbf{K19. Schleifenerkennung}.
            Dieses Feature hätte nur sehr selten Anwendung gefunden und wäre großteils unbenutzt geblieben.

        \item Auch ist das Wunschkriterium \textbf{K23. Lokale Sessions} gestrichen worden. 
            Zu viel Aufwand wäre in die Entwicklung dieses Features geflossen und wir haben beschlossen, dass dieses Feature den Aufwand nicht wert ist.

        \item Das Wunschkriterium \textbf{K25. Mobile-Darstellung} ist in dem jetzigen Entwurf nicht vorhanden.
            Die Gruppe lässt sich aber offen, falls noch Zeit bei der Implementierung übrig bleiben sollte, dieses Feature über das rückläufiges Wassermodell wieder einzufügen.

        \item Im Bereich Entwicklungsumgebung ist Bazel als \textbf{Build System} weggefallen, für \enquote{GWT} eignet sich Ant besser.
            Auch die \textbf{IDE} \enquote{Eclipse} wird von keinem Mitglied des Teams verwendet.

        \item Zur Wechseln zwischen dem Aurora Hauptfenster und dem Tutorial werden wir das von \enquote{GWT} bereitgestellte TabLayout anstatt der geplanten Hyperlinks verwenden.
    \end{itemize}


\section{UML Klassendiagramm}
\includepdf{UML_Klassen.pdf}


    
\end{document}

