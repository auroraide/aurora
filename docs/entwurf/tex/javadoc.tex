\begin{texdocpackage}{aurora.client.view.popup}
\label{texdoclet:aurora.client.view.popup}

\begin{texdocclass}{class}{InfoDialogBox}
\label{texdoclet:aurora.client.view.popup.InfoDialogBox}
\begin{texdocclassintro}
\end{texdocclassintro}
\begin{texdocclassconstructors}
\texdocconstructor{public}{InfoDialogBox}{()}{}{}
\end{texdocclassconstructors}
\end{texdocclass}


\begin{texdocclass}{class}{OptionsDialogBox}
\label{texdoclet:aurora.client.view.popup.OptionsDialogBox}
\begin{texdocclassintro}
\end{texdocclassintro}
\begin{texdocclassconstructors}
\texdocconstructor{public}{OptionsDialogBox}{()}{}{}
\end{texdocclassconstructors}
\end{texdocclass}


\begin{texdocclass}{class}{ShortLinkDialogBox}
\label{texdoclet:aurora.client.view.popup.ShortLinkDialogBox}
\begin{texdocclassintro}
\end{texdocclassintro}
\begin{texdocclassconstructors}
\texdocconstructor{public}{ShortLinkDialogBox}{()}{}{}
\end{texdocclassconstructors}
\begin{texdocclassmethods}
\texdocmethod{public}{void}{setShortLink}{(String shortLink)}{}{}
\end{texdocclassmethods}
\end{texdocclass}


\end{texdocpackage}



\begin{texdocpackage}{aurora.client.view.editor}
\label{texdoclet:aurora.client.view.editor}

\begin{texdocclass}{class}{Editor}
\label{texdoclet:aurora.client.view.editor.Editor}
\begin{texdocclassintro}
\end{texdocclassintro}
\begin{texdocclassconstructors}
\texdocconstructor{public}{Editor}{()}{}{}
\end{texdocclassconstructors}
\begin{texdocclassmethods}
\texdocmethod{public}{ActionBar}{getActionBar}{()}{}{}
\texdocmethod{public}{CodeMirrorPanel}{getCodeEditor}{()}{}{}
\texdocmethod{public}{OutputField}{getOutputField}{()}{}{}
\texdocmethod{public}{StepField}{getStepField}{()}{}{}
\end{texdocclassmethods}
\end{texdocclass}


\end{texdocpackage}



\begin{texdocpackage}{aurora.client.view.editor.components}
\label{texdoclet:aurora.client.view.editor.components}

\begin{texdocclass}{class}{ActionBar}
\label{texdoclet:aurora.client.view.editor.components.ActionBar}
\begin{texdocclassintro}
\end{texdocclassintro}
\begin{texdocclassconstructors}
\texdocconstructor{public}{ActionBar}{()}{}{}
\end{texdocclassconstructors}
\begin{texdocclassmethods}
\texdocmethod{public}{Button}{getResetButton}{()}{}{}
\texdocmethod{public}{Button}{getRunPauseContinueButton}{()}{}{}
\texdocmethod{public}{Button}{getStepButton}{()}{}{}
\end{texdocclassmethods}
\end{texdocclass}


\begin{texdocclass}{class}{InputField}
\label{texdoclet:aurora.client.view.editor.components.InputField}
\begin{texdocclassintro}
\end{texdocclassintro}
\begin{texdocclassconstructors}
\texdocconstructor{public}{InputField}{()}{}{}
\end{texdocclassconstructors}
\end{texdocclass}


\begin{texdocclass}{class}{OutputField}
\label{texdoclet:aurora.client.view.editor.components.OutputField}
\begin{texdocclassintro}
\end{texdocclassintro}
\begin{texdocclassconstructors}
\texdocconstructor{public}{OutputField}{()}{}{}
\end{texdocclassconstructors}
\end{texdocclass}


\begin{texdocclass}{class}{ResultField}
\label{texdoclet:aurora.client.view.editor.components.ResultField}
\begin{texdocclassintro}
\end{texdocclassintro}
\begin{texdocclassconstructors}
\texdocconstructor{public}{ResultField}{()}{}{}
\end{texdocclassconstructors}
\end{texdocclass}


\begin{texdocclass}{enum}{RPCButtonType}
\label{texdoclet:aurora.client.view.editor.components.RPCButtonType}
\begin{texdocclassintro}
\end{texdocclassintro}
\begin{texdocenums}
\texdocenum{CONTINUE}{}
\texdocenum{PAUSE}{}
\texdocenum{RUN}{}
\end{texdocenums}
\begin{texdocclassmethods}
\texdocmethod{public static}{RPCButtonType}{valueOf}{(String name)}{}{}
\texdocmethod{public static}{RPCButtonType}{values}{()}{}{}
\end{texdocclassmethods}
\end{texdocclass}


\begin{texdocclass}{class}{StepField}
\label{texdoclet:aurora.client.view.editor.components.StepField}
\begin{texdocclassintro}
\end{texdocclassintro}
\begin{texdocclassconstructors}
\texdocconstructor{public}{StepField}{()}{}{}
\end{texdocclassconstructors}
\begin{texdocclassmethods}
\texdocmethod{public}{void}{addLambdaTerm}{()}{}{}
\texdocmethod{public}{void}{clearStepsTable}{()}{}{}
\end{texdocclassmethods}
\end{texdocclass}


\end{texdocpackage}



\begin{texdocpackage}{aurora.client.view}
\label{texdoclet:aurora.client.view}

\begin{texdocclass}{class}{AuroraView}
\label{texdoclet:aurora.client.view.AuroraView}
\begin{texdocclassintro}
\end{texdocclassintro}
\begin{texdocclassconstructors}
\texdocconstructor{public}{AuroraView}{()}{}{}
\end{texdocclassconstructors}
\end{texdocclass}


\begin{texdocclass}{class}{TutorialView}
\label{texdoclet:aurora.client.view.TutorialView}
\begin{texdocclassintro}
\end{texdocclassintro}
\begin{texdocclassconstructors}
\texdocconstructor{public}{TutorialView}{()}{}{}
\end{texdocclassconstructors}
\end{texdocclass}


\end{texdocpackage}



\begin{texdocpackage}{aurora.client.view.sidebar.components.library}
\label{texdoclet:aurora.client.view.sidebar.components.library}

\begin{texdocclass}{class}{AddFunctionDialogBox}
\label{texdoclet:aurora.client.view.sidebar.components.library.AddFunctionDialogBox}
\begin{texdocclassintro}
\end{texdocclassintro}
\begin{texdocclassconstructors}
\texdocconstructor{public}{AddFunctionDialogBox}{()}{}{}
\end{texdocclassconstructors}
\begin{texdocclassmethods}
\texdocmethod{public}{Button}{getAddButton}{()}{}{}
\texdocmethod{public}{TextArea}{getDescriptionField}{()}{}{}
\texdocmethod{public}{TextBox}{getFunctionField}{()}{}{}
\texdocmethod{public}{TextBox}{getNameField}{()}{}{}
\end{texdocclassmethods}
\end{texdocclass}


\begin{texdocclass}{class}{LibraryTable}
\label{texdoclet:aurora.client.view.sidebar.components.library.LibraryTable}
\begin{texdocclassintro}
\end{texdocclassintro}
\begin{texdocclassconstructors}
\texdocconstructor{public}{LibraryTable}{()}{}{}
\end{texdocclassconstructors}
\end{texdocclass}


\begin{texdocclass}{class}{StandardLibraryTable}
\label{texdoclet:aurora.client.view.sidebar.components.library.StandardLibraryTable}
\begin{texdocclassintro}
\end{texdocclassintro}
\begin{texdocclassconstructors}
\texdocconstructor{public}{StandardLibraryTable}{()}{}{}
\end{texdocclassconstructors}
\end{texdocclass}


\begin{texdocclass}{class}{UserLibraryTable}
\label{texdoclet:aurora.client.view.sidebar.components.library.UserLibraryTable}
\begin{texdocclassintro}
\end{texdocclassintro}
\begin{texdocclassconstructors}
\texdocconstructor{public}{UserLibraryTable}{()}{}{}
\end{texdocclassconstructors}
\end{texdocclass}


\end{texdocpackage}



\begin{texdocpackage}{aurora.client.view.sidebar.components.strategy}
\label{texdoclet:aurora.client.view.sidebar.components.strategy}

\begin{texdocclass}{class}{StrategySelection}
\label{texdoclet:aurora.client.view.sidebar.components.strategy.StrategySelection}
\begin{texdocclassintro}
\end{texdocclassintro}
\begin{texdocclassconstructors}
\texdocconstructor{public}{StrategySelection}{()}{}{}
\end{texdocclassconstructors}
\begin{texdocclassmethods}
\texdocmethod{public}{RadioButton}{getCallByName}{()}{}{}
\texdocmethod{public}{RadioButton}{getCallByValue}{()}{}{}
\texdocmethod{public}{RadioButton}{getManualSelection}{()}{}{}
\texdocmethod{public}{RadioButton}{getNormalOrder}{()}{}{}
\end{texdocclassmethods}
\end{texdocclass}


\begin{texdocclass}{enum}{StrategyType}
\label{texdoclet:aurora.client.view.sidebar.components.strategy.StrategyType}
\begin{texdocclassintro}
\end{texdocclassintro}
\begin{texdocenums}
\texdocenum{CALLBYNAME}{}
\texdocenum{CALLBYVALUE}{}
\texdocenum{MANUALSELECTION}{}
\texdocenum{NORMALORDER}{}
\end{texdocenums}
\begin{texdocclassmethods}
\texdocmethod{public static}{StrategyType}{valueOf}{(String name)}{}{}
\texdocmethod{public static}{StrategyType}{values}{()}{}{}
\end{texdocclassmethods}
\end{texdocclass}


\end{texdocpackage}



\begin{texdocpackage}{aurora.client.view.sidebar.components.share}
\label{texdoclet:aurora.client.view.sidebar.components.share}

\begin{texdocclass}{class}{ShareLaTeXDialogBox}
\label{texdoclet:aurora.client.view.sidebar.components.share.ShareLaTeXDialogBox}
\begin{texdocclassintro}
\end{texdocclassintro}
\begin{texdocclassconstructors}
\texdocconstructor{public}{ShareLaTeXDialogBox}{()}{}{}
\end{texdocclassconstructors}
\end{texdocclass}


\end{texdocpackage}



\begin{texdocpackage}{aurora.client.view.sidebar}
\label{texdoclet:aurora.client.view.sidebar}

\begin{texdocclass}{class}{Sidebar}
\label{texdoclet:aurora.client.view.sidebar.Sidebar}
\begin{texdocclassintro}
\end{texdocclassintro}
\begin{texdocclassconstructors}
\texdocconstructor{public}{Sidebar}{()}{}{}
\end{texdocclassconstructors}
\end{texdocclass}


\end{texdocpackage}



\begin{texdocpackage}{aurora.client}
\label{texdoclet:aurora.client}

\begin{texdocclass}{class}{Aurora}
\label{texdoclet:aurora.client.Aurora}
\begin{texdocclassintro}
Entry point classes define \texttt{onModuleLoad()}.\end{texdocclassintro}
\begin{texdocclassconstructors}
\texdocconstructor{public}{Aurora}{()}{}{}
\end{texdocclassconstructors}
\begin{texdocclassmethods}
\texdocmethod{public}{void}{onModuleLoad}{()}{This is the entry point method.}{}
\end{texdocclassmethods}
\end{texdocclass}


\end{texdocpackage}



\begin{texdocpackage}{aurora.client.event}
\label{texdoclet:aurora.client.event}

\begin{texdocclass}{class}{AddFunctionEvent}
\label{texdoclet:aurora.client.event.AddFunctionEvent}
\begin{texdocclassintro}
\end{texdocclassintro}
\begin{texdocclassfields}
\texdocfield{public static}{}{TYPE}{}
\end{texdocclassfields}
\begin{texdocclassconstructors}
\texdocconstructor{public}{AddFunctionEvent}{(String name, String lambdaTerm, String description)}{}{}
\end{texdocclassconstructors}
\begin{texdocclassmethods}
\texdocmethod{protected}{void}{dispatch}{(AddFunctionEventHandler addFunctionEventHandler)}{}{}
\texdocmethod{public}{}{getAssociatedType}{()}{}{}
\texdocmethod{public}{String}{getDescription}{()}{}{}
\texdocmethod{public}{String}{getLambdaTerm}{()}{}{}
\texdocmethod{public}{String}{getName}{()}{}{}
\end{texdocclassmethods}
\end{texdocclass}


\begin{texdocclass}{interface}{AddFunctionEventHandler}
\label{texdoclet:aurora.client.event.AddFunctionEventHandler}
\begin{texdocclassintro}
\end{texdocclassintro}
\begin{texdocclassmethods}
\texdocmethod{public}{void}{onAddFunction}{(AddFunctionEvent event)}{}{}
\end{texdocclassmethods}
\end{texdocclass}


\begin{texdocclass}{class}{ContinueEvent}
\label{texdoclet:aurora.client.event.ContinueEvent}
\begin{texdocclassintro}
\end{texdocclassintro}
\begin{texdocclassfields}
\texdocfield{public static}{}{TYPE}{}
\end{texdocclassfields}
\begin{texdocclassconstructors}
\texdocconstructor{public}{ContinueEvent}{()}{}{}
\end{texdocclassconstructors}
\begin{texdocclassmethods}
\texdocmethod{protected}{void}{dispatch}{(ContinueEventHandler continueEventHandler)}{}{}
\texdocmethod{public}{}{getAssociatedType}{()}{}{}
\end{texdocclassmethods}
\end{texdocclass}


\begin{texdocclass}{interface}{ContinueEventHandler}
\label{texdoclet:aurora.client.event.ContinueEventHandler}
\begin{texdocclassintro}
\end{texdocclassintro}
\begin{texdocclassmethods}
\texdocmethod{public}{void}{onContinue}{(ContinueEvent event)}{}{}
\end{texdocclassmethods}
\end{texdocclass}


\begin{texdocclass}{class}{CopyToClipboardEvent}
\label{texdoclet:aurora.client.event.CopyToClipboardEvent}
\begin{texdocclassintro}
\end{texdocclassintro}
\begin{texdocclassfields}
\texdocfield{public static}{}{TYPE}{}
\end{texdocclassfields}
\begin{texdocclassconstructors}
\texdocconstructor{public}{CopyToClipboardEvent}{(String toBeCopied)}{}{}
\end{texdocclassconstructors}
\begin{texdocclassmethods}
\texdocmethod{protected}{void}{dispatch}{(CopyToClipboardEventHandler handler)}{Should only be called by HandlerManager (see \ref{texdoclet:HandlerManager}). In other words, do not use
 or call.}{\begin{texdocparameters}
\texdocparameter{handler}{handler}
\end{texdocparameters}
}
\texdocmethod{public}{}{getAssociatedType}{()}{}{}
\texdocmethod{public}{String}{getToBeCopied}{()}{}{}
\end{texdocclassmethods}
\end{texdocclass}


\begin{texdocclass}{interface}{CopyToClipboardEventHandler}
\label{texdoclet:aurora.client.event.CopyToClipboardEventHandler}
\begin{texdocclassintro}
\end{texdocclassintro}
\end{texdocclass}


\begin{texdocclass}{class}{EvaluationStrategyChangedEvent}
\label{texdoclet:aurora.client.event.EvaluationStrategyChangedEvent}
\begin{texdocclassintro}
\end{texdocclassintro}
\begin{texdocclassfields}
\texdocfield{public static}{}{TYPE}{}
\end{texdocclassfields}
\begin{texdocclassconstructors}
\texdocconstructor{public}{EvaluationStrategyChangedEvent}{(StrategyType strategyType)}{}{}
\end{texdocclassconstructors}
\begin{texdocclassmethods}
\texdocmethod{protected}{void}{dispatch}{(EvaluationStrategyChangedEventHandler evaluationStrategyChangedEventHandler)}{}{}
\texdocmethod{public}{}{getAssociatedType}{()}{}{}
\end{texdocclassmethods}
\end{texdocclass}


\begin{texdocclass}{interface}{EvaluationStrategyChangedEventHandler}
\label{texdoclet:aurora.client.event.EvaluationStrategyChangedEventHandler}
\begin{texdocclassintro}
\end{texdocclassintro}
\begin{texdocclassmethods}
\texdocmethod{public}{void}{onEvaluationStrategyChanged}{(EvaluationStrategyChangedEvent event)}{}{}
\end{texdocclassmethods}
\end{texdocclass}


\begin{texdocclass}{class}{ExportLatexEvent}
\label{texdoclet:aurora.client.event.ExportLatexEvent}
\begin{texdocclassintro}
\end{texdocclassintro}
\begin{texdocclassfields}
\texdocfield{public static}{}{TYPE}{}
\end{texdocclassfields}
\begin{texdocclassconstructors}
\texdocconstructor{public}{ExportLatexEvent}{(String lambdaTerm)}{}{}
\end{texdocclassconstructors}
\begin{texdocclassmethods}
\texdocmethod{protected}{void}{dispatch}{(ExportLatexEventHandler exportLatexEventHandler)}{}{}
\texdocmethod{public}{}{getAssociatedType}{()}{}{}
\texdocmethod{public}{String}{getLambdaTerm}{()}{}{}
\end{texdocclassmethods}
\end{texdocclass}


\begin{texdocclass}{interface}{ExportLatexEventHandler}
\label{texdoclet:aurora.client.event.ExportLatexEventHandler}
\begin{texdocclassintro}
\end{texdocclassintro}
\begin{texdocclassmethods}
\texdocmethod{public}{void}{onExportLatex}{(ExportLatexEvent event)}{}{}
\end{texdocclassmethods}
\end{texdocclass}


\begin{texdocclass}{class}{LanguageChangedEvent}
\label{texdoclet:aurora.client.event.LanguageChangedEvent}
\begin{texdocclassintro}
\end{texdocclassintro}
\begin{texdocclassfields}
\texdocfield{public static}{}{TYPE}{}
\end{texdocclassfields}
\begin{texdocclassconstructors}
\texdocconstructor{public}{LanguageChangedEvent}{()}{}{}
\end{texdocclassconstructors}
\begin{texdocclassmethods}
\texdocmethod{protected}{void}{dispatch}{(LanguageChangedEventHandler languageChangedEventHandler)}{}{}
\texdocmethod{public}{}{getAssociatedType}{()}{}{}
\end{texdocclassmethods}
\end{texdocclass}


\begin{texdocclass}{interface}{LanguageChangedEventHandler}
\label{texdoclet:aurora.client.event.LanguageChangedEventHandler}
\begin{texdocclassintro}
\end{texdocclassintro}
\begin{texdocclassmethods}
\texdocmethod{public}{void}{onLanguageChanged}{(LanguageChangedEvent event)}{}{}
\end{texdocclassmethods}
\end{texdocclass}


\begin{texdocclass}{class}{ManualEvaluationEvent}
\label{texdoclet:aurora.client.event.ManualEvaluationEvent}
\begin{texdocclassintro}
\end{texdocclassintro}
\begin{texdocclassfields}
\texdocfield{public static}{}{TYPE}{}
\end{texdocclassfields}
\begin{texdocclassconstructors}
\texdocconstructor{public}{ManualEvaluationEvent}{(String lambdaTerm)}{}{}
\end{texdocclassconstructors}
\begin{texdocclassmethods}
\texdocmethod{protected}{void}{dispatch}{(ManualEvaluationEventHandler manualEvaluationEventHandler)}{}{}
\texdocmethod{public}{}{getAssociatedType}{()}{}{}
\texdocmethod{public}{String}{getLambdaTerm}{()}{}{}
\end{texdocclassmethods}
\end{texdocclass}


\begin{texdocclass}{interface}{ManualEvaluationEventHandler}
\label{texdoclet:aurora.client.event.ManualEvaluationEventHandler}
\begin{texdocclassintro}
\end{texdocclassintro}
\begin{texdocclassmethods}
\texdocmethod{public}{void}{onManualEvaluation}{(ManualEvaluationEvent event)}{}{}
\end{texdocclassmethods}
\end{texdocclass}


\begin{texdocclass}{class}{OpenEmailClientEvent}
\label{texdoclet:aurora.client.event.OpenEmailClientEvent}
\begin{texdocclassintro}
\end{texdocclassintro}
\begin{texdocclassfields}
\texdocfield{public static}{}{TYPE}{}
\end{texdocclassfields}
\begin{texdocclassconstructors}
\texdocconstructor{public}{OpenEmailClientEvent}{(String url)}{}{}
\end{texdocclassconstructors}
\begin{texdocclassmethods}
\texdocmethod{protected}{void}{dispatch}{(OpenEmailClientEventHandler openEmailClientEventHandler)}{}{}
\texdocmethod{public}{}{getAssociatedType}{()}{}{}
\texdocmethod{public}{String}{getUrl}{()}{}{}
\end{texdocclassmethods}
\end{texdocclass}


\begin{texdocclass}{interface}{OpenEmailClientEventHandler}
\label{texdoclet:aurora.client.event.OpenEmailClientEventHandler}
\begin{texdocclassintro}
\end{texdocclassintro}
\begin{texdocclassmethods}
\texdocmethod{public}{void}{onOpenEmailClient}{(OpenEmailClientEvent event)}{}{}
\end{texdocclassmethods}
\end{texdocclass}


\begin{texdocclass}{class}{PauseEvent}
\label{texdoclet:aurora.client.event.PauseEvent}
\begin{texdocclassintro}
\end{texdocclassintro}
\begin{texdocclassfields}
\texdocfield{public static}{}{TYPE}{}
\end{texdocclassfields}
\begin{texdocclassconstructors}
\texdocconstructor{public}{PauseEvent}{()}{}{}
\end{texdocclassconstructors}
\begin{texdocclassmethods}
\texdocmethod{protected}{void}{dispatch}{(PauseEventHandler pauseEventHandler)}{}{}
\texdocmethod{public}{}{getAssociatedType}{()}{}{}
\end{texdocclassmethods}
\end{texdocclass}


\begin{texdocclass}{interface}{PauseEventHandler}
\label{texdoclet:aurora.client.event.PauseEventHandler}
\begin{texdocclassintro}
\end{texdocclassintro}
\begin{texdocclassmethods}
\texdocmethod{public}{void}{onPause}{(PauseEvent event)}{}{}
\end{texdocclassmethods}
\end{texdocclass}


\begin{texdocclass}{class}{ResetEvent}
\label{texdoclet:aurora.client.event.ResetEvent}
\begin{texdocclassintro}
\end{texdocclassintro}
\begin{texdocclassfields}
\texdocfield{public static}{}{TYPE}{}
\end{texdocclassfields}
\begin{texdocclassconstructors}
\texdocconstructor{public}{ResetEvent}{()}{}{}
\end{texdocclassconstructors}
\begin{texdocclassmethods}
\texdocmethod{protected}{void}{dispatch}{(ResetEventHandler resetEventHandler)}{}{}
\texdocmethod{public}{}{getAssociatedType}{()}{}{}
\end{texdocclassmethods}
\end{texdocclass}


\begin{texdocclass}{interface}{ResetEventHandler}
\label{texdoclet:aurora.client.event.ResetEventHandler}
\begin{texdocclassintro}
\end{texdocclassintro}
\begin{texdocclassmethods}
\texdocmethod{public}{void}{onReset}{(ResetEvent event)}{}{}
\end{texdocclassmethods}
\end{texdocclass}


\begin{texdocclass}{class}{RunEvent}
\label{texdoclet:aurora.client.event.RunEvent}
\begin{texdocclassintro}
\end{texdocclassintro}
\begin{texdocclassfields}
\texdocfield{public static}{}{TYPE}{}
\end{texdocclassfields}
\begin{texdocclassconstructors}
\texdocconstructor{public}{RunEvent}{()}{}{}
\end{texdocclassconstructors}
\begin{texdocclassmethods}
\texdocmethod{protected}{void}{dispatch}{(RunEventHandler runEventHandler)}{}{}
\texdocmethod{public}{}{getAssociatedType}{()}{}{}
\end{texdocclassmethods}
\end{texdocclass}


\begin{texdocclass}{interface}{RunEventHandler}
\label{texdoclet:aurora.client.event.RunEventHandler}
\begin{texdocclassintro}
\end{texdocclassintro}
\begin{texdocclassmethods}
\texdocmethod{public}{void}{onRun}{(RunEvent event)}{}{}
\end{texdocclassmethods}
\end{texdocclass}


\begin{texdocclass}{class}{ShortLinkEvent}
\label{texdoclet:aurora.client.event.ShortLinkEvent}
\begin{texdocclassintro}
\end{texdocclassintro}
\begin{texdocclassfields}
\texdocfield{public static}{}{TYPE}{}
\end{texdocclassfields}
\begin{texdocclassconstructors}
\texdocconstructor{public}{ShortLinkEvent}{(String rawInput, Library userLibrary)}{}{}
\end{texdocclassconstructors}
\begin{texdocclassmethods}
\texdocmethod{protected}{void}{dispatch}{(ShortLinkEventHandler shortLinkEventHandler)}{}{}
\texdocmethod{public}{}{getAssociatedType}{()}{}{}
\texdocmethod{public}{String}{getRawInput}{()}{}{}
\texdocmethod{public}{Library}{getUserLibrary}{()}{}{}
\end{texdocclassmethods}
\end{texdocclass}


\begin{texdocclass}{interface}{ShortLinkEventHandler}
\label{texdoclet:aurora.client.event.ShortLinkEventHandler}
\begin{texdocclassintro}
\end{texdocclassintro}
\begin{texdocclassmethods}
\texdocmethod{public}{void}{onShortLink}{(ShortLinkEvent event)}{}{}
\end{texdocclassmethods}
\end{texdocclass}


\begin{texdocclass}{class}{StepEvent}
\label{texdoclet:aurora.client.event.StepEvent}
\begin{texdocclassintro}
\end{texdocclassintro}
\begin{texdocclassfields}
\texdocfield{public static}{}{TYPE}{}
\end{texdocclassfields}
\begin{texdocclassconstructors}
\texdocconstructor{public}{StepEvent}{(int stepNumber)}{}{}
\end{texdocclassconstructors}
\begin{texdocclassmethods}
\texdocmethod{protected}{void}{dispatch}{(StepEventHandler stepEventHandler)}{}{}
\texdocmethod{public}{}{getAssociatedType}{()}{}{}
\texdocmethod{public}{int}{getStepNumber}{()}{}{}
\end{texdocclassmethods}
\end{texdocclass}


\begin{texdocclass}{interface}{StepEventHandler}
\label{texdoclet:aurora.client.event.StepEventHandler}
\begin{texdocclassintro}
\end{texdocclassintro}
\begin{texdocclassmethods}
\texdocmethod{public}{void}{onStep}{(StepEvent event)}{}{}
\end{texdocclassmethods}
\end{texdocclass}


\begin{texdocclass}{class}{StepValueChangedEvent}
\label{texdoclet:aurora.client.event.StepValueChangedEvent}
\begin{texdocclassintro}
\end{texdocclassintro}
\begin{texdocclassfields}
\texdocfield{public static}{}{TYPE}{}
\end{texdocclassfields}
\begin{texdocclassconstructors}
\texdocconstructor{public}{StepValueChangedEvent}{(int newStepNumber)}{}{}
\end{texdocclassconstructors}
\begin{texdocclassmethods}
\texdocmethod{protected}{void}{dispatch}{(StepValueChangedEventHandler stepValueChangedEventHandler)}{}{}
\texdocmethod{public}{}{getAssociatedType}{()}{}{}
\texdocmethod{public}{int}{getNewStepNumber}{()}{}{}
\end{texdocclassmethods}
\end{texdocclass}


\begin{texdocclass}{interface}{StepValueChangedEventHandler}
\label{texdoclet:aurora.client.event.StepValueChangedEventHandler}
\begin{texdocclassintro}
\end{texdocclassintro}
\begin{texdocclassmethods}
\texdocmethod{public}{void}{onStepValueChanged}{(StepValueChangedEvent event)}{}{}
\end{texdocclassmethods}
\end{texdocclass}


\end{texdocpackage}



\begin{texdocpackage}{aurora.client.presenter}
\label{texdoclet:aurora.client.presenter}

\begin{texdocclass}{class}{DesktopPresenter}
\label{texdoclet:aurora.client.presenter.DesktopPresenter}
\begin{texdocclassintro}
\end{texdocclassintro}
\begin{texdocclassconstructors}
\texdocconstructor{public}{DesktopPresenter}{()}{}{}
\end{texdocclassconstructors}
\begin{texdocclassmethods}
\texdocmethod{public}{void}{bind}{()}{}{}
\texdocmethod{public}{void}{go}{(Panel panel)}{}{}
\end{texdocclassmethods}
\end{texdocclass}


\begin{texdocclass}{interface}{DesktopPresenter.Display}
\label{texdoclet:aurora.client.presenter.DesktopPresenter.Display}
\begin{texdocclassintro}
\end{texdocclassintro}
\end{texdocclass}


\begin{texdocclass}{interface}{Presenter}
\label{texdoclet:aurora.client.presenter.Presenter}
\begin{texdocclassintro}
\end{texdocclassintro}
\begin{texdocclassmethods}
\texdocmethod{public}{void}{bind}{()}{}{}
\texdocmethod{public}{void}{go}{(Panel panel)}{}{}
\end{texdocclassmethods}
\end{texdocclass}


\begin{texdocclass}{class}{TutorialPresenter}
\label{texdoclet:aurora.client.presenter.TutorialPresenter}
\begin{texdocclassintro}
\end{texdocclassintro}
\begin{texdocclassconstructors}
\texdocconstructor{public}{TutorialPresenter}{()}{}{}
\end{texdocclassconstructors}
\end{texdocclass}


\end{texdocpackage}



\begin{texdocpackage}{aurora}
\label{texdoclet:aurora}

\begin{texdocclass}{class}{HelloLambda}
\label{texdoclet:aurora.HelloLambda}
\begin{texdocclassintro}
\end{texdocclassintro}
\begin{texdocclassconstructors}
\texdocconstructor{public}{HelloLambda}{()}{}{}
\end{texdocclassconstructors}
\begin{texdocclassmethods}
\texdocmethod{public static}{void}{main}{(String args)}{}{}
\end{texdocclassmethods}
\end{texdocclass}


\end{texdocpackage}



\begin{texdocpackage}{aurora.shared}
\label{texdoclet:aurora.shared}

\begin{texdocclass}{class}{FieldVerifier}
\label{texdoclet:aurora.shared.FieldVerifier}
\begin{texdocclassintro}
\begin{texdocp}
 FieldVerifier validates that the name the user enters is valid.
 \end{texdocp}
 \begin{texdocp}
 This class is in the \texttt{shared} package because we use it in both
 the client code and on the server. On the client, we verify that the name is
 valid before sending an RPC request so the user doesn't have to wait for a
 network round trip to get feedback. On the server, we verify that the name is
 correct to ensure that the input is correct regardless of where the RPC
 originates.
 \end{texdocp}
 \begin{texdocp}
 When creating a class that is used on both the client and the server, be sure
 that all code is translatable and does not use native JavaScript. Code that
 is not translatable (such as code that interacts with a database or the file
 system) cannot be compiled into client-side JavaScript. Code that uses native
 JavaScript (such as Widgets) cannot be run on the server.
 \end{texdocp}\end{texdocclassintro}
\begin{texdocclassconstructors}
\texdocconstructor{public}{FieldVerifier}{()}{}{}
\end{texdocclassconstructors}
\begin{texdocclassmethods}
\texdocmethod{public static}{boolean}{isValidName}{(String name)}{Verifies that the specified name is valid for our service.
 
 In this example, we only require that the name is at least four
 characters. In your application, you can use more complex checks to ensure
 that usernames, passwords, email addresses, URLs, and other fields have the
 proper syntax.}{\begin{texdocparameters}
\texdocparameter{name}{the name to validate}
\end{texdocparameters}
\texdocreturn{true if valid, false if invalid}
}
\end{texdocclassmethods}
\end{texdocclass}


\end{texdocpackage}



\begin{texdocpackage}{aurora.shared.backend.library}
\label{texdoclet:aurora.shared.backend.library}

\begin{texdocclass}{class}{Library}
\label{texdoclet:aurora.shared.backend.library.Library}
\begin{texdocclassintro}
Collection of lambda term definitions.\end{texdocclassintro}
\begin{texdocclassconstructors}
\texdocconstructor{public}{Library}{()}{Standard constructor.}{}
\end{texdocclassconstructors}
\begin{texdocclassmethods}
\texdocmethod{public}{void}{define}{(String name, String description, Term term)}{Create a new LibraryItem and add it to the Library.}{\begin{texdocparameters}
\texdocparameter{name}{The name of the Library item to be added.}
\texdocparameter{description}{The optional description of the Library item to be added.}
\texdocparameter{term}{The term of the Library item to be added.}
\end{texdocparameters}
}
\texdocmethod{public}{void}{define}{(Library.LibraryItem item)}{Add a LibraryItem to the Library.}{\begin{texdocparameters}
\texdocparameter{item}{The LibraryItem instance to be added.}
\end{texdocparameters}
}
\texdocmethod{public}{void}{define}{(Library library)}{Add the content of whole other Library to this instance.}{\begin{texdocparameters}
\texdocparameter{library}{}
\end{texdocparameters}
}
\texdocmethod{public}{Library.LibraryItem}{getItem}{(String name)}{Get the LibraryItem by name.}{\begin{texdocparameters}
\texdocparameter{name}{The name of the library item.}
\end{texdocparameters}
\texdocreturn{The LibraryItem associated with the given name.}
\begin{texdocthrows}
\texdocthrow{LibraryItemNotFoundException}{If there is no such entry in the library.}
\end{texdocthrows}
}
\texdocmethod{public}{void}{remove}{(String name)}{Remove a LibraryItem by name.}{\begin{texdocparameters}
\texdocparameter{name}{Name of the LibraryItem that should be removed.}
\end{texdocparameters}
}
\end{texdocclassmethods}
\end{texdocclass}


\begin{texdocclass}{class}{Library.LibraryItem}
\label{texdoclet:aurora.shared.backend.library.Library.LibraryItem}
\begin{texdocclassintro}
A single item (i.e., lambda term definition) defined in the library.\end{texdocclassintro}
\begin{texdocclassconstructors}
\texdocconstructor{public}{LibraryItem}{(String name, String description, Term term)}{Constructor that initializes a Library item.}{\begin{texdocparameters}
\texdocparameter{name}{The name of the library item.}
\texdocparameter{description}{An optional description.}
\texdocparameter{term}{The lambda term that defines this item.}
\end{texdocparameters}
}
\end{texdocclassconstructors}
\begin{texdocclassmethods}
\texdocmethod{public}{String}{getDescription}{()}{Get the library item description.}{\texdocreturn{The library item description.}
}
\texdocmethod{public}{String}{getName}{()}{Get the library item name.}{\texdocreturn{The library item name.}
}
\texdocmethod{public}{Term}{getTerm}{()}{Get the library item Term.}{\texdocreturn{The library item Term.}
}
\end{texdocclassmethods}
\end{texdocclass}


\end{texdocpackage}



\begin{texdocpackage}{aurora.shared.backend.library.exceptions}
\label{texdoclet:aurora.shared.backend.library.exceptions}

\begin{texdocclass}{class}{LibraryItemNotFoundException}
\label{texdoclet:aurora.shared.backend.library.exceptions.LibraryItemNotFoundException}
\begin{texdocclassintro}
Library item could not be found.\end{texdocclassintro}
\begin{texdocclassconstructors}
\texdocconstructor{public}{LibraryItemNotFoundException}{()}{}{}
\texdocconstructor{public}{LibraryItemNotFoundException}{(String message)}{}{}
\end{texdocclassconstructors}
\end{texdocclass}


\end{texdocpackage}



\begin{texdocpackage}{aurora.shared.backend.encoders.exceptions}
\label{texdoclet:aurora.shared.backend.encoders.exceptions}

\begin{texdocclass}{class}{DecodeException}
\label{texdoclet:aurora.shared.backend.encoders.exceptions.DecodeException}
\begin{texdocclassintro}
Something went wrong during decode.\end{texdocclassintro}
\begin{texdocclassconstructors}
\texdocconstructor{public}{DecodeException}{()}{}{}
\texdocconstructor{public}{DecodeException}{(String message)}{}{}
\end{texdocclassconstructors}
\end{texdocclass}


\end{texdocpackage}



\begin{texdocpackage}{aurora.shared.backend.encoders}
\label{texdoclet:aurora.shared.backend.encoders}

\begin{texdocclass}{class}{PastebinSessionEncoder}
\label{texdoclet:aurora.shared.backend.encoders.PastebinSessionEncoder}
\begin{texdocclassintro}
Save$/$restore sessions on {\bf pastebin.com} (at https:$/$$/$pastebin.com).\end{texdocclassintro}
\begin{texdocclassconstructors}
\texdocconstructor{public}{PastebinSessionEncoder}{()}{}{}
\end{texdocclassconstructors}
\begin{texdocclassmethods}
\texdocmethod{public}{SessionEncoder.Session}{decode}{(String encodedInput)}{}{}
\texdocmethod{public}{String}{encode}{(SessionEncoder.Session session)}{}{}
\end{texdocclassmethods}
\end{texdocclass}


\begin{texdocclass}{class}{SessionEncoder}
\label{texdoclet:aurora.shared.backend.encoders.SessionEncoder}
\begin{texdocclassintro}
Encode and decode facilities to save and restore sessions (i.e., raw lambda code along with Library entries).\end{texdocclassintro}
\begin{texdocclassconstructors}
\texdocconstructor{public}{SessionEncoder}{()}{}{}
\end{texdocclassconstructors}
\begin{texdocclassmethods}
\texdocmethod{public abstract}{SessionEncoder.Session}{decode}{(String encodedInput)}{Decode some previously encoded string.}{\begin{texdocparameters}
\texdocparameter{encodedInput}{The encoded string.}
\end{texdocparameters}
\texdocreturn{The decoded Session.}
\begin{texdocthrows}
\texdocthrow{DecodeException}{If the encoded input string could not be decoded.}
\end{texdocthrows}
}
\texdocmethod{public abstract}{String}{encode}{(SessionEncoder.Session session)}{Encode a Session to a string.}{\begin{texdocparameters}
\texdocparameter{session}{Session to be encoded.}
\end{texdocparameters}
\texdocreturn{Encoded string.}
}
\texdocmethod{public}{String}{encode}{(String rawInput, Library library)}{Encode raw input String along with a Library to a string.
 This is just a helper that creates the Session object for you.}{\begin{texdocparameters}
\texdocparameter{rawInput}{The raw input to be encoded.}
\texdocparameter{library}{The Library object to be encoded.}
\end{texdocparameters}
\texdocreturn{Encoded string.}
}
\end{texdocclassmethods}
\end{texdocclass}


\begin{texdocclass}{class}{SessionEncoder.Session}
\label{texdoclet:aurora.shared.backend.encoders.SessionEncoder.Session}
\begin{texdocclassintro}
A Session is lambda code (e.g., from user input) along with some Library context.\end{texdocclassintro}
\begin{texdocclassfields}
\texdocfield{public final}{Library}{library}{}
\texdocfield{public final}{String}{rawInput}{}
\end{texdocclassfields}
\begin{texdocclassconstructors}
\texdocconstructor{public}{Session}{(String rawInput, Library library)}{Construct a Session from raw input and Library instance.}{\begin{texdocparameters}
\texdocparameter{rawInput}{The raw input string.}
\texdocparameter{library}{The Library instance.}
\end{texdocparameters}
}
\end{texdocclassconstructors}
\end{texdocclass}


\end{texdocpackage}



\begin{texdocpackage}{aurora.shared.backend}
\label{texdoclet:aurora.shared.backend}

\begin{texdocclass}{class}{BetaReducer}
\label{texdoclet:aurora.shared.backend.BetaReducer}
\begin{texdocclassintro}
\end{texdocclassintro}
\begin{texdocclassconstructors}
\texdocconstructor{public}{BetaReducer}{(ReductionStrategy strategy)}{The constructor gets a strategy that is used for the reduction.}{\begin{texdocparameters}
\texdocparameter{strategy}{The chosen reduction strategy.}
\end{texdocparameters}
}
\end{texdocclassconstructors}
\begin{texdocclassmethods}
\texdocmethod{public}{Term}{reduce}{(Term term)}{This method performs one beta reduction.}{\begin{texdocparameters}
\texdocparameter{term}{The Term that will get reduced.}
\end{texdocparameters}
\texdocreturn{null if not reducible, otherwise reduced Term.}
}
\end{texdocclassmethods}
\end{texdocclass}


\begin{texdocclass}{class}{HighlightedLambdaExpression}
\label{texdoclet:aurora.shared.backend.HighlightedLambdaExpression}
\begin{texdocclassintro}
Encapsulates the lambda term combined with meta information about highlighting.\end{texdocclassintro}
\begin{texdocclassconstructors}
\texdocconstructor{public}{HighlightedLambdaExpression}{()}{Standard constructor.}{}
\texdocconstructor{public}{HighlightedLambdaExpression}{(Term t)}{Constructor.}{\begin{texdocparameters}
\texdocparameter{t}{}
\end{texdocparameters}
}
\end{texdocclassconstructors}
\begin{texdocclassmethods}
\texdocmethod{public}{void}{appendToken}{(HighlightedLambdaExpression.Token t)}{Add a Token to the Token list.}{}
\texdocmethod{public}{Iterator\textless{}HighlightedLambdaExpression.Token\textgreater{}}{iterator}{()}{}{}
\texdocmethod{public}{String}{toString}{()}{}{}
\end{texdocclassmethods}
\end{texdocclass}


\begin{texdocclass}{class}{HighlightedLambdaExpression.Token}
\label{texdoclet:aurora.shared.backend.HighlightedLambdaExpression.Token}
\begin{texdocclassintro}
A single token.\end{texdocclassintro}
\begin{texdocclassconstructors}
\texdocconstructor{public}{Token}{(HighlightedLambdaExpression.TokenType type, String name)}{Constructor.}{\begin{texdocparameters}
\texdocparameter{type}{}
\texdocparameter{name}{}
\end{texdocparameters}
}
\texdocconstructor{public}{Token}{(HighlightedLambdaExpression.TokenType type)}{Constructor with omitted name.}{\begin{texdocparameters}
\texdocparameter{type}{}
\end{texdocparameters}
}
\end{texdocclassconstructors}
\begin{texdocclassmethods}
\texdocmethod{public}{String}{getName}{()}{Get name.}{\texdocreturn{The name of this Token.}
}
\texdocmethod{public}{HighlightedLambdaExpression.TokenType}{getType}{()}{Get type.}{\texdocreturn{The type of this Token.}
}
\texdocmethod{public}{String}{toString}{()}{}{}
\end{texdocclassmethods}
\end{texdocclass}


\begin{texdocclass}{enum}{HighlightedLambdaExpression.TokenType}
\label{texdoclet:aurora.shared.backend.HighlightedLambdaExpression.TokenType}
\begin{texdocclassintro}
Token types.\end{texdocclassintro}
\begin{texdocenums}
\texdocenum{DOT}{}
\texdocenum{FUNCTION}{}
\texdocenum{LAMBDA}{}
\texdocenum{LEFT\_PARENTHESIS}{}
\texdocenum{NUMBER}{}
\texdocenum{RIGHT\_PARENTHESIS}{}
\texdocenum{VARIABLE}{}
\end{texdocenums}
\begin{texdocclassmethods}
\texdocmethod{public static}{HighlightedLambdaExpression.TokenType}{valueOf}{(String name)}{}{}
\texdocmethod{public static}{HighlightedLambdaExpression.TokenType}{values}{()}{}{}
\end{texdocclassmethods}
\end{texdocclass}


\begin{texdocclass}{class}{ShareLatex}
\label{texdoclet:aurora.shared.backend.ShareLatex}
\begin{texdocclassintro}
This class generates a latex snippet which can be pasted into a latex document.\end{texdocclassintro}
\begin{texdocclassconstructors}
\texdocconstructor{public}{ShareLatex}{(HighlightedLambdaExpression hle)}{This constructor gets a highlighted lambda expression so it can traverse the token list.}{\begin{texdocparameters}
\texdocparameter{hle}{The highlighted lambda expression.}
\end{texdocparameters}
}
\end{texdocclassconstructors}
\begin{texdocclassmethods}
\texdocmethod{public}{String}{generateLatex}{()}{This method creates a latex snippet as a string which can be copied into a latex document.}{\texdocreturn{A string which contains the latex code.}
}
\end{texdocclassmethods}
\end{texdocclass}


\begin{texdocclass}{class}{TreePath}
\label{texdoclet:aurora.shared.backend.TreePath}
\begin{texdocclassintro}
This class is a path that traverses a term and calculates an application.
 The path is a list.\end{texdocclassintro}
\begin{texdocclassconstructors}
\texdocconstructor{public}{TreePath}{()}{This constructor initializes an empty list.}{}
\end{texdocclassconstructors}
\begin{texdocclassmethods}
\texdocmethod{public}{Application}{get}{(Term term)}{A term gets traversed like the tree path says and returns the found application.}{\begin{texdocparameters}
\texdocparameter{term}{The term that will get traversed.}
\end{texdocparameters}
\texdocreturn{The application the treepath shows.}
}
\texdocmethod{public}{Iterator}{iterator}{()}{}{}
\texdocmethod{public}{void}{pop}{()}{This method deletes the last element of the list.}{}
\texdocmethod{public}{void}{push}{(TreePath.Direction d)}{This method adds a new enum to the List.}{\begin{texdocparameters}
\texdocparameter{d}{The enum left or right.}
\end{texdocparameters}
}
\end{texdocclassmethods}
\end{texdocclass}


\begin{texdocclass}{enum}{TreePath.Direction}
\label{texdoclet:aurora.shared.backend.TreePath.Direction}
\begin{texdocclassintro}
This enum is used to show the direction of the traversal of an application.\end{texdocclassintro}
\begin{texdocenums}
\texdocenum{LEFT}{}
\texdocenum{RIGHT}{}
\end{texdocenums}
\begin{texdocclassmethods}
\texdocmethod{public static}{TreePath.Direction}{valueOf}{(String name)}{}{}
\texdocmethod{public static}{TreePath.Direction}{values}{()}{}{}
\end{texdocclassmethods}
\end{texdocclass}


\end{texdocpackage}



\begin{texdocpackage}{aurora.shared.backend.simplifier}
\label{texdoclet:aurora.shared.backend.simplifier}

\begin{texdocclass}{class}{ChurchNumberSimplifier}
\label{texdoclet:aurora.shared.backend.simplifier.ChurchNumberSimplifier}
\begin{texdocclassintro}
Simplify a Term into is a ChurchNumber.\end{texdocclassintro}
\begin{texdocclassconstructors}
\texdocconstructor{public}{ChurchNumberSimplifier}{()}{}{}
\end{texdocclassconstructors}
\begin{texdocclassmethods}
\texdocmethod{public}{ChurchNumber}{simplify}{(Term t)}{}{}
\end{texdocclassmethods}
\end{texdocclass}


\begin{texdocclass}{class}{LibraryTermSimplifier}
\label{texdoclet:aurora.shared.backend.simplifier.LibraryTermSimplifier}
\begin{texdocclassintro}
Simplify a given Term into a LibraryTerm that is defined in some Library.\end{texdocclassintro}
\begin{texdocclassconstructors}
\texdocconstructor{public}{LibraryTermSimplifier}{(Library library)}{Constructor that takes a Library.}{\begin{texdocparameters}
\texdocparameter{library}{Library instance used for the lookup.}
\end{texdocparameters}
}
\end{texdocclassconstructors}
\begin{texdocclassmethods}
\texdocmethod{public}{LibraryTerm}{simplify}{(Term t)}{}{}
\end{texdocclassmethods}
\end{texdocclass}


\begin{texdocclass}{interface}{ResultSimplifier}
\label{texdoclet:aurora.shared.backend.simplifier.ResultSimplifier}
\begin{texdocclassintro}
Simplify a given result Term into some predefined Term.
 This can be helpful to get back to a more compact form (e.g., a number).\end{texdocclassintro}
\begin{texdocclassmethods}
\texdocmethod{public}{T}{simplify}{(Term t)}{Simplify a Term.}{\begin{texdocparameters}
\texdocparameter{t}{The term that you wish to simplify.}
\end{texdocparameters}
\texdocreturn{The simplified Term if possible or null.}
}
\end{texdocclassmethods}
\end{texdocclass}


\end{texdocpackage}



\begin{texdocpackage}{aurora.shared.backend.strategies}
\label{texdoclet:aurora.shared.backend.strategies}

\begin{texdocclass}{class}{CallByName}
\label{texdoclet:aurora.shared.backend.strategies.CallByName}
\begin{texdocclassintro}
This is the Call By Name Strategy. The Strategy reduces the leftmost redex, when not enclosed by an abstraction.
 This will be made by a depth first search, that doesn't go below abstractions.\end{texdocclassintro}
\begin{texdocclassconstructors}
\texdocconstructor{public}{CallByName}{()}{}{}
\end{texdocclassconstructors}
\begin{texdocclassmethods}
\texdocmethod{public}{TreePath}{getRedex}{(Term t)}{}{}
\end{texdocclassmethods}
\end{texdocclass}


\begin{texdocclass}{class}{CallByValue}
\label{texdoclet:aurora.shared.backend.strategies.CallByValue}
\begin{texdocclassintro}
This is the Call By Value strategy. It reduces an abstraction which has a "value" as it's parameter. A value is an abstraction or a free variable.\end{texdocclassintro}
\begin{texdocclassconstructors}
\texdocconstructor{public}{CallByValue}{()}{}{}
\end{texdocclassconstructors}
\begin{texdocclassmethods}
\texdocmethod{public}{TreePath}{getRedex}{(Term t)}{}{}
\end{texdocclassmethods}
\end{texdocclass}


\begin{texdocclass}{class}{NormalOrder}
\label{texdoclet:aurora.shared.backend.strategies.NormalOrder}
\begin{texdocclassintro}
The Normalorder is the default reduction strategy, it choses the leftmost redex.\end{texdocclassintro}
\begin{texdocclassconstructors}
\texdocconstructor{public}{NormalOrder}{()}{}{}
\end{texdocclassconstructors}
\begin{texdocclassmethods}
\texdocmethod{public}{TreePath}{getRedex}{(Term t)}{}{}
\end{texdocclassmethods}
\end{texdocclass}


\begin{texdocclass}{class}{ReductionStrategy}
\label{texdoclet:aurora.shared.backend.strategies.ReductionStrategy}
\begin{texdocclassintro}
This is the main class of the package "strategies". Every other strategy has to extend this class.\end{texdocclassintro}
\begin{texdocclassconstructors}
\texdocconstructor{public}{ReductionStrategy}{()}{}{}
\end{texdocclassconstructors}
\begin{texdocclassmethods}
\texdocmethod{public abstract}{TreePath}{getRedex}{(Term t)}{A strategy evaluates terms and choses a redex which gets reduced by the beta reducer. The Strategies return a treepath which show the chosen redex.}{\begin{texdocparameters}
\texdocparameter{t}{the term which gets evaluated.}
\end{texdocparameters}
\texdocreturn{the tree path to the chosen redex.}
}
\end{texdocclassmethods}
\end{texdocclass}


\begin{texdocclass}{class}{UserStrategy}
\label{texdoclet:aurora.shared.backend.strategies.UserStrategy}
\begin{texdocclassintro}
The User choses the Redex in this strategy.
 The user can chose from all possible redexes and clicks on one.
 The strategy calculates the Treepath to the chosen redex.\end{texdocclassintro}
\begin{texdocclassconstructors}
\texdocconstructor{public}{UserStrategy}{()}{}{}
\end{texdocclassconstructors}
\begin{texdocclassmethods}
\texdocmethod{public}{TreePath}{getRedex}{(Term t)}{}{}
\end{texdocclassmethods}
\end{texdocclass}


\end{texdocpackage}



\begin{texdocpackage}{aurora.shared.backend.parser}
\label{texdoclet:aurora.shared.backend.parser}

\begin{texdocclass}{class}{LambdaParser}
\label{texdoclet:aurora.shared.backend.parser.LambdaParser}
\begin{texdocclassintro}
Parser for lambda expressions.\end{texdocclassintro}
\begin{texdocclassconstructors}
\texdocconstructor{public}{LambdaParser}{()}{}{}
\end{texdocclassconstructors}
\begin{texdocclassmethods}
\texdocmethod{public}{Term}{parse}{(String code)}{Parse a lambda expression string into a tree of Terms.}{\begin{texdocparameters}
\texdocparameter{code}{The lambda expression as a string.}
\end{texdocparameters}
\texdocreturn{The root node of the corresponding Term tree if parsing was successful.}
\begin{texdocthrows}
\texdocthrow{SyntaxException}{In case of a syntax error.}
\texdocthrow{SemanticException}{In case of a semantic error.}
\end{texdocthrows}
}
\end{texdocclassmethods}
\end{texdocclass}


\end{texdocpackage}



\begin{texdocpackage}{aurora.shared.backend.parser.exceptions}
\label{texdoclet:aurora.shared.backend.parser.exceptions}

\begin{texdocclass}{class}{SemanticException}
\label{texdoclet:aurora.shared.backend.parser.exceptions.SemanticException}
\begin{texdocclassintro}
Thrown by Parser in case of a semantic error.\end{texdocclassintro}
\begin{texdocclassconstructors}
\texdocconstructor{public}{SemanticException}{()}{}{}
\texdocconstructor{public}{SemanticException}{(String message)}{}{}
\texdocconstructor{public}{SemanticException}{(String message, int line, int column)}{}{}
\end{texdocclassconstructors}
\begin{texdocclassmethods}
\texdocmethod{public}{int}{getColumn}{()}{Standard getter.}{\texdocreturn{The column of the semantic error.}
}
\texdocmethod{public}{int}{getLine}{()}{Standard getter.}{\texdocreturn{The line of the semantic error.}
}
\end{texdocclassmethods}
\end{texdocclass}


\begin{texdocclass}{class}{SyntaxException}
\label{texdoclet:aurora.shared.backend.parser.exceptions.SyntaxException}
\begin{texdocclassintro}
* Thrown by Parser in case of a syntax error.\end{texdocclassintro}
\begin{texdocclassconstructors}
\texdocconstructor{public}{SyntaxException}{()}{}{}
\texdocconstructor{public}{SyntaxException}{(String message)}{}{}
\texdocconstructor{public}{SyntaxException}{(String message, int line, int column)}{}{}
\end{texdocclassconstructors}
\begin{texdocclassmethods}
\texdocmethod{public}{int}{getColumn}{()}{Standard getter.}{\texdocreturn{The column of the syntax error}
}
\texdocmethod{public}{int}{getLine}{()}{Standard getter.}{\texdocreturn{The line of the syntax error.}
}
\end{texdocclassmethods}
\end{texdocclass}


\end{texdocpackage}



\begin{texdocpackage}{aurora.shared.backend.visitors}
\label{texdoclet:aurora.shared.backend.visitors}

\begin{texdocclass}{class}{RedexFinderVisitor}
\label{texdoclet:aurora.shared.backend.visitors.RedexFinderVisitor}
\begin{texdocclassintro}
Visitor that traverses the Term tree and identifies redexes.\end{texdocclassintro}
\begin{texdocclassconstructors}
\texdocconstructor{public}{RedexFinderVisitor}{()}{Standard constructor.}{}
\end{texdocclassconstructors}
\begin{texdocclassmethods}
\texdocmethod{public}{List}{getResult}{()}{Get list of found redexes.}{\texdocreturn{The list of TreePath objects that describe the locations of the redexes that have been found.}
}
\texdocmethod{public}{Void}{visit}{(Abstraction abs)}{}{}
\texdocmethod{public}{Void}{visit}{(Application app)}{}{}
\texdocmethod{public}{Void}{visit}{(BoundVariable bvar)}{}{}
\texdocmethod{public}{Void}{visit}{(FreeVariable fvar)}{}{}
\texdocmethod{public}{Void}{visit}{(LibraryTerm libterm)}{}{}
\texdocmethod{public}{Void}{visit}{(ChurchNumber c)}{}{}
\texdocmethod{public}{Void}{visit}{(Parenthesis p)}{}{}
\end{texdocclassmethods}
\end{texdocclass}


\begin{texdocclass}{class}{ReplaceVisitor}
\label{texdoclet:aurora.shared.backend.visitors.ReplaceVisitor}
\begin{texdocclassintro}
Visitor that allows replacing an Application with an arbitrary Term.\end{texdocclassintro}
\begin{texdocclassconstructors}
\texdocconstructor{public}{ReplaceVisitor}{(TreePath path, Term with)}{Constructor that initializes the ReplaceVisitor.}{\begin{texdocparameters}
\texdocparameter{path}{Location of the Application to be replaced.}
\texdocparameter{with}{The Term that the Application shall be replaced with.}
\end{texdocparameters}
}
\end{texdocclassconstructors}
\begin{texdocclassmethods}
\texdocmethod{public}{Term}{visit}{(Abstraction abs)}{}{}
\texdocmethod{public}{Term}{visit}{(Application app)}{}{}
\texdocmethod{public}{Term}{visit}{(BoundVariable bvar)}{}{}
\texdocmethod{public}{Term}{visit}{(FreeVariable fvar)}{}{}
\texdocmethod{public}{Term}{visit}{(LibraryTerm libterm)}{}{}
\texdocmethod{public}{Term}{visit}{(ChurchNumber c)}{}{}
\texdocmethod{public}{Term}{visit}{(Parenthesis p)}{}{}
\end{texdocclassmethods}
\end{texdocclass}


\begin{texdocclass}{class}{SubstitutionVisitor}
\label{texdoclet:aurora.shared.backend.visitors.SubstitutionVisitor}
\begin{texdocclassintro}
Visitor that traverses the Term tree and substitutes a BoundVariable with a given Term.\end{texdocclassintro}
\begin{texdocclassconstructors}
\texdocconstructor{public}{SubstitutionVisitor}{(Term with)}{This constructor gets a term. The index will automatically be 0.}{\begin{texdocparameters}
\texdocparameter{with}{The term that will get substituted.}
\end{texdocparameters}
}
\end{texdocclassconstructors}
\begin{texdocclassmethods}
\texdocmethod{public}{Term}{visit}{(Abstraction abs)}{}{}
\texdocmethod{public}{Term}{visit}{(Application app)}{}{}
\texdocmethod{public}{Term}{visit}{(BoundVariable bvar)}{}{}
\texdocmethod{public}{Term}{visit}{(FreeVariable fvar)}{}{}
\texdocmethod{public}{Term}{visit}{(LibraryTerm libterm)}{}{}
\texdocmethod{public}{Term}{visit}{(ChurchNumber c)}{}{}
\texdocmethod{public}{Term}{visit}{(Parenthesis p)}{}{}
\end{texdocclassmethods}
\end{texdocclass}


\begin{texdocclass}{interface}{TermVisitor}
\label{texdoclet:aurora.shared.backend.visitors.TermVisitor}
\begin{texdocclassintro}
Term tree Visitor with generic return type.\end{texdocclassintro}
\begin{texdocclassmethods}
\texdocmethod{public}{T}{visit}{(Abstraction abs)}{Called with Abstraction.}{\begin{texdocparameters}
\texdocparameter{abs}{The caller.}
\end{texdocparameters}
\texdocreturn{Result or null.}
}
\texdocmethod{public}{T}{visit}{(Application app)}{Called with Application.}{\begin{texdocparameters}
\texdocparameter{app}{The caller.}
\end{texdocparameters}
\texdocreturn{Result or null.}
}
\texdocmethod{public}{T}{visit}{(BoundVariable bvar)}{Called by BoundVariables.}{\begin{texdocparameters}
\texdocparameter{bvar}{The caller.}
\end{texdocparameters}
\texdocreturn{Result or null.}
}
\texdocmethod{public}{T}{visit}{(FreeVariable fvar)}{Called with FreeVariable.}{\begin{texdocparameters}
\texdocparameter{fvar}{The caller.}
\end{texdocparameters}
\texdocreturn{Result or null.}
}
\texdocmethod{public}{T}{visit}{(LibraryTerm libterm)}{Called with LibraryTerm.}{\begin{texdocparameters}
\texdocparameter{libterm}{The caller.}
\end{texdocparameters}
\texdocreturn{Result or null.}
}
\texdocmethod{public}{T}{visit}{(ChurchNumber c)}{Called with ChurchNumber.}{\begin{texdocparameters}
\texdocparameter{c}{The caller.}
\end{texdocparameters}
\texdocreturn{Result or null.}
}
\texdocmethod{public}{T}{visit}{(Parenthesis p)}{Called with Parenthesis.}{\begin{texdocparameters}
\texdocparameter{p}{The caller.}
\end{texdocparameters}
\texdocreturn{Result or null.}
}
\end{texdocclassmethods}
\end{texdocclass}


\end{texdocpackage}



\begin{texdocpackage}{aurora.shared.backend.tree}
\label{texdoclet:aurora.shared.backend.tree}

\begin{texdocclass}{class}{Abstraction}
\label{texdoclet:aurora.shared.backend.tree.Abstraction}
\begin{texdocclassintro}
An Abstraction is a "lambda", a "variable" a "." and a "body" in this order. The body is a Term.
 The "lambda" and the "." don't have to be saved, only the variable and the body.\end{texdocclassintro}
\begin{texdocclassconstructors}
\texdocconstructor{public}{Abstraction}{(Term body, String name)}{The Abstractions gets a body as a term. It also gets a string as a name for the variable.
 The name has to consist of lower case letters.}{}
\end{texdocclassconstructors}
\begin{texdocclassmethods}
\texdocmethod{public}{T}{accept}{(TermVisitor\textless{}T\textgreater{} visitor)}{}{}
\texdocmethod{public}{Term}{getBody}{()}{Standard getter, it returns the body.}{\texdocreturn{The body of the abstraction. It is a Term.}
}
\texdocmethod{public}{String}{getName}{()}{Standard getter, it returns the name of the variable.}{\texdocreturn{The name of the variable.}
}
\end{texdocclassmethods}
\end{texdocclass}


\begin{texdocclass}{class}{Application}
\label{texdoclet:aurora.shared.backend.tree.Application}
\begin{texdocclassintro}
This is the Application class. An application has a left and a right Term.
 Only Applications can be Redexes.\end{texdocclassintro}
\begin{texdocclassconstructors}
\texdocconstructor{public}{Application}{(Term left, Term right)}{The constructor gets a left and a right term.}{\begin{texdocparameters}
\texdocparameter{left}{The left term.}
\texdocparameter{right}{The right term.}
\end{texdocparameters}
}
\end{texdocclassconstructors}
\begin{texdocclassmethods}
\texdocmethod{public}{T}{accept}{(TermVisitor\textless{}T\textgreater{} visitor)}{}{}
\texdocmethod{public}{Term}{getLeft}{()}{A standard getter for the left term of the application.}{\texdocreturn{The left term of the application.}
}
\texdocmethod{public}{Term}{getRight}{()}{A standard getter for the right term of the application.}{\texdocreturn{The right term of the application.}
}
\end{texdocclassmethods}
\end{texdocclass}


\begin{texdocclass}{class}{BoundVariable}
\label{texdoclet:aurora.shared.backend.tree.BoundVariable}
\begin{texdocclassintro}
A bound variable is a variable, which is bound by an abstraction.
 It is possible to find the matching abstraction to the bounded variable.\end{texdocclassintro}
\begin{texdocclassconstructors}
\texdocconstructor{public}{BoundVariable}{(int index)}{This constructor gets an index which points to the matching abstraction. The index is commonly called De-Bruijn index.}{\begin{texdocparameters}
\texdocparameter{index}{the De-Bruijn Index.}
\end{texdocparameters}
}
\end{texdocclassconstructors}
\begin{texdocclassmethods}
\texdocmethod{public}{T}{accept}{(TermVisitor\textless{}T\textgreater{} visitor)}{}{}
\texdocmethod{public}{int}{getIndex}{()}{A standard getter that returns the index.}{\texdocreturn{The index.}
}
\end{texdocclassmethods}
\end{texdocclass}


\begin{texdocclass}{class}{ChurchNumber}
\label{texdoclet:aurora.shared.backend.tree.ChurchNumber}
\begin{texdocclassintro}
A church number is a representation of a number in lambda calculus.\end{texdocclassintro}
\begin{texdocclassconstructors}
\texdocconstructor{public}{ChurchNumber}{(int value)}{This constructor gets the value of the church number as a numerical value.}{\begin{texdocparameters}
\texdocparameter{value}{The value as Integer.}
\end{texdocparameters}
}
\end{texdocclassconstructors}
\begin{texdocclassmethods}
\texdocmethod{public}{T}{accept}{(TermVisitor\textless{}T\textgreater{} visitor)}{}{}
\texdocmethod{public}{Abstraction}{getAbstraction}{()}{This method takes the numerical number and returns the number as a lambda expression. Every number can be represented as an abstraction.}{\texdocreturn{the number which got converted into an abstraction.}
}
\texdocmethod{public}{int}{getValue}{()}{Standard getter, that returns the value of the church number.}{\texdocreturn{The value as an Integer.}
}
\end{texdocclassmethods}
\end{texdocclass}


\begin{texdocclass}{class}{FreeVariable}
\label{texdoclet:aurora.shared.backend.tree.FreeVariable}
\begin{texdocclassintro}
This class models a free variable. A free variable is not bound by an abstraciton.
 Every free variable has a name , which can only be lower case letters.\end{texdocclassintro}
\begin{texdocclassconstructors}
\texdocconstructor{public}{FreeVariable}{(String name)}{this constructor gets the name of the free variable as a string.}{\begin{texdocparameters}
\texdocparameter{name}{}
\end{texdocparameters}
}
\end{texdocclassconstructors}
\begin{texdocclassmethods}
\texdocmethod{public}{T}{accept}{(TermVisitor\textless{}T\textgreater{} visitor)}{}{}
\texdocmethod{public}{String}{getName}{()}{A standard getter, that returns the name of the free variable.}{\texdocreturn{the name as a string.}
}
\end{texdocclassmethods}
\end{texdocclass}


\begin{texdocclass}{class}{LibraryTerm}
\label{texdoclet:aurora.shared.backend.tree.LibraryTerm}
\begin{texdocclassintro}
This class represents the terms of the standard- and the userlibrary.
 Every LibraryTerm has a name which starts with a "\$" and then consists of lower case letters.\end{texdocclassintro}
\begin{texdocclassconstructors}
\texdocconstructor{public}{LibraryTerm}{(String name)}{The constructor of the class gets a String (which starts with a \$), which is used as the name of the library term.}{\begin{texdocparameters}
\texdocparameter{name}{The name of the library term.}
\end{texdocparameters}
}
\end{texdocclassconstructors}
\begin{texdocclassmethods}
\texdocmethod{public}{T}{accept}{(TermVisitor\textless{}T\textgreater{} visitor)}{}{}
\texdocmethod{public}{String}{getName}{()}{This is a standard getter, it returns the name of the library term.}{\texdocreturn{the name of the library term as a string.}
}
\end{texdocclassmethods}
\end{texdocclass}


\begin{texdocclass}{class}{Parenthesis}
\label{texdoclet:aurora.shared.backend.tree.Parenthesis}
\begin{texdocclassintro}
This is a Parenthesis which is around a term.\end{texdocclassintro}
\begin{texdocclassconstructors}
\texdocconstructor{public}{Parenthesis}{(Term inner)}{The constructor puts the parenthesis around a term.}{\begin{texdocparameters}
\texdocparameter{inner}{The parentheses are around this inner term.}
\end{texdocparameters}
}
\end{texdocclassconstructors}
\begin{texdocclassmethods}
\texdocmethod{public}{T}{accept}{(TermVisitor\textless{}T\textgreater{} visitor)}{}{}
\texdocmethod{public}{Term}{getInner}{()}{A standard getter, it returns the Term inside the parentheses.}{\texdocreturn{The Term inside the parentheses.}
}
\end{texdocclassmethods}
\end{texdocclass}


\begin{texdocclass}{class}{Term}
\label{texdoclet:aurora.shared.backend.tree.Term}
\begin{texdocclassintro}
The term class is the main class of the package "tree". Every other class in the package extends term.
 The Term class accepts a visitor and is the "Element" in the visitor pattern.
 Every other class that extends Term in this package is the "ConcreteElement" in the visitor pattern.\end{texdocclassintro}
\begin{texdocclassconstructors}
\texdocconstructor{public}{Term}{()}{}{}
\end{texdocclassconstructors}
\begin{texdocclassmethods}
\texdocmethod{public abstract}{T}{accept}{(TermVisitor\textless{}T\textgreater{} visitor)}{This method accepts a visitor}{\begin{texdocparameters}
\texdocparameter{visitor}{this is the visitor which is accepted by the term}
\end{texdocparameters}
\texdocreturn{todo}
}
\end{texdocclassmethods}
\end{texdocclass}


\end{texdocpackage}



