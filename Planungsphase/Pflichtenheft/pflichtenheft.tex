\documentclass[parskip=full,11pt,twoside]{scrartcl}
\usepackage[utf8]{inputenc}

\begin{titlepage}

\subject{Pflichtenheft}
\title{$\lambda$urora}
\subtitle{The Lambda Calculus IDE}


\author{Alexander von Heyden, Julia Patrusheva\\
 Max Nowak, Nikolai Polley\\
 Randy Seng, Younis Bensalah}

\newpage
\end{titlepage}

% section numbers in margins:
\renewcommand\sectionlinesformat[4]{\makebox[0pt][r]{#3}#4}

% header & footer
\usepackage{scrlayer-scrpage}
\lofoot{\today}
\refoot{\today}
\pagestyle{scrheadings}

\usepackage[sfdefault,light]{roboto}
\usepackage[T1]{fontenc}
\usepackage[german]{babel}
\usepackage[yyyymmdd]{datetime} % must be after babel
\renewcommand{\dateseparator}{-} % ISO8601 date format
\usepackage{hyperref}
\usepackage[nameinlink]{cleveref}
\crefname{figure}{Abb}{Abb}
\usepackage[section]{placeins}
\usepackage{xcolor}
\usepackage{graphicx}
\hypersetup{
	pdftitle={Pflichtenheft},
	bookmarks=true,
}
\usepackage{csquotes}

\usepackage{amsmath} % for $\text{}$
\newcommand\urlpart[2]{$\underbrace{\text{\texttt{#1}}}_{\text{#2}}$}
\usepackage{color}

\usepackage{pflichtenheft}




\begin{document}

\maketitle
\newpage
\tableofcontents

\newpage
\section{Einleitung}


\pagebreak
\section{Kriterien}
% Diese Section sollte kurz und knapp "für Manager" sein
% und auf eine Seite passen.


\subsection{Musskriterien}

\criterium{Eingabe}{crt:text-editor}
Es gibt ein Eingabefeld um $\lambda$-Terme einzugeben.

\criterium{Input editieren}{fnc:not-editable}
Der Input kann nur bearbeitet werden, solange das Programm nicht rechnet.

\criterium{Auswertung starten}{crt:evaluate-term}
Der Benutzer kann die Auswertung des eingegebenen $\lambda$-Terms starten. Im Ausgabefenster erscheint das Ergebnis in Normalform. 

\criterium{Auswertung stoppen}{crt:stop-feature}
Der Benutzer kann die Auswertung des $\lambda$-Terms jederzeit stoppen.

\criterium{Zwischenschritte}{crt:intermediate-step-output}
Während der Auswertung des $\lambda$-Terms können Zwischenergebnisse ausgegeben werden.

\criterium{Auswertungsstrategie}{crt:default-eval-strategy}
Die Auswertungsstrategie bei $\beta$-Reduktion von $\lambda$-Termen ist die Normalreihenfolge. Weitere Auswertungsstrategien sind in den Wunschkriterien.


\newpage
\subsection{Wunschkriterien}

\criteriumOptional{Zeilennummer}{crt:linenumbers}
Neben dem Eingabefeld soll ausgeblasst eine Zeilennumerierung dargestellt werden um eine einfache Zeilenzuteilung zu ermöglichen.

\criteriumOptional{Kommentare einfügen}{crt:comments}
Man kann mit \enquote{ \# } die jeweils restliche Zeile auskommentieren. Sie wird vom Parser übersprungen und nicht gelesen.

\criteriumOptional{Autovervollständigung von Klammern}{crt:autobracket}
Wenn der Benutzer eine Klammer öffnet \enquote{(} dann schließt das Programm automatisch diese Klammer mit \enquote{)}

\criteriumOptional{Syntax-Highlighting}{crt:syntax-highlighting}
Schlüsselwörter aus den Bibliotheken und $\lambda$s werden farblich hervorgehoben. Beim hovern über einer Klammer wird das entsprechende Klammerpaar farblich hervorgehoben.

\criteriumOptional{Erweitertes Syntax-Highlighting}{crt:syntax-highlighting-premium}
Der nächste auszuwertende Redex wird farblich hinterlegt, im nächsten Schritt in einem helleren Ton derselben Farbe.

\criteriumOptional{Standardbibliothek}{crt:std-library}
Bibliothek, in der vordefinierte Funktionen und Church-Zahlen verfügbar sind. Durch Schlüsselwörter kann der Benutzer diese direkt in seinen Code einbinden.

\criteriumOptional{Benutzerbibliothek}{crt:user-library}
Bibliothek, in der der Benutzer eigene Funktionen definieren und speichern kann. Durch Schlüsselwörter kann der Benutzer diese direkt in seinen Code einbinden.

\criteriumOptional{Weitere Auswertungsstrategien}{crt:opt-eval-strategy}
Zusätzliche Auswertungsstrategien sind \enquote{Call by Name} und \enquote{Call by Value}.

\criteriumOptional{Benutzerdefinierte Auswahl}{crt:own-eval-strategy}
Eigene Auswahl bei der $\beta$-Reduzierung eines Redex.

\criteriumOptional{Share-Button}{crt:share-feature}
Ein Share-Button ermöglicht das Erstellen von short-URLs sowie das Kopieren dieser in den Zwischenspeicher.

<<<<<<< Updated upstream
\criteriumOptional{LaTeX export}{crt:latex-export}
$\lambda$-Terme aus dem Text-Editor oder der Schrittliste können in \LaTeX-Snippet exportieren werden. Bis zu einer gewissen größe ist es auch möglich, den kompletten Rechenweg mit zu exportieren.
=======
<<<<<<< Updated upstream
\criteriumOptional{$\lambda$-Term vom Text-Editor oder Schrittliste in LaTeX-Snippet exportieren}{crt:latex-export}
=======
\criteriumOptional{LaTeX export}{crt:latex-export}
$\lambda$-Terme aus dem Text-Editor oder der Schrittliste können in \LaTeX-Snippet exportieren werden. Bis zu einer gewissen Größe ist es auch möglich, den kompletten Rechenweg mit zu exportieren.
>>>>>>> Stashed changes
>>>>>>> Stashed changes

\criteriumOptional{\enquote{Night Mode}}{crt:night-mode}
Die Funktion \enquote{Night Mode} ändert das äußere Erscheinungsbild der Applikation zu einem dunkleren Farbschema.

\criteriumOptional{Schleifenerkennung}{crt:loop-detection}
Das Programm soll einfache Schleifen wie z.B. ($\lambda$x. x x) ($\lambda$x. x x) erkennen. Insbesondere kann es aber weiterhin vorkommen, dass komplexere Schleifen nicht erkannt werden.

\criteriumOptional{$\lambda$urora Tutorial}{crt:tutorial}
Das Tutorial erklärt und zeigt dem Benutzer, wie man die Applikation $\lambda$urora nutzt.

\criteriumOptional{De Bruijn Indizes}{crt:debruijn-output}
Ausgabe des Programmes kann wahlweise mit De Bruijn Indizes erfolgen.

\criteriumOptional{Lokale Sessions}{crt:local-sessions}
Die Applikation merkt sich im localStorage des Webbrowsers die Historie von vorherigen Eingaben. Der Benutzer kann außerdem diese Historie verändern.

\criteriumOptional{Tastatur-Shortcuts}{crt:shortcuts}
$\lambda$urora stellt Tastatur-Shortcuts zum vereinfachten Ausführen von Features bereit.

\textbf{Verfügbare Tastatur-Shortcuts}

\begin{tabular}{|r|l|}
    
    \hline 
    shortcut & Beschreibung \\ \hline
    \textbackslash & $\lambda$ \\ \hline
    
\end{tabular}


% Hilft den Entwicklern beim Debuggen.













\subsection{Abgrenzungskriterien}

\criteriumNot{Keine Server-Komponente.}{crt:no-server-side}
\textcolor{green}{HIER FEHLT BESCHREIBUNG}

\criteriumNot{Berechnungen laufen Client-seitig.}{crt:client-side}
\textcolor{green}{HIER FEHLT BESCHREIBUNG}

\criteriumNot{Kein typisiertes $\lambda$-Kalkül.}{crt:no-typed-calculus}
\textcolor{green}{HIER FEHLT BESCHREIBUNG}

\criteriumNot{Applikation bietet keine Projekt-Unterstützung.}{crt:no-projects}
\textcolor{green}{HIER FEHLT BESCHREIBUNG}


\pagebreak
%%%%%%%%%%%%%%
\section{Produkteinsatz}
\subsection{Anwendungsbereiche}

Der Anwendungsbereich umfasst die Studenten und wissenschaftliche Mitarbeiter, die sich mit dem $\lambda$-Kalkül beschäftigen. Aurora erlaubt dem Benutzer $\lambda$-Terme zu berechnen und sich dabei einzelne Schritte der $\beta$-Reduktion anzeigen zu lassen. Durch vorgesehene Shortcuts und der Standardbibliothek mit Funktionen soll die Benutzung der Aurora Web App den Benutzer an eine normale IDE erinnern, was die Verwendung von Aurora erleichtern soll.

\subsection{Zielgruppen}

Eine Zielgruppe sind Informatik-Studenten und wissenschaftliche Mitarbeiter, die sich mit dem $\lambda$-Kalkül beschäftigen, die z.B. einen $\lambda$-Kalkül Term  üben oder einfach schnell ausrechnen wollen. Auch Interessierte des $\lambda$-Kalküls sollen das Programm benutzen können. Die Plattform ist für Einzelpersonen gedacht, eine Gruppenarbeit mit dem Programm ist nur eingeschränkt möglich. Die Aurora Web App soll von den Zielgruppen als Werkzeug benutzt werden, welches während des Studiums, wissenschaftlicher Arbeiten oder der privaten Freude am $\lambda$-Kalkül, den Benutzer unterstützt.

Für die Benutzung der Aurora Web App, werden Basiskenntnisse in Internetnutzung vorausgesetzt. Auch vorausgesetzt sind außerdem,  Kenntnisse über den untypisierten $\lambda$-Kalkül. \newline
Die Basissprache der Web App ist Englisch, daher sind Englisch Grundkenntnisse von Vorteil.

\subsection{Betriebsbedingungen}
Bezüglich den Betriebsbedingungen, soll sich Aurora nicht wesentlich von anderen Web Apps unterscheiden. Die maximale Anzahl der Benutzer von Aurora ist unbegrenzt bzw. vom Webserver abhängig. (Die Betriebsdauer soll 24 Stunden betragen.) 


\section{Produktumgebung}



%%%%%%%%%%%
\section{Funktionale Anforderungen}

\functionality{Stop}{fnc:stop-exec} \fulfills{crt:stop-feature}
Der Benutzer kann eine laufende Berechnung über einen \enquote{Stop}-Button vorzeitig beenden.

\functionality{No Timeout}{fnc:run-no-timeout}\fulfills{crt:infiniteloop}
Ein vom Benutzer eingegebenes Programm kann potentiell unendlich laufen.

\functionality{Historie}{fnc:localstorage-autosave}\fulfills{crt:local-sessions}
Eine ausgeführte Benutzereingabe, wird automatisch in die Historie eingefügt.

\functionality{Permanenz der Historie}{fnc:localstorage-permanent}\fulfills{crt:local-sessions}
Einträge in der Historie werden nach schließen des Browsers nicht gelöscht.

\functionality{Wiederherstellen der Historie}{fnc:localstorage-restore}\fulfills{crt:local-sessions}
Der Benutzer kann Eingaben aus der Historie wiederherstellen.

\functionality{Löschen der Historie}{fnc:localstorage-flush}\fulfills{crt:local-sessions}
Der Benutzer kann die Einträge in der Historie löschen.

\functionality{Benutzerbibliothek in Historie}{fnc:localstorage-userlib}\fulfills{crt:local-sessions}
Eine vom Benutzer definierte Funktion wird automatisch in die Historie eingefügt.

\functionality{Highlighting des nächsten auszuwertenden Redex}{fnc:highlight-next}\fulfills{crt:syntax-highlighting-premium}
In jedem Schritt wird im Schrittfeld nächste auszuwertende Redex farblich hinterlegt.

\functionality{Highlighting des zuletzt ausgewerteten Redex}{fnc:highlight-prev}\fulfills{crt:syntax-highlighting-premium}
In jedem Schritt wird der zuletzt eingesetzte Term farblich hinterlegt.

\functionality{Night Mode}{fnc:night-mode}\fulfills{crt:night-mode}
Das äußere Erscheinungsbild der Applikation kann über die \enquote{Night mode}-Funktion zu einem dunklen Farbschema geändert werden. Dies kann über das Hamburger-Menü ein- und ausgestellt werden.

\functionality{Standardbibliothek}{fnc:stdlib}\fulfills{crt:std-library}
Der Benutzer kann Funktionen aus der Standardbibliothek im Eingabefeld verwenden.

\functionality{Funktionen der Standardbibliothek}{fnc:func-name-def}\fulfills{crt:std-library}
Funktionen aus der Standardbibliothek haben einen Namen und eine Definition ($\lambda$-Term).

\functionality{Benutzerbibliothek}{fnc:userlib}\fulfills{crt:user-library}
Der Benutzer kann Funktionen zur Benutzerbibliothek hinzufügen.

\functionality{Funktionen der Benutzerbibliothek}{fnc:user-vs-std}\fulfills{crt:user-library}
Funktionen aus der Benutzerbibliothek können analog zu Funktionen der Standardbibliothek verwendet werden.

\functionality{Äquivalenz von Funktionen und deren Definition}{fnc:func-lambda-equivalent}\fulfills{crt:user-library}\fulfills{crt:std-library}
Die Eingabe von definierten Funktionen in der Standard- und Benutzerbibliothek ist semantisch äquivalent zur Eingabe deren entsprechenden Definition als $\lambda$-Term.

\functionality{Definition einer Funktion}{fnc:func-def}\fulfills{crt:std-library}\fulfills{crt:user-library}
Eine definierte Funktion der Standardbibliothek oder Benutzerbibliothek entspricht genau einem $\lambda$-Term.

\functionality{Eingabefeld}{fnc:input}\fulfills{crt:text-editor}
Der Benutzer kann $\lambda$-Terme über das Eingabefeld mit einer Tastatur eingeben. Die Zeichen erscheinen in Echtzeit im Eingabefeld.

\functionality{Eingabefeld bearbeiten}{fnc:editor}\fulfills{crt:text-editor}
Der Benutzer kann den Inhalt das Eingabefelds durch gängige Methoden bearbeiten.
Er kann mit der Maus Text auswählen, mit \enquote{Delete} Zeichen löschen usw.

\functionality{Run}{fnc:run}\fulfills{crt:evaluate-term}
Der Benutzer kann die Berechnung des $\lambda$-Terms über den \enquote{Run}-Button starten.

\functionality{Run Eingabefeld}{fnc:run-input}\fulfills{crt:evaluate-term}
Die \enquote{Run}-Funktion liest $\lambda$-Terme über das Eingabefeld ein.

\functionality{Die Eingabe kann nur geändert werden, falls keine Berechnung läuft.}{fnc:no-edit-while-running}\fulfills{crt:not-editable}
Der Benutzer kann das Eingabefeld nur genau dann editieren, wenn keine Berechnung am laufen ist.

\functionality{Ausgabefeld}{fnc:output}\fulfills{crt:evaluate-term}
Falls die Berechnung terminiert, steht das Ergebnis im Ausgabefeld.

\functionality{Ausgabefeld nicht editierbar}{fnc:output-readonly}\fulfills{crt:not-editable}
Der Inhalt des Ausgabefelds kann nicht bearbeitet werden.

\functionality{Normalreihenfolge}{fnc:eval-order-normal}\fulfills{crt:default-eval-strategy}
$\lambda$-Terme werden standardmäßig in Normalreihenfolge ausgewertet.
Das heißt, dass immer der linkeste Redex ausgewertet wird.

\functionality{Auswahl der Auswertungsstrategie}{fnc:eval-order}\fulfills{crt:opt-eval-strategy}
Der Benutzer kann die Auswertungsstrategie der $\lambda$-Terme auswählen.

\functionality{Auswertungsstrategien}{fnc:eval-order-options}\fulfills{crt:opt-eval-strategy}
Die zur Verfügung stehenden Auswertungsstrategien sind \enquote{Normalreihenfolge}, \enquote{Call-by-Name} und \enquote{Call-by-Value}.

\functionality{De Bruijin Indizes}{fnc:debruijn-output}
Ausgabe der $\lambda$-Ausdrücke mit De Bruijn Indizes.

\functionality{Schleifen erkennung}{fnc:loop-test}
Eine laufende Berechnung wird automatisch auf Schleifen überprüft.

\functionality{Share-Button}{fnc:share-button}
Der Benutzer kann mithilfe des Share-Buttons seinen Input in einer short-URL zum Teilen speichern.

\functionality{URL öffnen}{fnc:open-url}
Der Benutzer gibt eine mit ihm geteilte URL in die Adressleiste seines Browsers ein, das Programm kopiert den geteilten Code in das Eingabefeld.



\subsection{Eingabe}
\functionality{Textfeld}{fnc:output}
\functionality{Shortcuts}{fnc:output}
\functionality{Auto-complete}{fnc:output}
\functionality{Highlightning Lambda}{fnc:output}
\functionality{Highlightning Text}{fnc:output}

\functionality{Brackets Auto-complete}{fnc:bracket}
\fulfills{crt:autobracket}
Wenn der Benutzer eine Klammer öffnet \enquote{(} dann erkennt das Programm in Echtzeit, dass eine Klammer geöffnet wurde und fügt automatisch eine schließende Klammer \enquote{)} hinzu



\subsection{Darstellungen}
\functionality{DarkModus}{fnc:output}
\textcolor{green}{HIER FEHLT BESCHREIBUNG}

\functionality{Tutorial zur Web App}{fnc:output}
\textcolor{green}{HIER FEHLT BESCHREIBUNG}

\functionality{Highlighting vom Ergebnis}{fnc:output}
\textcolor{green}{HIER FEHLT BESCHREIBUNG}

\functionality{Highlighting von dem Redex, der als nächstes reduziert wird}{fnc:output}
\textcolor{green}{HIER FEHLT BESCHREIBUNG}

\functionality{Der Benutzer kann selber Redex auswählen, der als nächstes reduziert wird}{fnc:output}
\textcolor{green}{HIER FEHLT BESCHREIBUNG}

\functionality{Finales Ergebnis wird angezeigt}{fnc:output}
\textcolor{green}{HIER FEHLT BESCHREIBUNG}

\functionality{Nach Wunsch der User werden die $\beta$-Reduktionsschritte angezeigt}{fnc:output}
\textcolor{green}{HIER FEHLT BESCHREIBUNG}

\functionality{Sprache ist auswählbar}{fnc:language-selection}
Der Benutzer kann aus einem Satz von menschlichen Sprachen, die Sprache der Applikation einstellen.

\fulfills{crt:default-language}
\fulfills{crt:opt-language}

\functionality{GUI wird korrekt auf Google Chrome Web Browser angezeigt}{fnc:output}
\textcolor{green}{HIER FEHLT BESCHREIBUNG}

\functionality{Die Web App wird korrekt in Google Chrome Web Browser funktionieren}{fnc:output}
\textcolor{green}{HIER FEHLT BESCHREIBUNG}

\functionality{Der Stop Button endet die Berechnung}{fnc:output}
\textcolor{green}{HIER FEHLT BESCHREIBUNG}

\functionality{Darstellung der Berechnung mit De Bruijn Indizes}{fnc:debruijn-output}


\functionality{Mobile Darstellung ist im begrenzten Modus möglich}{fnc:output}
\textcolor{green}{HIER FEHLT BESCHREIBUNG}

\subsection{Hintergrundablauf}
\functionality{Eingabe wird geparsed}{fnc:output}
\textcolor{green}{HIER FEHLT BESCHREIBUNG}

\functionality{$\alpha$-Konversion mit De Bruijn Indices}{fnc:output}
\textcolor{green}{HIER FEHLT BESCHREIBUNG}

\functionality{Ermittlung der Eingestellten Auswertungsstrategie}{fnc:output}
\textcolor{green}{HIER FEHLT BESCHREIBUNG}

\functionality{Erkennung von offensichtlichen endlosen Schleifen}{fnc:output}
\functionality{$\beta$-Reduktion}{fnc:output}
\functionality{Das Program erkennt wenn keine weitere $\beta$-Reduktion möglich ist}{fnc:output}

\subsection{Speichern}
\functionality{Share Funktion von link}{fnc:output}
\textcolor{green}{HIER FEHLT BESCHREIBUNG}

\functionality{Session und Benutzer Bibliothek Speichern}{fnc:output}
\textcolor{green}{HIER FEHLT BESCHREIBUNG}

\functionality{Berechnung in LaTeX schreiweise umwandeln und zum Copy-to-Clipboard hinzufügen}{fnc:output}
\textcolor{green}{HIER FEHLT BESCHREIBUNG}

\functionality{Stop-Button}{fnc:but}
Wenn der Stop-button betätigt wird muss er ohne große Latenz und und ohne einfrieren des Programmes das Berechnen beenden können.


\functionality{Anschaufunktion auf mobilen Geräten}{fnc:gui-mobile-devices}
Die Funktionalität der Aurora Web App ist auf mobilen Geräten gegenüber der Desktop Version eingeschränkt.
%\functionality{Login-Möglichkeit auf Homepage}{fnc:login}
%\fulfills{crt:login}
%\fulfills{crt:github}

%Auf der Homepage \texttt{http://atu.rl/} sieht ein Besucher
%einen \enquote{Login via Facebook} Knopf.
%Weitere Knöpfe wie \enquote{Login via Github} sind möglich.
%Siehe \cref{fig:homepage}.



%%%%%%%%%%%
\section{Nicht-Funktionale Anforderungen}

\nonFunctionality{Benutzbarkeit}{nfc:usability}
Das Programm soll auch ohne dem Tutorial verständlich bedienbar sein.

\nonFunctionality{Aussehen}{nfc:appearence}
Die Grafische Oberfläche soll schlicht aber modern wirken. 

\nonFunctionality{Geschwindigkeit}{nfc:latency}
Das Programm soll sich schnell und responsiv anfühlen. Die Aktion die ein Button ausführt soll schnell erfolgen. 

\nonFunctionality{Programmsprache}{crt:default-language}
Die Standard Sprache des Programmes ist Englisch.

\nonFunctionality{Erweiterte Sprachenauswahl}{nfc:opt-language}
Das Programm ist durch außerdem in den Sprachen Deutsch, Russisch, Französisch verfügbar.

\nonFunctionality{Browserkompatibilität}{nfc:default-browser}
Das Programm funktioniert auf dem Webbrowser \enquote{Google Chrome} 62.

\nonFunctionality{Erweiterte Browserkompatibilität}{nfc:advances-default-browser}
Das Programm funktioniert auf aktuellen Webbrowsern, die Javascript unterstützen.

\nonFunctionality{nfc:supported-languages}
Die Programmarchitektur soll einfaches Hinzufügen neuer, menschlicher Sprachen unterstützen.

\nonFunctionality{Hash-Aufwand}{nfc:hash-time}
Die Berechnung der Hashwerte ist nicht mehr als 10\% der Gesamtrechenzeit in Anspruch.

\nonFunctionality{Standardbibliothek}{nfc:standard-library-content}
Die Standardbibliothek kann bis zu 20 vordefinierte Funktionen beinhalten.

\nonFunctionality{Benutzerbibliothek}{nfc:user-library-content}
Die Benutzerbibliothek kann bis zu 10 selbstdefinierte Funktionen beinhalten.

\nonFunctionality{Begrenzte Zwischenergebnisse}{nfc:limit-intermediate-results}
Die Anzahl der sichtbaren Zwischenergebnisse ist auf XXX beschränkt.

\nonFunctionality{Begrenzter LaTeX export}{nfc:limited-latex-export}
Der \LaTeX-Export für die komplette Rechnung kann nur ausgeführt werden, falls weniger als 20 Zwischenergebnisse vorliegen.



%%%%%%%%%%%
\section{Tests}


%\tests{fnc:impressum-link}
%\tests{fnc:datenschutz-link}
\teststep {Das Programm liegt auf einem Server}
{Der Benutzer tippt den link des Programms \texttt{aurora.kit.edu} in den Browser \enquote {Google Chrome} }
{ Die Webseite wird geöffnet und wird korrekt dargestellt je nach Browserauflösung}

\teststep {Das Programm ist geöffnet und es ist im Default-State}
{ Der Benutzer gibt in den Editor \enquote {($\lambda$ x.x) z } ein. Dannach drückt er auf den Button \enquote {Run}}
{ Das Programm führt eine $\beta$-Reduktion aus und gibt als Ausgabe \enquote {z} aus. Es wird auch angezeigt, dass es nur eine $\beta$-Reduktion durchgeführt hat.}

\teststep {Das Programm ist geöffnet und es ist im Default-State}
{Der Benutzer gibt den Term \enquote {$\lambda$ x.x} in das Eingabefeld ein und drückt auf Run}
{Das Programm kann keine $\beta$-Reduktion durchführen und gibt als Ausgabe $\lambda$ x.x}

\teststep {Das Programm ist geöffnet und findet sich im Default-State. An der rechten Seite ist die Standardfunktionsbibliothek implementiert }
{Der Benutzer gibt in den Editor \enquote {plus 2 2} ein und drückt auf den Buttorn \enquote {Run}}
{ Das Programm erkennt das Wort \enquote {plus} aus der Standardbibliothek und ersetzt es durch \enquote {$\lambda$n.$\lambda$m.$\lambda$s.$\lambda$z.n s (m s z)} die 2en erkennt er jeweils als die Churchzahl 2 und ersetzt sie durch \enquote {$\lambda$s.$\lambda$z.s(s z)}. Dann führt das Programm $\beta$-Reduktion aus und bekommt als Ergebnis\enquote {$\lambda$s.$\lambda$z. (s (s (s (s z))))} dies erkennt er ist die Churchzahl 4 und gibt dies aus. }

\teststep { Der vorherige Test wurde soeben durchgeführt}
{ Der Benutzer drückt auf die drei Punkte um alle Schritte zu sehen die bei der $\beta$-Reduktion durchgeführt wurden }
{ Das Programm zeigt folgende Schritte :
\newline ($\lambda$n.$\lambda$m.$\lambda$s.$\lambda$z.n s (m s z)) ($\lambda$s.$\lambda$z.s(s z)) ($\lambda$s.$\lambda$z.s(s z))
\newline ($\lambda$m.$\lambda$s.$\lambda$z.($\lambda$s.$\lambda$z.s(s z)) s (m s z)) ($\lambda$s.$\lambda$z.s(s z))
\newline $\lambda$s.$\lambda$z.($\lambda$s.$\lambda$z.s(s z)) s (($\lambda$s.$\lambda$z.s(s z)) s z)
\newline $\lambda$s.$\lambda$z.($\lambda$z.s(s z)) (($\lambda$s.$\lambda$z.s(s z)) s z)
\newline $\lambda$s.$\lambda$z. ( s ( s (($\lambda$s.$\lambda$z.s(s z)) s z) ))
\newline $\lambda$s.$\lambda$z. ( s ( s (($\lambda$z.s(s z)) z) ))
\newline $\lambda$s.$\lambda$z. ( s ( s ( s (s z))))
 }

\teststep{Das Programm ist im Default-State}
{der Benutzer gibt in den Editor ($\lambda$x.xx)($\lambda$x.xx) ein }
{ das Programm erkennt die Endlosschleife und warnt den Benutzer }

\teststep {Das Programm ist im Default-State}
{ Der Benutzer drückt auf den \enquote {Plus} Button und gibt \enquote {pair} und $\lambda$a.$\lambda$b.$\lambda$f.f a b ein }
{Wenn er auf den Button drückt öffnet sich ein Popup-Fenster in dem der Benutzer als Name der neuen Funktion Pair eingeben kann und als lambda term den von ihm gewünschten lambda term eingibt. Dieser Name wird mit dem Lambdaterm gespeichert und man kann im Editor anstelle des Lambdaterms den Namen eingeben. }

\teststep { Der vorherige Test wurde ausgeführt}
{ der Benutzer gibt in den Editor \enquote {first (pair a b)} ein. Dannach drückt er auf Run}
{ Das Programm erkennt first aus seiner Standardbibliothek und ihm ist $\lambda$p.p($\lambda$a.$\lambda$b.a) zugewiesen. Das Programm erkennt pair aus dem im letzten Testfall hinzugefügten Lambdaterm und kann nun $\beta$-Reduktion druchführen. Das Programm gibt als Ergebnis a aus.}

\teststep {Das Programm berechnet zurzeit einen Lambdaterm }
{Der Benutzer drückt auf den Stopp-Button}
{ Das Programm beendet die Berechnung und gibt kein Ergebnis aus }

\teststep {Im Editor steht der Term ($\lambda$t.$\lambda$f.f)(($\lambda$y.($\lambda$x.x x))(($\lambda$x.x)($\lambda$x.x)))($\lambda$t.$\lambda$f.f) }
{Der Benutzer benutzt den Step-Button einmal mit Normalreihenfolge  und einmal mit der Auswertungsstrategie Call-By-Value welche in den Einstellungen zu finden ist}
{ Bei der Normalreihenfolge wird der Redex ($\lambda$t.$\lambda$f.f) als erstes reduziert. bei der Auswertungsstrategie wird der Term ($\lambda$x.x) als erstes reduziert }

\teststep { Das Programm ist im Default-State}
{ Der Benutzer gibt ($\lambda$y.$\lambda$x.y x) x a in den Editor ein und drückt auf Run}
{ Durch die Alphakonversion mit De Bruijn Indizes ist das Ergebnis nicht a  a  }

\teststep {Das Programm hat einen Lambda-Term im Editor stehen} { Der Benutzer drückt auf einen Redex}
{ Dieser Redex wird als nächstes ausgewertet unabhängig davon welche Auswertungsstrategie ausgewählt ist. }

\teststep { Das Programm ist im Default-State}
{ Der Benutzer gibt den (($\lambda$x.x)( in den Editor ein }
{ Das Programm gibt die Fehlermeldung aus, dass dies kein gültiger LambdaTerm ist }

\teststep { Das Programm ist im Default -State }
{ Der Benutzer drückt shift l }
{Im Editor erscheint ein $\lambda$}

\teststep {Das Programm hat einen Lambdaterm ausgerechntet}
{ Der Benutzer drückt auf den Share button und wählt copy-link}
{ Eine gekürzte Url wird in sein Clipboard gespeichert }

\teststep {Der Benutzer hat von vorherigem Test eine Url erhalten}
{Der Benutzer gibt den Link in seine URL Leiste ein }
{ Das Programm öffnet sich mit dem LambdaTerm bereits in der Eingabe }

\teststep {Das Programm ist im Default-State}
{ Der Benutzer gibt sub 1 1 ein und drückt auf Run }
{ Das Programm erkennt die funktion substraction und die churchzahlen und gibt als ergebnis 0 aus. Da das Programm nicht erkennen kann ob es die Churchzahl 0 oder der boolean false ist gibt es immer 0 aus. Im Tutorial des Programmes wird das dem Benutzer erläutert }

\teststep {Das Programm ist im Default-State}
{Der Benutzer gibt ($\lambda$x.xx)($\lambda$x.xx) in den Editor ein.}
{Das Programm erkennt die Endlosschleife und terminiert.}

\teststep {Das Programm ist im Default-State}
{Der Benutzer drückt im Feld Benutzerbibliothek auf den Plus-Button.}
{Es öffnet sich ein Popup, indem der Benutzer eine Funktion definieren kann.}




%\teststep{Besucher \enquote{Jayne Cobb} ist auf der Homepage}
%{Er folgt dem Link mit dem Text \enquote{Datenschutz}}
%{Ein Text mit allen Datenschutzinformationen wird ihm angezeigt.}

%\teststep{}
%{Jayne folgt dem Link mit dem Text \enquote{Impressum}}
%{Ein Text mit Informationen des Betreibers wird ihm angezeigt.}

\section{Entwicklungsumgebung}
  \begin{description}
	\item[Benutzte Software die für das Erstellen des Programmes verwendet wurden]~\par
	\begin{itemize}
		\item \textbf{Programmiersprache}
		\begin{itemize}
			\item Java 8
		\end{itemize}
		\item \textbf{Build System}
		\begin{itemize}
			\item Bazel
		\end{itemize}
		\item \textbf{IDE}
		\begin{itemize}
			\item Jeder Entwickler wählt selber. Eclipse, Intellij oder Sublimetext
		\end{itemize}
		\item \textbf{Webframework}
		\begin{itemize}
			\item Google Web Toolkit \enquote{GWT}
		\end{itemize}
		\item \textbf{Versionskontrolle}
		\begin{itemize}
			\item Git auf scc.kit.edu. Webanwendung ist Gitlab
		\end{itemize}
		\item \textbf{Dokument Erstellung}
		\begin{itemize}
			\item LaTeX
		\end{itemize}
		\item \textbf{Testen}
		\begin{itemize}
			\item JUnit
			\item JaCoCo
		\end{itemize}
		\item \textbf{GUI Entwürfe}
			\begin{itemize}
				\item draw.io
			\end{itemize}
		\item \textbf{UMl und Diagramme zeichnen}
			\begin{itemize}
				\item UMLet
				\item Umbrella
				\item BOUML
				
			\end{itemize}
		\item \textbf{endgültige Programmiersprache auf der Website}
			\begin{itemize}
				\item Javascript
			\end{itemize}
	\end{itemize}
%%%%%%%%%%%%%
\newpage
\appendix
\end{description}
\section{Seitenentwürfe}
\textbf{GUI HIER}
% made via https://gomockingbird.com/projects/mnf0cwf/4gXVnC

%\begin{figure}[hb]
%\fbox{\includegraphics[width=\textwidth]{image/login.png}}
%\caption{\label{fig:homepage}
%Homepage mit Login-Funktion
%}
%\end{figure}


\section{Glossar}

%\textbf{Homepage}:
%Seite, die beim Besuchen der Betreiberdomain \emph{ohne Pfad} angezeigt wird. Auch %\enquote{Startseite}.
\textbf{Standardbibliothek}
Eine Ansammlung von Lambdatermen die einen Funktionsnamen bekommen haben. Die Entwickler von Aurora haben die wichtigsten Lambdaterme zusammen getragen damit der Benutzer häufig vorkommende Lambdaterme nicht selber tippen muss. Der Benutzer kann den Namen der Funktion im Editor verwenden und das Programm ersetzt den Namen automatisch in den gewählten Lambdaterm.

\textbf{Benutzerbibliothek}
Der Benutzer kann selber Funktionen mit Namen und Lambdaterm definieren. er kann dann den gewählten Namen im Editor verwenden und der Name wird automatisch durch den Lambdaterm ersetzt.

\textbf {Redex}
Redex steht für \enquote{reducible expression} oder auf deutsch \enquote{reduzibler Ausdruck} ist ein Subterm der durch die Auswertungsstrategien mit Beta-Reductionen reduzierbar sind.
Bei dem Term ($\lambda$x.x) y ist der ($\lambda$x.x) der Redex.

\textbf{Normalform}
Die Normalform ist ein $\lambda$-term der nicht mehr durch Betareduktion reduzierbar ist. Dies wird im Programm als \enquote{Result} ausgegeben

\textbf{Auswertungsstrategie}
Die Auswertungsstrategie gibt an welcher Redex als nächstes ausgewertet wird und dann mit Beta-Reduktion reduziert wird.

\textbf{Normalreihenfolge}
Der linkeste äußerste Redex wird als erstes ausgewertet und reduziert.

\textbf{Call-By-Name}
Der linkeste äußerste Redex der nicht von einem $\lambda$ umgeben ist wird als erstes ausgewertet und reduziert

\textbf{Call-By-Value}
Der linkeste Redex der nicht von einem $\lambda$ umgeben ist und dessen Argument ein Wert ist wird als erstes ausgewertet und reduziert.

\textbf{Hover}
Mit dem Mauszeiger über einem Element \enquote{schewben}, ohne es jedoch zu drücken.
 
\end{document}
